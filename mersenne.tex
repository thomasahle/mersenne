\documentclass[11pt]{article}  
%\renewcommand\baselinestretch{0.95}
\usepackage{amsmath}
\usepackage{amsthm}
\usepackage{amsfonts}
\usepackage{amssymb}

\usepackage{times}
\usepackage{fullpage}
\usepackage{epsfig}
\usepackage{graphicx}
\usepackage{epstopdf}
\usepackage{todonotes}
\usepackage{hyperref}
\usepackage{cleveref}
\usepackage{algorithm}
\usepackage[noend]{algpseudocode}
%\usepackage[ruled,vlined,commentsnumbered,titlenotnumbered]{algorithm2e}
\newcommand{\suchthat}{\mathrel{}\mathclose{}\ifnum\currentgrouptype=16\middle\fi\vert\mathopen{}\mathrel{}}

\DeclareMathOperator*{\E}{E}
\DeclareMathOperator*{\Var}{Var}

\newcommand{\ppmod}{\rule{-1.5ex}{0ex}\pmod}
\newcommand{\floor}[1]{\lfloor {#1} \rfloor}
\newcommand{\ceil}[1]{\lceil {#1}\rceil}
\newcommand{\Prp}[1]{\Pr\left[{#1} \right]}
\newcommand{\Ep}[1]{{\E}\left[{#1} \right]}
\newcommand{\req}[1]{(\ref{#1})}
\newcommand\eps\varepsilon
\newcommand\Z{\mathbb Z}

\newtheorem {lemma} {Lemma}[section]
\newtheorem {fact} [lemma] {Fact}
\newtheorem {property} {Property}
\newtheorem {definition} {Definition}
\newtheorem {corollary} [lemma] {Corollary}
\newtheorem {theorem}[lemma] {Theorem}
\newtheorem {observation}[lemma] {Observation}
\newtheorem {question}{Question}[section]
\newtheorem {exercise}[question]{Exercise}

\newcommand{\unif}{\mathcal{U}}


% Eva's comments in a different colour
\usepackage{color}
\usepackage{xcolor}
\newcommand{\er}[1]{\textcolor{blue}{#1}}
\newcommand{\erdel}[1]{\textcolor{LightGreen}{#1}}

% Drawing
\newcommand{\andtt}{ \mathbin{\texttt{\&}} }
\newcommand{\xor}{\oplus}
\newcommand{\ls}{ \mathbin{\texttt{<\!<}} }
\newcommand{\rs}{ \mathbin{\texttt{>\!>}} }


\title{The Power of Hashing with Mersenne Primes}
\author{Thomas Dybdahl Ahle, Jakob Tejs Bæk Knudsen, Mikkel Thorup}

\begin{document}
\maketitle

\begin{abstract}
The classic way of computing a $k$-universal hash function is to use a random degree-$(k-1)$ polynomial over a prime field $\mathbb Z_p$.
For a fast computation of the polynomial, the prime $p$ is often chosen as a Mersenne prime $p=2^b-1$.

In this paper, we show that there are other nice advantages to using Mersenne primes.
Our view is that the output of the hash function is a $b$-bit integer that is uniformly distributed in $[2^b]$, except that $p$ (the all \texttt1s value) is missing.
Uniform bit strings have many nice properties, such as splitting into substrings, which
%Thinking of the hash values as almost uniform $b$-bit integers
leads to simple efficient code with strong theoretical qualities.
We will demonstrate this with focus on the 4-universal hashing in the classic count-sketch for second moment estimation.

From an algorithmic perspective we provide a new algorithm for division and modulus with Pseudo-Mersenne primes
$p=2^b-c$ for small $c$,
which improves upon a classical algorithm of Crandall, and 
expands the availability and speed of Mersenne based techniques.
\end{abstract}

\tableofcontents

%! TEX root = ../mersenne.tex
\section{Introduction}

\begin{figure}
	\centering
	\begin{tikzpicture}[darkstyle/.style={circle,draw,fill=gray!40,minimum size=20}]
		\newcommand*{\figb}{8}
		% The red box
		\draw[pattern=north west lines, pattern color=red] (0,0) rectangle (\figb,1);
		% Horizontal lines
		\foreach \y in {0,...,6}
		\draw (0, \y) -- (\figb, \y);
		% Vertical lines
		\foreach \x in {0,...,\figb}
		\draw (\x, 0) -- (\x, 6);
		% Most of the numbers
		\pgfmathsetmacro{\figbthree}{\figb - 3}
		\pgfmathsetmacro{\figbtwo}{\figb - 2}
		\pgfmathsetmacro{\figbone}{\figb - 1}
		\foreach \y in {0,...,2}
		\foreach \x in {0,...,\figbthree}
		\node [draw=none] at (.5+\x,.5+\y) {1};
		\foreach \x in {\figbthree,...,\figbone}
		\node [draw=none] at (.5+\x,.5+3) {.};
		\foreach \y in {4,...,5}
		\foreach \x in {0,...,\figbtwo}
		\node [draw=none] at (.5+\x,.5+\y) {0};
		% The rest of the numbers
		\node [draw=none] at (.5+\figb-1,.5+0) {1};
		\node [draw=none] at (.5+\figb-2,.5+0) {1};
		\node [draw=none] at (.5+\figb-1,.5+1) {0};
		\node [draw=none] at (.5+\figb-2,.5+1) {1};
		\node [draw=none] at (.5+\figb-1,.5+2) {1};
		\node [draw=none] at (.5+\figb-2,.5+2) {0};
		\node [draw=none] at (.5+\figb-1,.5+4) {1};
		\node [draw=none] at (.5+\figb-1,.5+5) {0};
	\end{tikzpicture}
	\caption{The output of a random polynomial modulo $p=2^b-1$ is uniformly distributed in $[p]$, so each bit has the same distribution, which is only $1/p$ biased towards 0.}
	\label{fig:bits}
\end{figure}

The classic way to implement $k$-universal hashing is to use a random degree $(k-1)$-polynomial over a finite field \cite{wegman81kwise}.
Mersenne primes, which are prime numbers on the form $2^b-1$, have been used to implement finite fields efficiently for more than 40 years using standard portable code \cite{carter77universal}.

The speed of hashing is important because it is often an inner-loop
bottle-neck in data analysis. A good example is when hashing is used
in the sketching of high volume data streams, such as traffic through
an Internet router, and then this speed is critical to keep up with
the stream. A running example in this paper is the classic second
moment estimation using 4-universal hashing in count sketches
\cite{charikar04count-sketch}.
Count Sketches are linear maps that statistically preserve the Euclidean norm.
They are also popular in machine learning under the name
``Feature Hashing''~\cite{moody1989fast,weinberger2009feature}.

In this paper, we argue that uniform random values from Mersenne prime fields
%with prime $p=2^b-1$
are not only fast to compute but
\emph{have special advantages different from any other field.}
While it is natural to consider values mod $p=2^b-1$ as ``nearly'' uniform $b$-bit strings
(see \Cref{fig:bits}),
we show that the small bias in our hash values can usually be turned into an advantage.
In particular our analysis justify splitting single hash values into two or more for a significant computational speed-up, what we call the ``Two for one'' trick.

We also show that while the $1/p$ bias of such strings would usually result in relative errors of order $n/p$ for Count Sketch, a specialized analysis yields relative errors of just $n/p^2$.
The analysis is based on simple moments, and give similar improvements for any algorithm analyzed this way.
Loosely speaking, this means that we for a desired small error can reduce
the bit-length of the primes to less than half. This saves not only
space, it means that we can speed up the multiplications
with a factor of 2.

Finally we provide a fast, simple and branch-free algorithm for division and modulus with Mersenne primes.
Contrasting our analytic work, this code generalizes to so-called Pseudo-Mersenne primes~\cite{van2014encyclopedia} of the form $p=2^b-c$ for small $c$.
Our new code is simpler and faster than the classical algorithm of Crandall~\cite{crandall1992method}.

We provide experiments of both algorithms in \Cref{sec:experiments}.
For the rest of the introduction we will give a more detailed review of the new results.

\subsection{Hashing uniformly into b bits}\label{sec:b-bit?}
A main point in this paper is that having hash values uniform in $[2^b-1]=\{0,\dots,2^b-2\}$
is almost as good as having uniform $b$-bit strings, but of course,
it would be even better if we just had uniform $b$-bit strings.

We do have the fast multiply-shift scheme of Dietzfelbinger~\cite{dietzfel96universal}, which directly gives 2-universal
hashing from $b$-bit strings to $\ell$-bit strings, but for $k>2$,
there is no such fast $k$-universal hashing scheme that
can be implemented with standard portable code.

More recently it has been suggested to use carry-less multiplication
for $k$-universal hashing into bit strings (see, e.g., Lemire
\cite{lemire2014strongly}) but contrasting the hashing with Mersenne primes,
this is less standard (takes some work to get it to run on different
computers) and slower (by about 30-50\% for larger $k$ on the computers we tested in \Cref{sec:experiments}).
Moreover, the code for different bit-lengths $b$ is quite different
because we need quite different irreducible polynomials.

Another alternative is to use tabulation based methods which are fast
but use a lot of space \cite{Siegel04,Tho13:simple-simple}, that is,
space $s=2^{\Omega(b)}$ to calculate $k$-universal hash function in
constant time from $b$-bit keys to $\ell$-bit hash values. The large
space can be problematic.

A classic example where constant space hash functions are needed is in static two-level hash functions \cite{FKS84}.
To store n keys with constant access time, you use $n$ second-level hash tables, each with its own hash function.
Another example is small sketches such as the Count Sketch \cite{charikar04count-sketch} discussed in this paper.
Here we may want to store the hash function as part of the sketch, e.g., to query the value of a given key.
Then the hash value has to be directly computable from the small representation, ruling out tabulation based methods (see further explanation at the end of \Cref{sec:count-sketch}).

It can thus be problematic to get efficient $k$-universal hashing directly into
$b$-bit strings, and this is why we in this paper analyse the
hash values from Mersenne prime fields that are much easier to generate.

\subsection{Polynomial hashing using Mersenne primes}

Before discussing the special properties of Mersenne primes in algorithm analysis, we show how they are classically used to do fast field computations, and propose a new simple algorithm for further speed-ups in the hashing case.

The definition of $k$-universal hashing
goes back to Carter and Wegman~\cite{wegman81kwise}.
\begin{definition}
	A random hash function $h:U\to R$ is $k$-universal if for $k$
	distinct keys $x_0,\ldots,x_{k-1}\in U$, the $k$-tuple
	$(h(x_0),\ldots,h(x_{k-1}))$ is uniform in $R^k$.
\end{definition}
\noindent
Note that the definition also implies the values
$h(x_0),\ldots,h(x_{k-1})$ are independent.
A very similar concept is that of $k$-independence, which has only this requirement but doesn't include that values must be uniform.

For $k>2$ the standard $k$-universal hash function is uniformly random degree-$(k-1)$ polynomial over a prime field
$\Z_p$, that is, we pick a uniformly random vector
$\vec a=(a_0,\ldots,a_{k-1})\in \Z_p^k$ of $k$ coefficients, and define
$h_{\vec a}:[p]\to[p]$,
\footnote{ We use the notation $[s]=\{0,\ldots,s-1\}$.  }
by
\[h_{\vec a}(x)=\sum_{i\in[k]}a_i x^i \mod p.\]
%
Given a desired key domain $[u]$ and range $[r]$ for the hash values, we pick
$p\geq \max\{u,r\}$ and define
$h^r_{\vec a}:[u]\to[r]$ by
\[h^r_{\vec a}(x)=h_{\vec a}(x)\bmod r.\]
The  hash values of $k$ distinct keys remain independent while staying as close as possible to the uniform distribution on $[r]$.
(This will turn out to be very important.)

In terms of speed, the main bottleneck in the above approach is the mod operations.
If we assume $r=2^\ell$, the mod $r$ operation above can be replaced by a binary {\sc and} (\texttt{\&}): $x \bmod r = x \andtt r-1$.
Similarly, Carter and Wegman \cite{carter77universal} used a
Mersenne prime $p=2^b-1$,\footnote{e.g., $p=2^{61}-1$ for hashing 32-bit keys or
$p=2^{89}-1$ for hashing 64-bit keys.}
to speed up the computation of the (mod $p$) operations:
\begin{equation}
	y% \bmod (2^b-1)
	\equiv y - \floor{y/2^b}(2^b-1)
	= (y\bmod 2^{b}) + \floor{y/2^b}
	%   \equiv (y \andtt p) + (y \rs b)
	\pmod {p}.
	\label{eq:Mersenne}
\end{equation}
Again allowing us to use the very fast bit-wise {\sc and} ($\andtt$) and the right-shift ($\rs$),
instead of the expensive modulo operation.

Of course, \eqref{eq:Mersenne} only reduces $y$ to an equivalent value mod $p$, not to the smallest one, which is what we usually want.
For this reason one typically adds a test ``if $y \ge p$ then $y \gets y - p$''.
We show an implementation in \Cref{alg:Mersenne} below with one further improvement:
By assuming that $p=2^b-1\geq 2u-1$
(which is automatically satisfied in the typical case where $u$ is a power
of two, e.g., $2^{32}$ or $2^{64}$)
we can get away with only doing this test once, rather than at every loop.
Note the proof by loop invariant in the comments.

\begin{algorithm}[H]
	\caption{
	For $x\in [u]$, prime $p=2^b-1\geq 2u-1$,
	and $\vec a=(a_0,\ldots,a_{k-1})\in[p]^k$,
	computes $y=h_{\vec a}(x)=\sum_{i\in[k]}a_i x^i\mod p$.
	}\label{alg:Mersenne}
	\begin{algorithmic}
		\State $y\gets a_{k-1}$
		\For{$i=q-2,\ldots,0$}
		\Comment{Invariant: $\quad y<2p$}

		\State $y\gets y*x+a_i$
		\Comment{$\quad y<2p(u-1)+(p-1)<(2u-1)p\leq p^2$}

		\State $y\gets (y\andtt p)+(y\rs b)$
		\Comment{$\quad y<p+p^2/2^b<2p$}
		\EndFor
		\If{$y\geq p$}
		\State $y\gets y-p$
		\Comment{$y<p$}
		\EndIf
		%\State \Return $y$
	\end{algorithmic}
\end{algorithm}


In \Cref{subsec:intro-division} we will give one further improvement to \Cref{alg:Mersenne}.
In the next sections we will argue that Mersenne primes are not only fast, but have special properties not found in other finite fields.

%The main point of this paper is our note from before, that the values hashed to $[r]$ are not completely uniform, as they would have been if $p=2^b$ was a prime.
%It turns out that with a novel analysis, bits from Mersenne primes are actually
%almost as good as if
%they were uniformly distributed $b$-bit strings (we are only missing
%the all \texttt{1}s value $2^b-1$). 


\subsubsection{Selecting arbitrary bits}\label{sec:power-of-two}
If we had b uniform bits, we could partition them any way we’d like and get smaller independent
strings of uniform random bits. The first property of random values modulo Mersenne primes
we discuss is what happens when the same thing is done on a random value in $[2^b - 1]$ instead.

More formally, let $\mu:[2^b]\to[2^\ell]$ be any map that selects
$\ell$ distinct bits, that is, for some $0\leq j_1<\cdots<j_{\ell}<b$,
$\mu(y)=y_{j_1}\cdots y_{j_\ell}$. For example, if $j_i=i-1$, then we
are selecting the most significant bits, and then $\mu$ can be
implemented as $y\mapsto y\rs (b-\ell)$. Alternatively, if $j_i=b-i$,
then we are selecting the least significant bits, and then $\mu$ can
be implemented as $y\mapsto y\andtt (2^\ell-1)=y\andtt (r-1)$.


We assume a $k$-universal hash function $h:[u]\to[p]$, e.g.,
the one from \Cref{alg:Mersenne}. To get hash values in $[r]$,
we use $\mu\circ h$. Since $\mu$ is deterministic,
the hash values of up to $k$ distinct keys remain
independent with $\mu\circ h$. The issue is that hash values from
$\mu\circ h$ are not quite uniform in $[r]$.

Recall that for any key $x$, we have $h(x)$ uniformly distributed in $[2^b-1]$.
This is the uniform distribution on $b$-bit strings except that we are
missing $p=2^b-1$. Now $p$ is the all \texttt{1}s, and
$\mu(p) = r-1$.
Therefore
\begin{align}
	\text{for $i < r-1$,}\quad
	\Pr[\mu(h(x))=i]
	 & =\lceil p/r\rceil/p
	=((p+1)/r)/p
	=(1+1/p)/r
	\label{eq:coll-ell<r-1}
	\\
	\text{and}\quad
	\Pr[\mu(h(x))=r-1]
	 & =\lfloor p/r\rfloor/p=((p+1-r)/r)/p
	=(1-(r-1)/p)/r.
	\label{eq:coll-ell=r-1}
\end{align}
Thus $\Pr[\mu(h(x))=i]\leq (1+1/p)/r$ for all $i\in[r]$.
This upper-bound only has a relative error of $1/p$ from the uniform $1/r$.

Combining \eqref{eq:coll-ell<r-1} and \eqref{eq:coll-ell=r-1} with
pairwise independence, for any distinct keys $x,y\in [u]$, we show that the
collision probability is bounded
\begin{align}
	\Pr[\mu(h(x))=\mu(h(y))]
	 & =(r-1)((1+1/p)/r)^2+((1-(r-1)r/p)/r)^2 \nonumber
	%\\&= \frac{r +(r^2-r)/p^2}{r^2}
	\\&=(1+(r-1)/p^2)/r
	%\\& <(1+r/p^2)/r
	.\label{eq:coll}
\end{align}
Thus the relative error $r/p^2$ is small as long as $p$ is large.

\paragraph{The problem with non-Mersenne Primes}

Suppose $c\neq 1$ and we want to select arbitrary bits like in the arguments above.
If we pick the least significant bits we get a generic upper bound of $(1+c/p)/r$, which is not too bad for small $c$.
Here there is no conceptual difference to our Mersenne results.

However take the opposite extreme where
we pick just the one most significant bit
and $c=2^{b-1}-1$ (so $p=2^{b-1}+1$, a Fermat prime).
That bit is $0$ with probability $1-1/p$ and 1 only with probability $p$ -- virtually a constant.
We might try to fix this by xoring the output with a random number from $[2^b]$ (or add $C\in[2^b]$ and take mod $2^b$), but that will only make the bits uniform, not actually dependent on the key.
Thus if we hash two keys, $x_1$ and $x_2$, mod $2^{b-1}+1$ and take the top bit from each one, \emph{they will nearly always be the same, independent of whether $x_1=x_2$}.

More realistically,
say we pick the $\ell$ most significant bits
and $c\leq 2^{b-1}-2^{b-\ell}$, then $2^{b-\ell}$ elements from
$[p]$ map to $0$ while only $\max\{0,2^{b-\ell}-c\}$
map to the all \texttt{1}s.
More concretely, take $\ell=b/2$ and $c=2^{b/2}\approx\sqrt{p}$ (typical for generalized Mersenne primes) \emph{then the top $\ell$ bits hit the all \texttt{1}s with 0 probability}, while the all \texttt{0}s is twice as common as the remaining values.


\subsection{Two-for-one hash functions in second moment estimation}
In this section, we discuss how we can get several hash functions for
the price of one, and apply the idea to second moment estimation using
Count Sketches \cite{charikar04count-sketch}.

Suppose we had a $k$-universal hash function into $b$-bit strings.
We note that using standard programming languages such as C, we have
no simple and efficient method of computing such hash
functions when $k>2$. However, later we will argue that polynomial
hashing using a Mersenne prime $2^b-1$ delivers a better-than-expected
approximation.

Let $h:U\to [2^b]$ be $k$-universal. By definition this
means that if we have $j\leq k$ distinct keys $x_0,\ldots,x_{j - 1}$, then
$(h(x_0),\ldots,h(x_{j - 1}))$ is uniform in $[2^b]^j\equiv [2]^{bj}$,
so this means that \emph{all} the bits in $h(x_0),\ldots,h(x_{j - 1})$ are
independent and uniform. We can use this to split our $b$-bit hash
values into smaller segments, and sometimes use them as if
they were the output of universally computed hash functions.

We illustrate this idea below in the context of the second moment estimation.
For this purpose the ``split'' we will be considering is into the first bit and the remaining bits.

\subsubsection{Second moment estimation}\label{sec:count-sketch}
We now review the second moment estimation of streams based on Count Sketches \cite{charikar04count-sketch} (which are based on the
celebrated second moment AMS-estimator from \cite{alon96frequency}.)

The basic setup is as follows:
For keys in $[u]$ and integer values in $\Z$, we are given a stream of key/value $(x_0,\Delta_0),\ldots, (x_{n-1},\Delta_{n-1})\in [u]\times\Z$. The
total value of key $x\in[u]$ is
\[f_x=\sum_{i\in[n],x_i=x} \Delta_i.\]
We let $n\leq u$ be  the number of non-zero values
$f_x\neq 0$, $x\in [u]$. Often $n$ is much smaller than $u$.
We define the $m$th moment $F_m = \sum_{x\in [u]}f_y^m$. The goal here is to
estimate the second moment $F_2 = \sum_{x\in [u]}f_x^2=\|f\|^2_2$.

\begin{algorithm}[H]
	\caption{\label{alg:count-sketch} Count Sketch. Uses a
	vector/array $C$ of $r$ integers and two independent
	4-universal hash functions $i:[u]\to[r]$ and $s:[u]\to\{-1,1\}$.
	}
	\begin{algorithmic}
		\Procedure{Initialize}{}
		\State For $i\in[t]$, set $C[i]\gets 0$.
		\EndProcedure
		\Procedure{Process}{$x, \Delta$}
		\State $C[i(x)]\gets C[i(x)]+s(x) \Delta$.
		\EndProcedure
		\Procedure{Output}{}
		\State \Return $\sum_{i\in[t]} C[i]^2$.
		\EndProcedure
	\end{algorithmic}
\end{algorithm}
The standard analysis \cite{charikar04count-sketch} shows that
\begin{align}
	\E[X]   & = F_2 \label{eq:E-F2}                      \\
	\Var[X] & =2(F_2^2 - F_4)/r<2F_2^2/r \label{eq:V-F2}
\end{align}
We see that by choosing larger and larger r we can make X concentrate around $F_2=\|f\|^2_2$. Here
$X=\sum_{i\in[r]} C[i]^2=\|C\|^2_2$. Now $C$ is a randomized function
of $f$, and as $r$ grows, we get $\|C(f)\|^2_2\approx\|f\|^2_2$,
implying $\|C(f)\|_2\approx\|f\|_2$, that is, the Euclidean norm is
statistically preserved by the Count Sketch. However, the Count Sketch
is also a linear function, so Euclidean distances are statistically
preserved, that is, for any $f,g\in \Z^u$,
\[\|f-g\|_2\approx \|C(f-g)\|_2=\|C(f)-C(g)\|_2.\]
Thus, when we want to find close vectors, we can just work with the
much smaller Count Sketches.
%This is crucial to machine learning, where they adopted Count Sketches under the new name feature hashing~\cite{WDLSA09}.
The count sketch $C$ can also be used to estimate any single value $f_x$.
To do this, we use the unbiased estimator $X_x=s(x)C[i(x)]$.
This is yet another standard use of count sketch \cite{charikar04count-sketch}.
It requires direct access to both the sketch $C$ and the two hash functions $s$ and $i$.
To get concentration one takes the median of multiple such estimators.

\subsubsection{Two-for-one hash functions with b-bit hash values}
As the count sketch is described above,
it uses two independent 4-universal hash functions
$i:[u]\to[r]$ and $s:[u]\to\{-1,1\}$, but 4-universal hash functions
are generally slow to compute, so, aiming to save roughly a factor 2
in speed, a tempting idea is to compute them both using a single hash
function.

The analysis behind \eqref{eq:E-F2} and \eqref{eq:V-F2} does not quite
require $i:[u]\to[r]$ and $s:[u]\to\{-1,1\}$ to be independent.
It suffices that the hash values are uniform and that for any
given set of $j\leq 4$ distinct keys $x_0,\ldots,x_{j - 1}$, the $2j$ hash
values $i(x_0),\ldots,i(x_{j - 1}),s(x_0),\ldots,s(x_{j - 1})$ are independent.
A critical step in the analysis is that if
a value $A$ depends on the first $j-1$ values ($A=A(i(x_0),\ldots,i(x_{j - 1}),s(x_1),\ldots,s(x_{j - 1}))$), but doesn't depend
on $s(x_0)$, then
\begin{equation}\label{eq:E-0}
	\E[s(x_0) A] = 0 .
\end{equation}
This follows because $\E[s(x_0)]=0$ by uniformity of $s(x_0)$ and because $s(x_0)$ is independent of $A$.


Assuming that $r=2^\ell$ is a power of two, we can easily construct
$i:[u]\to[r]$ and $s:[u]\to\{-1,1\}$ using a single $4$-universal
hash function $h:[u]\to[2^b]$ where $b>\ell$. Recall that all the bits in
$h(x_0),\ldots,h(x_3)$ are independent. We can therefore use the
$\ell$ least significant bits of $h(x)$ for $i(x)$ and the most
significant bit of $h(x)$ for a bit $a(x)\in[2]$, and finally set
$s(x)=1-2a(x)$. It is then easy to show that if $h$ is $4$-universal
then $h$ satisfies \cref{eq:E-0}.
\begin{algorithm}[H]
	\caption{For key $x\in [u]$, compute $i(x)=i_x\in[2^\ell]$ and
	$s(x)=s_x\in\{-1,1\}$,\rule{5ex}{0ex}
	using $h:[u]\to [2^b]$ where $b>\ell$.}
	\label{alg:h-and-s}
	\begin{algorithmic}
		\State $h_x\gets h(x)$
		\Comment $h_x$ uses $b$ bits
		\State $i_x\gets h_x \andtt (2^\ell-1)$
		\Comment $i_x$ gets $\ell$ least significant bits of $h_x$
		\State $a_x\gets h_x\rs (b-1)$
		\Comment $a_x$ gets the most significant bit of $h_x$
		\State $s_x\gets 1-(a_x\ls1)$
		\Comment $a_x\in[2]$ is converted to a sign $s_x\in\{-1,1\}$
	\end{algorithmic}
\end{algorithm}
% \begin{lemma}\label{lem:b-bit-hashing} Suppose $h:[u]\to[2^b]$ is $k$-universal. Let
%    $i:[u]\to[2^\ell]$ and
%    $s:[u]\to\{-1,1\}$ be constructed from $h$ as described in Algorithm \ref{alg:h-and-s}. Then $h$ and $s$ are both $k$-universal. Moreover, for
%    any $j\leq k$ distinct keys $x_1,\ldots,x_j$, the $2j$ hash
%    values $i(x_1),\ldots,i(x_j),s(x_1),\ldots,s(x_j)$ are universal.
%    In particular, if $A$ depends on
%    $i(x_1),\ldots,i(x_j),s(x_2),\ldots,s(x_j)$, but not on $s(x_1)$, then
%    \begin{equation}\label{eq:E-0}
%       \E[s(x_1)A]=0
%    \end{equation}
% \end{lemma}
Note that Algorithm \ref{alg:h-and-s} is well defined as long as
$h$ returns a $b$-bit integer. However, \cref{eq:E-0} requires
that $h$ is $k$-universal into $[2^b]$, which in particular implies that
the hash values are uniform in $[2^b]$.


\subsubsection{Two-for-one hashing with  Mersenne primes}\label{sec:two-for-one}
Above we discussed how useful it would be with $k$-universal hashing
mapping uniformly into $b$-bit strings. The issue was that the lack of
efficient implementations with standard portable code if
$k>2$. However, when $2^b-1$ is a Mersenne prime $p\geq u$, then we do
have the efficient computation from Algorithm \ref{alg:Mersenne}
of a $k$-universal hash function $h:[u]\to[2^b-1]$. The hash values
are $b$-bit integers, and they are uniformly distributed, except that
we are missing the all \texttt{1}s value $p=2^b-1$. We want to
understand how this missing value affects us if we try to split the
hash values as in Algorithm \ref{alg:h-and-s}. Thus, we assume a
$k$-universal hash function $h:[u]\to[2^b-1]$ from which we construct
$i:[u]\to[2^\ell]$ and $s:[u]\to\{-1,1\}$ as
described in Algorithm \ref{alg:h-and-s}. As usual, we assume $2^\ell>1$.
Since $i_x$ and $s_x$ are
both obtained by selection of bits from $h_x$, we know from Section
\ref{sec:power-of-two} that each of them have close to uniform
distributions. However, we need a good replacement for \eqref{eq:E-0}
which besides uniformity, requires $i_x$ and $s_x$ to be independent,
and this is certainly not the case.

Before getting into the analysis, we argue that we really do get two
hash functions for the price of one. The point is that our efficient
computation in Algorithm \ref{alg:Mersenne} requires that we use a
Mersenne prime $2^b-1$ such that $u\leq 2^{b-1}$, and this is even if
our final target is to produce just a single bit for the sign function
$s:[u]\to\{-1,1\}$. We also know that $2^\ell<u$, for otherwise we
get perfect results implementing $i:[u]\to[2^\ell]$ as the identity
function (perfect because it is collision-free).  Thus we can assume
$\ell<b$, hence that $h$ provides enough bits for both $s$ and $i$.


We now consider the effect of the hash values from $h$ being uniform
in $[2^b-1]$ instead of in $[2^b]$. Suppose we want to compute the
expected value of an expression $B$ depending only on the independent
hash values $h(x_0),\ldots,h(x_{j - 1})$ of $j\leq k$ distinct keys
$x_0,\ldots,x_{j - 1}$.

Our generic idea is to play with the distribution of $h(x_0)$ while
leaving the distributions of the other independent hash values
$h(x_0)\ldots,h(x_{j - 1})$ unchanged, that is, they remain uniform in
$[2^b-1]$. We will consider having $h(x_0)$ uniformly distributed in
$[2^b]$, denoted $h(x_0) \sim \unif[2^b]$, but then we later have to
subtract the ``fake'' case where $h(x_0)=p=2^b-1$.  Making the
distribution of $h(x_0)$ explicit, we get
\begin{equation}\begin{split}
		\E_{h(x_0) \sim \unif[p]}[B]&=\sum_{y\in[p]}\E[B \mid h(x_0)=y]/p
		\\&=\sum_{y\in[2^b]}\E[B \mid h(x_0)=y]/p - \E[B \mid h(x_0)=p]/p
		\\ &=\E_{h(x_0) \sim \unif[2^b]}[B](p+1)/p - \E[B \mid h(x_0)=p]/p.\label{eq:play-with-dist}
	\end{split}\end{equation}
Let us now apply this idea our situation where $i:[u]\to[2^\ell]$ and
$s:[u]\to\{-1,1\}$ are constructed from $h$ as described in Algorithm
\ref{alg:h-and-s}. We will prove
\begin{lemma}\label{lem:remove-si}  Consider distinct keys $x_0,\ldots,x_{j - 1}$, $j\leq k$ and an expression $B=s(x_0)A$ where $A$
	depends on $i(x_0),\ldots,i(x_{j - 1})$ and $s(x_1),\ldots,s(x_{j - 1})$ but not
	$s(x_0)$. Then
	\begin{equation}\label{eq:remove-si}
		\E[s(x_0)A]=\E[A\mid i(x_0)=2^\ell-1]/p.
	\end{equation}
\end{lemma}
\begin{proof}
	When $h(x_0) \sim \unif[2^b]$, then $s(x_0)$ is uniform
	in $\{-1,1\}$ and independent of $i(x_0)$. The remaining
	$(i(x_i),s(x_i))$, $i\ge 1$, are independent of $s(x_0)$ because they
	are functions of $h(x_i)$ which is independent of $h(x_0)$, so
	we conclude that
	\[\E_{h(x) \sim \unif[2^b]}[s(x_0)A]=0\]
	Finally, when $h(x_0)=p$, we get $s(x_0)=-1$ and $i(x_0)=2^\ell-1$,
	so applying \eqref{eq:play-with-dist}, we conclude
	that
	\[\E[s(x_0)A] = -\E[s(x_0) A \mid h(x_0) = p]/p = \E[A \mid i(x_0)=2^\ell-1]/p.\]
\end{proof}
Above \eqref{eq:remove-si} is our replacement for \eqref{eq:E-0}, that is,
when the hash values from $h$ are uniform in $[2^b-1]$ instead of
in $[2^b]$, then $\E[s(x_0)B]$ is reduced by $\E[B \mid i(x_0)=2^\ell-1]/p$.
For large $p$, this is a small additive error. Using this in a careful
analysis, we will show that our fast second moment estimation
based on Mersenne primes performs almost perfectly:

\begin{theorem}\label{thm:h-and-s-p}
	Let $r>1$ and $u>r$ be powers of two and let $p=2^b-1>u$ be a
	Mersenne prime.
	Suppose we have a 4-universal hash function $h:[u]\to[2^b-1]$, e.g.,
	generated using Algorithm \ref{alg:Mersenne}. Suppose
	$i:[u]\to[r]$ and
	$s:[u]\to\{-1,1\}$ are constructed from $h$ as described in
	Algorithm \ref{alg:h-and-s}. Using this $i$ and $s$
	in the Count Sketch Algorithm \ref{alg:count-sketch}, the second moment
	estimate $X=\sum_{i\in[k]} C_i^2$ satisfies:
	\begin{align*}
		\E[X] < (1+n/p^2)\,F_2,
		\quad
		|\E[X] - F_2 | \le F_2 (n - 1)/p^2,
		\quad
		\Var[X]< 2F_2^2/r.
	\end{align*}
\end{theorem}
The difference from \eqref{eq:E-F2} and \eqref{eq:V-F2}
is negligible when $p$ is large. Theorem \ref{thm:h-and-s-p} will be
proved in Section \ref{sec:analysis-two-for-one}.

Recall our discussion from the end of Section
\ref{sec:power-of-two}. If we instead had used the $b$-bit prime
$p=2^{b-1}+1$, then the sign-bit $a_x$ would be extremely biased with
$\Pr[a_x=0]=1-1/p$ while $\Pr[a_x=1]=1/p$, leading to extremely poor
performance.


\subsection{An arbitrary number of buckets}\label{sec:most-uniform}
We now consider the general case where we want to hash into a set of
buckets $R$ whose size is not a power of two.  Suppose we have a
$2$-universal hash function $h:U\to Q$.  We will compose $h$ with a
map $\mu:Q\to R$, and use $\mu\circ h$ as a hash function from $U$ to
$R$.  Let $q=|Q|$ and $r=|R|$.  We want the map $\mu$ to be \emph{most
	uniform} in the sense that for bucket $i\in R$, the number of
elements from $Q$ mapping to $i$ is either $\floor{q/r}$ or
$\ceil{q/r}$.  Then the uniformity of hash values with $h$ implies for
any key $x$ and bucket $i\in R$ \[\floor{q/r}/q\leq
	\Pr[\mu(h(x))=i]\leq \ceil{q/r}/q.\] Below we typically have $Q=[q]$
and $R=[r]$.  A standard example of a most uniform map $\mu:[q]\to[r]$
is $\mu(x)=x\bmod r$ which the one used above when we defined
$h^r:[u]\to[r]$, but as we mentioned before, the modulo operation is
quite slow unless $r$ is a power of two.

Another example of a most uniform map $\mu:[q]\to[r]$
is $\mu(x)=\floor{xr/q}$,
which is also quite slow in general, but if $q=2^b$ is a power of two,
it can be implemented as $\mu(x)=(xr)\rs\,b$ where
$\rs$ denotes right-shift. This would be yet another advantage
of having $k$-universal hashing into $[2^b]$.

Now, our interest is the case where $q$ is a Mersenne prime $p=2^b-1$. We want
an efficient and most uniform map $\mu:[2^b-1]$ into any given $[r]$.
Our simple solution is to define
\begin{equation}\label{eq:most-uniform}
	\mu(v)=\floor{(v+1)r/2^b}=((v+1)r)\rs b.
\end{equation}
Lemma \ref{lem:most-uniform} (iii) below
states that \eqref{eq:most-uniform} indeed
gives a most uniform map.
\begin{lemma}\label{lem:most-uniform} Let $r$ and $b$ be positive integers.
	%, and let $v\in [2^b-1]$.
	Then
	\begin{itemize}
		\item[(i)] $v\mapsto (vr)\rs\,b$ is a most
		      uniform map from $[2^b]$ to $[r]$.
		\item[(ii)] $v\mapsto (vr)\rs\,b$ is a most
		      uniform map from $[2^b]\setminus\{0\}=\{1,\ldots,2^b-1\}$ to $[r]$.
		\item[(iii)] $v\mapsto ((v+1)r)\rs \, b$ is a most
		      uniform map from $[2^b-1]$ to $[r]$.
	\end{itemize}
\end{lemma}
\begin{proof}
	Trivially (ii) implies (iii).
	The statement (i) is folklore and easy to prove, so we know that every
	$i\in[r]$ gets hit by $\floor {2^b/r}$ or $\ceil{2^b/r}$ elements from
	$[2^b]$. It is also clear that $\ceil{2^b/r}$ elements, including $0$,
	map to $0$. To prove (ii), we remove $0$ from $[2^b]$,
	implying that only
	$\ceil{2^b/r}-1$ elements map to $0$. For all positive integers $q$
	and $r$, $\ceil{(q+1)/r}-1=\floor{q/r}$, and we use this here with
	$q=2^b-1$. It follows that all buckets from $[r]$ get $\floor{q/r}$
	or $\floor{q/r}+1$ elements from $Q=\{1,\ldots,q\}$. If $r$ does
	not divide $q$ then $\floor{q/r}+1=\ceil{q/r}$, as desired. However,
	if $r$ divides $q$, then $\floor{q/r}=q/r$, and this
	is the least number of elements from $Q$ hitting any bucket in $[r]$. Then
	no bucket from $[r]$ can get hit by more than $q/r=\ceil{q/r}$
	elements from $Q$. This completes the proof of (ii), and hence of (iii).
\end{proof}
We note that our trick does not work when $q=2^b-c$ for $c\geq 2$, that is,
using $v\mapsto ((v+c)r)\rs  b$, for in this general case,
the number of elements hashing to $0$ is $\ceil {2^b/r}-c$, or $0$ if
$c\geq \floor {2^b/r}$.
One may try many other hash functions $(c_1 v r+ c_2 v+ c_3 r + c_4) \rs b$ similarly without any luck.
Our new uniform map from \eqref{eq:most-uniform} is thus very specific to Mersenne prime fields.

% TODO: Hvad er det her?
%For general $c\ge 2$ we provide a scheme using two shifts in
%Section \ref{sec:pseudo-arbitrary}.




\subsection{Division and Modulo with (Pseudo) Mersenne Primes}\label{subsec:intro-division}

We now describe a new algorithm for truncated division with Mersenne primes, and more generalized numbers on the form $2^b-c$.
We show this implies a fast branch-free computation of $\bmod\,p$ for
Mersenne primes $p=2^b-1$.
An annoyance in Algorithm \ref{alg:Mersenne}
is that the if-statement at the end can be slow in case of branch mis-predictions.
This method solves that issue.

More specifically, in Algorithm \ref{alg:Mersenne}, after the last
multiplication, we have a number $y<p^2$ and we want to compute the
final hash value $y\bmod p$. We obtained this using the following
statements, each of which preserves the value modulo $p$, starting from
$y<p^2$:
\begin{algorithmic}
	\State $y \gets (y\andtt p)+(y\rs b)$
	\Comment $y<2p$
	\If{$y\ge p$}
	\State $y\gets y-p$
	\Comment  $y<p$
	\EndIf
\end{algorithmic}
To avoid the if-statement, in Algorithm \ref{alg:div-simple}, we suggest
a branch-free code that starting
from $v<2^{2b}$ computes both $y=v\bmod p$ and $z=\floor{v/p}$ using
a small number of AC$^0$ instructions.
\begin{algorithm}[H]
	\caption{For Mersenne prime $p=2^b-1$ and $v< 2^{2b}$, compute
		\label{alg:div-simple}
		$y=v\bmod p$ and $z=\floor{v/p}$}
	\begin{algorithmic}
		\State $\rhd$ First we compute $z=\floor{v/p}$
		\State $v'=v+1$
		\State $z \gets(( v' \rs b)+v')\rs b$
		%\State $y \gets v - (z \ls b) + z$.
		\State $\rhd$ Next we compute $y=v\bmod p$ given $z=\floor{v/p}$
		\State $y \gets (v + z) \andtt p $
	\end{algorithmic}
\end{algorithm}
In Algorithm \ref{alg:div-simple}, we use
$z=\floor{v/p}$ to compute $y=v\bmod p$. If we only want the
division $z=\floor{v/p}$, then we can skip the last statement.

Below we will generalize Algorithm \ref{alg:div-simple} to work for
arbitrary $v$, not only $v<2^{2b}$. Moreover, we will generalize
to work for different kinds of primes generalizing Mersenne primes:
\begin{description}
	\item[Pseudo-Mersenne Primes]
	      are primes of the form $2^b-c$, where is usually required that $c < 2^{\lfloor b/2\rfloor}$~\cite{van2014encyclopedia}.
	      Crandal patented a method for working with Pseudo-Mersenne Primes in 1992~\cite{crandall1992method},
	      why those primes are also sometimes called ``Crandal-primes''.
	      The method was formalized and extended by Jaewook Chung and Anwar Hasan in 2003~\cite{chung2003more}. The method we present is simpler with
	      stronger guarantees and better practical performance.
	      We provide a comparison with the Crandal-Chung-Hansan method in Section 4.
	\item[Generalized Mersenne Primes]
	      also sometimes known as Solinas primes~\cite{Solinas2011}, are sparse numbers, that is $f(2^b)$ where $f(x)$ is a low-degree polynomial.
	      Examples from the Internet Research Task Force's document ``Elliptic Curves for Security''~\cite{rfc7748}:
	      $p_{25519} = 2^{255} - 19$
	      and
	      $p_{448} = 2^{448}-2^{224}-1$.
	      We simply note that Solinas primes form a special case of
	      Pseudo-Mersenne Primes, where multiplication with $c$
	      can be done using a few shifts and additions.
\end{description}
We will now first generalize the division from Algorithm \ref{alg:div-simple} to cover arbitrary $v$ and division with an arbitrary Pseudo-Mersenne primes $p=2^b-c$.
This is done in Algorithm \ref{alg:division-generalized} below which
works also if $p=2^b-c$ is not a prime.  The
simple division in Algorithm \ref{alg:div-simple} corresponds to the case
where $c=1$ and $m=2$.
\begin{algorithm}[H]
	\caption{Given integers $p=2^b-c$ and $m$.
		For any $v< (2^b/c)^m$, compute $z=\floor{v/p}$}
	\label{alg:division-generalized}
	\begin{algorithmic}
		%\Procedure{Divide}{v, n, c}
		\State $v' \gets v + c$
		\State $z \gets v' \rs b$
		%\For{$i\gets 1$ \textbf{to} $m$}
		\For{ $m-1$ times}
		\State $z \gets (z * c + v')\rs b$
		\EndFor
		%\EndFor
		%\State \Return $v$
		%\EndProcedure
	\end{algorithmic}
\end{algorithm}
The proof that Algorithm \ref{alg:division-generalized} correctly computes
$z=\floor{v/p}$ is provided in Section \ref{sec:division}.
Note that $m$ can be computed in advance from $p$, and there is no requirement that it is chosen as small as possible.
For Mersenne and Solinas primes, the multiplication $z*c$ can be done very fast.

Mathematically the algorithm computes the nested division
$$
	\bbfloor{\frac{v}{q-c}}
	=
	\bbfloor{\frac{
			\bfloor{\frac{
					\floor{\frac{
							\dots+v+c
						}{q}}c +v+c
				}{q}}c +v+c
		}{q}}
	\vspace{-1em} % Move the next line further up
$$
which is visually similar to the series expansion
$
	\frac{v}{q-c}
	= \frac{v}{q}\sum_{i=0}^\infty (\frac{c}{q})^i
	%= v\frac{1+\frac{c+\frac{c^2 + \dots}{q}}{q}}{q}
	= \frac{\frac{\frac{\dots+v}{q}c+v}{q}c+v}{q}.
$
It is natural to truncate this after $m$ steps for a $(c/q)^m$ approximation.
The less intuitive part is that we need to add $v+c$ rather than $v$ at each step, to compensate for rounding down the intermediate divisions.

\paragraph{Computing mod}
We will now compute the $\bmod$ operation assuming that
we have already computed $z=\floor{v/p}$. Then
\begin{align}
	v \bmod p
	= v - pz
	= v - (2^b-c)z
	= v - (z\ls b) - c*z,
\end{align}
which is only two additions, a shift, and a multiplication with $c$ on top of the division algorithm.
As $pz = \floor{v/p}p \le v$ there is no danger of overflow.
We can save one operation by noting
that if $v = z (2^b-c) + y$, then
$$v\bmod p = y=\left(v+c*z \right) \bmod 2^b.$$
This is the method presented in Algorithm \ref{alg:mod-generalized} and applied with $c=1$ in Algorithm \ref{alg:div-simple}.
\begin{algorithm}[H]
	\caption{For integers $p=2^b-c$ and $z=\floor{v/p}$ compute
		$y=v \bmod p$.}
	\label{alg:mod-generalized}
	\begin{algorithmic}
		\State $y \gets (v + z*c) \andtt (2^b-1)$
	\end{algorithmic}
\end{algorithm}



%In the case $x\le 2^{2b}$ and $c=1$, we get the simplified Algorithm \ref{alg:div-simple} described above: $ \left\lfloor\frac{x}{2^n-1}\right\rfloor = (x+1 \rs n)+x+1 \rs n$.

% TODO: These applications need to be written properly.
% \subsubsection{Applications}\label{sec:general-applications}
% 
% \paragraph{To an arbitrary number of buckets}
% In Subsection~\ref{sec:most-uniform} we discussed how $\floor{\frac{h(x)r}{2^b-1}}$ provides a most uniform map from $[2^b-1]\to[r]$.
% To avoid the division step, we instead considered the map
% $\floor{\frac{(h(x)+1)r}{2^b}}$.
% However, for primes of the form $2^b-c$, $c>1$ this approach doesn't provide a most-uniform map.
% %
% Instead, we may use Algorithm \ref{alg:division-generalized} to compute
% $$\left\lfloor\frac{h(x)r}{2^b-c}\right\rfloor$$
% directly, getting a perfect most-uniform map.
% %(Another alternative was to pre-compute $q = \lfloor2^b/p\rfloor$ and take
% %$\floor{\frac{h(x)rq}{2^b}}$, however that requires larger words to store the product $h(x)rq$.)
% 
% \paragraph{Application to Finger Printing}
% A classical idea by Rabin~\cite{rabin1981fingerprinting} is to use test the equality of two large numbers by comparing their value modulo a random prime.
% A beautiful example of this is King and Sagert's Algorithm for Maintaining the Transitive Closure.~\cite{DBLP:journals/jcss/KingS02}
% These cases are typically non-trivial to convert to random fields other than $\Z_p$, and we need a reasonably large set of random primes to choose from.
% The generalized Mersenne primes with $c$ of the other $2^{b(1-\eps)}$ are a good candidate, since the range ...
% Using an idea like \eqref{eq:Mersenne} we can note
% \begin{equation}
% 	y \equiv y - \floor{y/2^b}(2^b-c)
% 	= (y\bmod 2^{b}) + c\floor{y/2^b}
% 	\pmod {2^b-c}.
% \end{equation}
% Each application reduces $y$ by a factor $\approx 2^{-\eps b}$.
% So if we multiply two numbers



% On the number of Mersenne Primes:
%Unfortunately there are only 45 of them known.
%The most useful one perhaps being.
%Heuristically there are $O(\log x)$ Mersenne primes up to $x$.
%Trivia: Euler proved that an even number $n$ is perfect if and only if it is of the form $n=2^{q-1}M_q$, where $M_q=2^q-1$ is prime.
%(Usually we know a number is perfect if its divisors sum to the number itself, e.g. $6=1+2+3$ or $28=1+2+4+7+14$.)


%[[Curve448]] uses the Solinas prime <math>2^{448} - 2^{224} - 1</math>



%! TEX root = ../mersenne.tex

\section{Analysis of second moment estimation using Mersenne primes}
\label{sec:analysis-two-for-one}
In this section, we will prove Theorem \ref{thm:h-and-s-p}---that a single Mersenne hash function works for Count Sketch.
Recall that for each key $x\in [u]$, we have a value $f_x\in \Z$, and the
goal was to estimate the second moment $F_2 = \sum_{x\in u}f_x^2$.

We had two functions $i:[u]\to[r]$ and $s:[u]\to\{-1,1\}$. 
For notational convenience, we define $i_x=i(x)$ and $s_x=s(x)$.
We let $r=2^\ell>1$ and $u>r$ both be powers of two and $p=2^b-1>u$ a Mersenne prime.
For each $i\in [r]$, we have a counter 
$C_i=\sum_{x\in[u]} s_x f_x[i_x=i]$, and we define the 
estimator $X=\sum_{i\in[r]} C_i^2$. We want to study how
well it approximates $F_2$.
We have 
\begin{align}
X=\sum_{i\in[r]}\left( \sum_{x\in[u]}s_x f_x[i_x=i]\right)^2
%=\sum_{i\in[r]}\sum_{x,y\in[u]}s_x s_y f_x f_y [i_x = i_y = i]
=\sum_{x,y\in[u]}s_x s_y f_x f_y[i_x=i_y]
=\sum_{x\in[u]} f_x^2+Y,
\label{eq:decomp}
\end{align}
where $Y=\sum_{x,y\in[u],x\neq y} s_x s_y f_x f_y [i_x = i_y]$.
The goal is thus to bound mean and variance of the error $Y$.

As discussed in the introduction, one of the critical steps in the analysis of count sketch in the classical case is \cref{eq:E-0}.
We formalize this into the following property:
\begin{property}[Sign Cancellation]\label{prop:independence}
    For distinct keys $x_0, \ldots x_{j - 1}$, $j \le k$
    and an expression $A(i_{x_0}, \ldots, i_{x_{j - 1}}, s_{x_1}, \ldots, s_{x_{j - 1}})$,
    which depends on $i_{x_0}, \ldots, i_{x_{j - 1}}$ and $s_{x_1}, \ldots, s_{x_{j - 1}}$
    but not on $s_{x_0}$
    \begin{align}
        \E[s_{x_0} A(i_{x_0}, \ldots, i_{x_{j - 1}}, s_{x_1}, \ldots, s_{x_{j - 1}})] = 0\; .
    \end{align}
\end{property}

In the case where we use a Mersenne prime for our hash function we have that $h$ is uniform in $[2^b - 1]$ and not in $[2^b]$, hence \Cref{prop:independence} is not satisfied.
Instead, we have \cref{eq:E-0} which is almost as good, and will replace \Cref{prop:independence} in the analysis for count sketch.
We formalize this as follows:
\begin{property}[Sign Near Cancellation]\label{prop:near-independence}
    Given $k, p$ and $\delta$,
    there exists $t \in [r]$ such that for distinct keys $x_0, \ldots x_{j - 1}$, $j \le k$
    and an expression $A(i_{x_0}, \ldots, i_{x_{j - 1}}, s_{x_1}, \dots, s_{x_{j - 1}})$,
    which depends on $i_{x_0}, \ldots, i_{x_{j - 1}}$
    and $s_{x_1}, \ldots, s_{x_{j - 1}}$,
    but not on $s_{x_0}$,
    \begin{align}
        \E[s_{x_0} A(i_{x_0}, \ldots, i_{x_{j - 1}}, s_{x_1}, \ldots, s_{x_{j - 1}})]
            &= \frac1p \E[A(i_{x_0}, \ldots, i_{x_{j - 1}}, s_{x_1}, \ldots, s_{x_{j - 1}}) \mid i_{x_0} = t].
         \label{eq:near-independence}
            \\
            \text{and}\quad
    \Pr[i_x = t] &\le (1 + \delta)/r
    \quad\text{for any key $x$}.
         \label{eq:prob-special-value}
    \end{align}
\end{property}

When the hash function $h$ is not uniform then it is not guaranteed that
the collision probability is $1/r$, but \eqref{eq:coll} showed that for
Mersenne primes the collision probability is $(1 + (r - 1)/p^2)/r$.
We formalize this into the following property.
\begin{property}[Low Collisions]\label{prop:collision}
   We say the hash function has $(1+\eps)/r$-low collision probability, if
    for distinct keys $x \neq y$,
    \begin{align}\label{eq:collision}
        \Pr[i_x = i_y] \le (1 + \eps)/r\; .
    \end{align}
\end{property}

\subsection{The analysis in the classical case}
First, as a warm-up for later comparison, we analyse the
case where we have Sign Cancellation, but
the collision probability bound is only $(1+\eps)/r$.
This will come in useful in \Cref{sec:arbitrary-buckets} where we will consider the case of an arbitrary number of buckets, not necessarily a power of two.
\begin{lemma}\label{lem:count-classic}
   If the hash function has Sign Cancellation for $k = 4$ and $(1+\eps)/r$-low collision probability, then
    \begin{align}
        \E[X] &= F_2 \\
        \Var[X] &\le 2(1 + \eps)(F_2^2 - F_4)/r \le 2(1 + \eps)F_2^2/r .
    \end{align}
\end{lemma}
\begin{proof}
   Recall the decomposition $X=F_2+Y$ from \cref{eq:decomp}.
    We will first show that $\E[Y] = 0$.
    By \Cref{prop:independence} we have that $\E[s_x s_y f_x f_y [i_x = i_y]] = 0$
    for $x \neq y$ and thus $\E[Y] = \sum_{x,y\in[u],x\neq y} \E[s_x s_y f_x f_y [i_x = i_y]] = 0$.

    Now we want to bound the variance of $X$. We note that since $\E[Y] = 0$ and $X = F_2 + Y$
    \begin{align*}
        \Var[X] = \Var[Y] = \E[Y^2]
            = \sum_{\substack{x, y, x', y' \in [u]\\ x \neq y, x' \neq y'}} \E[(s_x s_y f_x f_y [i_x = i_y])(s_{x'} s_{y'} f_{x'} f_{y'} [i_{x'} = i_{y'}])] .
    \end{align*}
    Now we consider one of the terms $\E[(s_x s_y f_x f_y [i_x = i_y])(s_{x'} s_{y'} f_{x'} f_{y'} [i_{x'} = i_{y'}])]$.
    Suppose that one of the keys, say $x$, is unique, i.e. $x \not\in \{y, x', y'\}$.
    Then the Sign Cancellation Property implies that 
    \[
        \E[(s_x s_y f_x f_y [i_x = i_y])(s_{x'} s_{y'} f_{x'} f_{y'} [i_{x'} = i_{y'}])] = 0 .
    \]
    Thus we can now assume that there are no unique keys. Since $x \neq y$ and $x' \neq y'$, we conclude
    that $(x, y) = (x', y')$ or $(x, y) = (y', x')$. Therefore
    \begin{align*}
       \Var[X] &= \sum_{\substack{x, y, x', y' \in [u]\\ x \neq y, x' \neq y'}}
                \E[(s_x s_y f_x f_y [i_x = i_y])(s_{x'} s_{y'} f_{x'} f_{y'} [i_{x'} = i_{y'}])]
            \\&= 2 \sum_{\substack{x, y, x', y' \in [u]\\ x \neq y, (x', y') = (x, y)}}
                \E[(s_x s_y f_x f_y [i_x = i_y])(s_{x'} s_{y'} f_{x'} f_{y'} [i_{x'} = i_{y'}])]
            \\&= 2\sum_{x,y\in[u],x\neq y} \E[(s_x s_y f_x f_y[i_x=i_y])^2]
            \\&= 2\sum_{x,y\in[u],x\neq y} \E[(f_x^2f_y^2[i_x=i_y])]
            \\&\le 2\sum_{x,y\in[u],x\neq y} (f_x^2f_y^2)(1 + \eps)/r
            \\&= 2(1 + \eps) (F_2^2-F_4)/r.
    \end{align*}
    The inequality follows by \Cref{prop:collision}.
\end{proof}
%Something something finish this part\todo{Write this properly}
%This completes the proof of \eqref{eq:V-F2}.
In the above analysis, we
did not need $s$ and $i$ to be completely independent. All we needed
was that for any $j\leq 4$ distinct keys $x_0,\ldots,x_{j - 1}$, the hash
values $s(x_0),\ldots,s(x_{j - 1})$ and $i(x_0),\ldots,i(x_{j - 1})$ are all
independent and uniform in the desired domain. This was why we could
use a single 4-universal hash function $h:[u]\to[2^b]$ with
$b>\ell$, and use it to construct $s:[u]\to\{-1, 1\}$ and
$i:[u]\to[2^\ell]$ as described in Algorithm \ref{alg:h-and-s}.
% This was removed from the introduction
%(c.f. Lemma \ref{lem:b-bit-hashing}).

\subsection{The analysis of two-for-one using Mersenne primes}
We will now analyse the case where the functions $s : [u] \to \{-1, 1\}$
and $i : [u] \to [2^l]$ are constructed as in \Cref{alg:Mersenne} from a
single $k$-universal hash function $h : [u] \to [2^b - 1]$ where $2^b - 1$
is a Mersenne prime.
We now only have \nameref{prop:near-independence}.
We will show that this does
not change the expectation and variance too much. Similarly, to the
analysis of the classical case, we will analyse a slightly more general
problem, which will be useful in \Cref{sec:arbitrary-buckets}.
\begin{lemma}\label{lem:count-mersenne}
   If we have \nameref{prop:near-independence} with $
    \Pr[i_x = t] \le (1 + \delta)/r$
   and $(1+\eps)/r$-low collision probability,
   then
    \begin{align}
        \E[X] &= F_2 + (F_1^2 - F_2)/p^2 \\
        | \E[X] - F_2 | &\le F_2 (n - 1)/p^2 \\
        \Var[X] &\le 2F_2^2/r + F_2^2 (2\eps/r + 4(1 + \delta)n / (rp^2) + n^2/p^4 - 2 /(rn))
    \end{align}
\end{lemma}
\begin{proof}
    We first bound $\E[s_x s_y f_x f_y [i_x = i_y]]$ for distinct keys
    $x \neq y$.
    Let $t$ be the special index given by Sign Near Independence.
    Using \cref{eq:near-independence} twice we get that
    \begin{equation}\begin{split}\label{eq:twice-split}
        \E[s_x s_y f_x f_y [i_x = i_y]]
            &= \E[s_x f_x f_y [i_x = i_y] \mid i_y = t]/p
            \\&= \E[s_x f_x f_y [i_x = t]]/p
            \\&= \E[f_x f_y [i_x = t] \mid i_x = t]/p
            \\&= f_x f_y / p^2 \; .
    \end{split}\end{equation}
    From this, we can calculate $\E[X]$.
    \begin{align*}
        \E[X]
            = F_2 + \sum_{x \neq y} \E[s_x s_y f_x f_y [i_x = i_y]]
            = F_2 + (F_1^2 - F_2)/p^2 .
    \end{align*}
    Now we note that $0 \le F_1^2 \le n F_2$ by Cauchy-Schwarz, hence we get that
    $| \E[X] - F_2 | \le (n - 1)/p^2$.

    The same method is applied to the analysis of the variance, which is
    \[
        \Var[X]
            = \Var[Y]
            \le \E[Y^2]
            = \sum_{x,y,x',y' \in [u], x \neq y, x' \neq y'} \E[(s_x s_y f_x f_y [i_x = i_y]) (s_{x'} s_{y'} f_{x'} f_{y'}[i_{x'} = i_{y'}])]
        \; .
    \] 
    Consider any term in the sum. Suppose some key, say $x$, is unique in the
    sense that $x \not \in \{y,x',y'\}$. Then we can apply \cref{eq:near-independence}.
    Given that $x \neq y$ and $x'\neq y'$, we have either $2$ or $4$ such unique keys.
    If all 4 keys are distinct, as in \cref{eq:twice-split}, we get
    \begin{align*}
        \E[(s_x s_y f_x f_y [i_x = i_y]) &(s_{x'} s_{y'} f_{x'} f_{y'}[i_{x'} = i_{y'}])]
            \\&= \E[(s_x s_y f_x f_y [i_x = i_y])] \E[s_{x'} s_{y'} f_{x'} f_{y'}[i_{x'} = i_{y'}])]
            \\&= (f_x f_y/p^2)(f_{x'} f_{y'}/p^2)
            \\&= f_x f_y f_{x'} f_{y'}/p^4
        \; .
    \end{align*}
    The expected sum over all such terms is thus bounded
    as 
    \begin{equation}\begin{split}
        \sum_{{\rm distinct}\, x,y,x',y'\in[u]}& \E[(s_x s_y f_x f_y [i_x = i_y]) (s_{x'} s_{y'} f_{x'} f_{y'}[i_{x'} = i_{y'}])]
            \\&= \sum_{{\rm distinct}\,x,y,x',y'\in[u]} f_xf_yf_{x'}f_{y'}/p^4
            \\&\le F_1^4 /p^4
            \\&\le F_2^2 n^2/p^4.\label{eq:distinct}
    \end{split}\end{equation}
    Where the last inequality used Cauchy-Schwarz. We also have to consider all the cases with
    two unique keys, e.g., $x$ and $x'$ unique while $y=y'$. Then using \cref{eq:near-independence}
    and \cref{eq:prob-special-value}, we get
    \begin{align*}
        \E[(s_x s_y f_x f_y [i_x = i_y]) &(s_{x'} s_{y'} f_{x'} f_{y'}[i_{x'} = i_{y'}])]
            \\&= f_x f_{x'} f_y^2 \E[s_x s_{x'} [i_x = i_{x'} = i_y]]
            \\&= f_x f_{x'} f_y^2 \E[s_{x'} [t = i_{x'} = i_y]]/p
            \\&= f_x f_{x'} f_y^2 \E[t = i_y]/p^2
            \\&\le f_x f_{x'} f_y^2(1 + \delta)/(rp^2).
    \end{align*}    
    Summing over all terms with $x$ and $x'$ unique while $y=y'$, and
    using Cauchy-Schwarz and $u\leq p$, we get 
    \begin{align*}
        \sum_{{\rm distinct}\,x,x',y} f_xf_{x'}f_y^2 (1 + \delta) /(rp^2) 
            \le F_1^2 F_2 (1 + \delta)/(rp^2)
            \le F_2^2 n(1 + \delta)/(rp^2).
    \end{align*}
    There are four ways we can pick the two unique keys $a\in \{x,y\}$
    and $b\in \{x',y'\}$, so we conclude that
    \begin{equation}\label{eq:one-pair}
        \sum_{\substack{
            x,y,x',y'\in[u], x\neq y, x'\neq y',\\
            (x,y)=(x',y')\,\vee\,(x,y)=(y',x')
        }}
        \E[(s_x s_y f_x f_y [i_x = i_y]) (s_{x'} s_{y'} f_{x'} f_{y'}[i_{x'} = i_{y'}])]
            \le 4 F_2^2 n(1 + \delta)/(rp^2) .
    \end{equation}
    Finally, we need to reconsider the terms with two pairs, that
    is where $(x,y)=(x',y')$ or $(x,y)=(y',x')$. In
    this case, $(s_x s_y f_x f_y [i_x = i_y]) (s_{x'} s_{y'} f_{x'} f_{y'}[i_{x'} = i_{y'}]) = f_x^2 f_y^2 [i_x = i_y]$.
    By \cref{eq:collision}, we get 
    \begin{equation}\begin{split}    
        \sum_{\substack{
            x,y,x',y'\in[u], x\neq y, x'\neq y',\\
            (x,y)=(x',y')\,\vee\,(x,y)=(y',x')
        }}&
            \E[(s_x s_y f_x f_y [i_x = i_y]) (s_{x'} s_{y'} f_{x'} f_{y'}[i_{x'} = i_{y'}])]
            \\&=2\sum_{x,y\in[u],x\neq y} f_x^2f_y^2 \Pr[i_x=i_y]
            \\&=2\sum_{x,y\in[u],x\neq y} f_x^2f_y^2 (1 + \eps)/r
            \\&=2(F_2^2 - F_4)(1 + \eps)/r .\label{eq:two-pairs}
    \end{split}\end{equation}
    Adding up add \eqref{eq:distinct}, \eqref{eq:one-pair}, and
    \eqref{eq:two-pairs}, we get 
    \begin{align*}
        \Var[Y]
            &\le 2(1 + \eps)(F_2^2 - F_4)/r + F_2^2(4(1 + \delta) n / (rp^2) + n^2/p^4)
            \\&\le 2F_2^2/r + F_2^2 (2\eps/r + 4(1 + \delta)n / (rp^2) + n^2/p^4 - 2 /(rn)) .
    \end{align*}
    This finishes the proof.
\end{proof}

We are now ready to prove \Cref{thm:h-and-s-p}.
\begingroup
    \def\thelemma{\ref{thm:h-and-s-p}}
    \begin{theorem}
        Let $r>1$ and $u>r$ be powers of two and let $p=2^b-1>u$ be a
        Mersenne prime.
        Suppose we have a 4-universal hash function $h:[u]\to[2^b-1]$, e.g.,
        generated using Algorithm \ref{alg:Mersenne}. Suppose
        $i:[u]\to[r]$ and
        $s:[u]\to\{-1,1\}$ are constructed from $h$ as described in
        Algorithm \ref{alg:h-and-s}. Using this $i$ and $s$ 
        in the Count Sketch Algorithm \ref{alg:count-sketch}, the second moment 
        estimate $X=\sum_{i\in[k]} C_i^2$ satisfies:
        \begin{align}
           \E[X] &= F_2+(F_1^2-F_2)/p^2, \label{eq:E-F2-p}\\
           | \E[X] - F_2 | &\le F_2 (n - 1)/p^2, \label{eq:E-F2-p-com}\\
           \Var[X]&< 2F_2^2/r.\label{eq:V-F2-p}
        \end{align}
    \end{theorem}
    \addtocounter{lemma}{-1}
\endgroup

From \Cref{eq:remove-si} and \Cref{eq:coll-ell=r-1} we have
\nameref{prop:near-independence} with $
    \Pr[i_x = 2^b - 1] \le (1 - (r - 1)/p)/r$
   and \Cref{eq:coll} $(1 + (r - 1)/p^2)/r$-low collision probability
% that \Cref{prop:near-independence} is satisfied with $t = r - 1$
% and $\delta = -(r - 1)/p$, and \Cref{eq:coll} implies that
% \Cref{prop:collision} is satisfied with $\eps = (r - 1)/p^2$.
Now \Cref{lem:count-mersenne} give us \eqref{eq:E-F2-p}
and \eqref{eq:E-F2-p-com}. Furthermore, we have that
\begin{align*}
    \Var[X] 
        &\le 2F_2^2/r + F_2^2 (2\eps/r + 4(1 + \delta)n / (rp^2) + n^2/p^4 - 2 /(rn))
        \\&= 2F_2^2/r + F_2^2(2/p^2 + 4n/(r p^2) + n^2/p^4 - 2/(rn)) .
\end{align*}

We know that $2 \le r \le u/2 \le (p + 1)/4$ and $n \le u$.
This implies that $p \ge 7$ and that $n/p \le u/p \le 4/7$.
We want to prove that
$2/p^2 + 4n/(r p^2) + n^2/p^4 - 2/(rn) \le 0$ which would
prove our result. We get that
\begin{align*}
    2/p^2 + 4n/(r p^2) + n^2/p^4 - 2/(rn)
        \le 2/p^2 + 4u/(r p^2) + u^2/p^4 - 2/(ru) .
\end{align*}
Now we note that $4u/(r p^2) - 2/(ru) = (2u^2 - p^2)/(u p^2 r) \le 0$
since $u \le (p + 1)/2$ so it maximized when $r = u/2$. We then get
that
\begin{align*}
    2/p^2 + 4u/(r p^2) + u^2/p^4 - 2/(ru)
        \le 2/p^2 + 8/p^2 + u^2 / p^4 - 4/u^2 .
\end{align*}
We now use that $u/p \le (4/7)^2$ and get that
\begin{align*}
    2/p^2 + 8/p^2 + u^2 / p^4 - 4/u^2
        \le (10 + (4/7)^2 - 4 (7/4)^2)/p^2
        \le 0 .
\end{align*}
This finishes the proof of \eqref{eq:V-F2-p} and thus also of \Cref{thm:h-and-s-p}.



%! TEX root = ../mersenne.tex
\section{Algorithms and analysis with arbitrary number of buckets}\label{sec:arbitrary-buckets}
We now consider the case where we want to hash into
a number of buckets. We will analyse the collision probability
with most uniform maps introduced in Section \ref{sec:most-uniform},
and later we will show how it can be used in connection with the
two-for-one hashing from Section \ref{sec:two-for-one}.

\subsection{Two-for-one hashing from uniform bits to arbitrary number of buckets}
We have a hash function $h:U\to Q$, but we want hash values in $R$, so
we need a map $\mu:Q\to R$, and then use $\mu\circ h$ as
our hash function from $U$ to $R$. We normally assume that the hash values 
with $h$ are pairwise independent, that is, for any distinct $x$ and $y$,
the hash values $h(x)$ and $h(y)$ are independent, but then 
$\mu(h(x))$ and $\mu(h(y))$ are also independent. This means
that the collision probability can be calculated
as 
\[\Pr[\mu(h(x))=\mu(h(y))]=\sum_{i\in R}\Pr[\mu(h(x))=\mu(h(y))=i]=\sum_{i\in R}\Pr[\mu(h(x)=i)]^2.\]
This sum of squared probabilities attains is minimum value $1/|R|$
exactly when $\mu(h(x))$ is uniform in $R$. 

Let $q=|Q|$ and $r=|R|$. Suppose that $h$ is $2$-universal. Then
$h(x)$ is uniform in $Q$, and then we get the lowest collision
probability with $\mu\circ h$ if $\mu$ is most uniform as defined in
Section \ref{sec:most-uniform}, that is, the number of elements from
$Q$ mapping to any $i\in[r]$ is either $\floor{q/r}$ or
$\ceil{q/r}$. To calculate the collision probability,
Let $a\in[r]$ be such that $r$ divides $q+a$. Then the map $\mu$ maps
$\ceil{q/r}=(q+a)/r$ balls to $r-a$ buckets and
$\floor{q/r}=(q+a-r)/r$ balls to $a$ buckets. For a key $x\in [u]$, we
thus have $r-a$ buckets hit with probability $(1+a/q)/r$ and
$a$ buckets hit with probability $(1-(r-a)/q)/r$.
The collision probability is then
\begin{equation}\begin{split}
   \Pr[\mu(h(x))=\mu(h(y))]
                  &= (r-a)((1+a/q)/r)^2+a((1-(r-a)r/q)/r)^2
%                 \\&=\frac{(r-a)+(r-a)2a/q+(r-a)a^2/q^2+ a-a^2(r-a)/p+a(r-a)^2/q^2}{r^2}
%                 \\&=\frac{r +r a (r-a)/q^2}{r^2}
                 \\&=(1+a(r-a)/q^2)/r
                 \\&\le \left(1+(r/(2q))^2\right)/r.\label{eq:coll-a}
\end{split}\end{equation}
Note that the above calculation generalizes the one for \eqref{eq:coll} which
had $a=1$. We will think of $(r/(2q))^2$ as the general relative rounding
cost when we do not have any information about how $r$ divides $q$.

\subsection{Two-for-one hashing from uniform bits to arbitrary number of buckets}
We will now briefly discuss how would get the two-for-one hash
functions in count sketches with an arbitrary number $r$ of buckets based
on a single $4$-universal hash function $h:[u]\to [2^b]$.  We want to
construct the two hash functions $s:[u]\to\{-1,1\}$ and
$i:[u]\to[r]$. As usual the results with uniform $b$-bit strings will
set the bar that we later compare with when from $h$ we get hash values that
are only uniform in $[2^b-1]$.

The construction of $s$ and $i$ is presented in 
Algorithm \ref{alg:b-bit-arb-r}.
\begin{algorithm}[H]
   \caption{For key $x\in [u]$, compute $i(x)=i_x\in[r]$ and $s(x)=s_x\in\{-1,1\}$.
   \newline
    Uses 4-universal $h:[u]\to [2^b]$.}
   \label{alg:b-bit-arb-r}
   \begin{algorithmic}
      \State $h_x\gets h(x)$
      \Comment $h_x$ has $b$ uniform bits
      \State $j_x\gets h_x\andtt(2^{b-1}-1)$
      \Comment $j_x$ gets $b-1$ least significant bits of $h_x$
      \State $i_x\gets (r*j_x)\rs (b-1)$
      \Comment $i_x$ is most uniform in $[r]$
      \State $a_x\gets h_x\rs (b-1)$
      \Comment $a_x$ gets the most significant bit of $h_x$
      \State $s_x\gets (a_x\ls 1)-1$
      \Comment $s_x$ is uniform in $\{-1,1\}$ and independent of $i_x$.
   \end{algorithmic}
\end{algorithm}
The difference relative to Algorithm \ref{alg:h-and-s} is the computation
of $i_x$ where we now first pick out the $(b-1)$-bit string $j_x$ from
$h_x$, and then apply the most uniform map $(rj_x)\rs (b-1)$
to get $i_x$. This does not affect $s_x$ which remains independent
of $i_x$, hence we still have \nameref{prop:independence}.
But $i_x$ is no longer uniform in $[r]$ and only most uniform
so by \eqref{eq:coll-a} we have $(1 + (r/2^b)^2)/r$-low collision probability.
Now \Cref{lem:count-classic} give us $\E[X] = F_2$ and
\begin{equation}\label{eq:Var-b-bit-arb-r}
   \Var[X] \le 2(F_2^2 - F_4)\left(1+(r/2^b)^2\right)/r
      \le 2 F_2^2\left(1 + (r/2^b)^2 \right)/r .
\end{equation}

\subsection{Two-for-one hashing from Mersenne primes to arbitrary number of buckets}
We will now show how wan get the two-for-one hash functions in count
sketches with an arbitrary number $r$ of buckets based on a single
$4$-universal hash function $h:[u]\to [2^b-1]$.  Again we want to
construct the two hash functions $s:[u]\to\{-1,1\}$ and
$i:[u]\to[r]$.  The construction will be the same as we had in
Algorithm \ref{alg:b-bit-arb-r} when $h$ returned uniform values in
$[2^b]$ with the change that we set $h_x\gets h(x)+1$, so that it
becomes uniform in $[2^b]\setminus\{0\}$. It is also convenient to
swap the sign of the sign-bit $s_x$ setting $s_x\gets 2a_x - 1$ instead
of $s_x\gets 1-2a_x$. The basic reason is that this makes the analysis
cleaner. The resulting algorithm
is presented as Algorithm \ref{alg:Mersenne-arb-r}.
\begin{algorithm}[H]
   \caption{For key $x\in [u]$, compute $i(x)=i_x\in[r]$ and
      $s(x)=s_x\in\{-1,1\}$.\rule{5ex}{0ex}
   Uses 4-universal $h:[u]\to [p]$ for Mersenne prime $p=2^b-1\geq u$.}
   \label{alg:Mersenne-arb-r}
   \begin{algorithmic}
      \State $h_x\gets h(x)+1$
      \Comment $h_x$ uses $b$ bits uniformly except $h_x\neq 0$
      \State $j_x\gets h_x\andtt(2^{b-1}-1)$
      \Comment $j_x$ gets $b-1$ least significant bits of $h_x$
      \State $i_x\gets (r*j_x)\rs (b-1)$
      \Comment $i_x$ is quite uniform in $[r]$
      \State $a_x\gets h_x\rs (b-1)$
      \Comment $a_x$ gets the most significant bit of $h_x$
      \State $s_x\gets 1-(a_x\ls1)$
      \Comment $s_x$ is quite uniform in $\{-1,1\}$ and quite independent of $i_x$.
   \end{algorithmic}
\end{algorithm}
The rest of Algorithm \ref{alg:Mersenne-arb-r} is exactly like 
Algorithm \ref{alg:b-bit-arb-r}, and we will now discuss the new
distributions of the resulting variables. We had
$h_x$ uniform in $[2^b]\setminus\{0\}$, and then we set
$j_x \gets h_x\andtt(2^{b-1}-1)$. Then $j_x\in[2^{b-1}]$ with 
$\Pr[j_x=0]=1/(2^{b}-1)$ while  $\Pr[j_x=j]=2/(2^{b}-1)$ for all $j>0$.

Next we set $i_x\gets (rj_x)\rs b-1$. We know from Lemma
\ref{lem:most-uniform} (i) that this is a most uniform map from
$[2^{b-1}]$ to $[r]$.  It maps a maximal number of elements from
$[2^{b-1}]$ to $0$, including $0$ which had half probability for
$j_x$.
We conclude
\begin{align}
   \Pr[i_x=0] &= (\ceil{2^{b-1}/r}2-1)/(2^{b}-1)
   \label{eq:prix0}
   \\
   \Pr[i_x = i] &\in
   \{\floor{2^{b-1}/r}2/(2^{b}-1), \ceil{2^{b-1}/r}2/(2^{b}-1)\}
   \mbox{ for $i \neq  0$}
   \label{eq:prixneq0}
   .
\end{align}
%all $i\in[r]\setminus\{0\}$ have probability
%while
%$0$ has probability $(\ceil{2^{b-1}/r}2-1)/(2^{b}-1)$. 
We note
that the probability for $0$ is in the middle of the two other
bounds and often this yields a more uniform distribution on $[r]$ than
the most uniform distribution we could get from the
uniform distribution on $[2^{b-1}]$.

With
more careful calculations, we can get some nicer bounds
that we shall later use.
\begin{lemma}\label{lem:ix-r-dist} For any distinct $x,y\in [u]$, 
   \begin{align}
      \Pr[i_x=0]&\le(1+r/2^b)/r\label{eq:ix=0}\\
      \Pr[i_x=i_y]&\leq \left(1+(r/2^b)^2\right)/r.\label{eq:ix=iy}
   \end{align}
\end{lemma}
\begin{proof}
   The proof of \eqref{eq:ix=0} is a simple calculation.
   Using \eqref{eq:prix0} and the fact $\ceil{2^{b-1}/r}\le(2^{b-1}+r-1)/r$ we have
   \begin{align*}
      \Pr[i_x=0]&\le (2(2^{b-1}+r-1)/r)-1)/(2^{b}-1)\\
                %&=((2^b+r-2)/r)/(2^b-1)\\
                &=\left(1+(r-1)/(2^b-1)\right)/r\\
                &\le\left(1+r/2^b\right)/r.
   \end{align*}
   The last inequality follows because $r<u<2^b$.

   % The proof of \ref{eq:ix=iy} follows from {\color{red}Thomas}.
   For \ref{eq:ix=iy},
   let $q=2^{b-1}$ and $p=1/(2q-1)$.
   We define $a\ge 0$ to be the smallest integer, such that $r\setminus q+a$.
   In particular this means 
   $\lceil q/r\rceil = (q+a)/r$ and
   $\lfloor q/r\rfloor = (q-r+a)/r$.

   We bound the sum
   $$
   \Pr[i_x=i_y]
   = \sum_{k=0}^{r-1} \Pr[i_x = k]^2
   $$
   by splitting into three cases:
   1) The case $i_x=0$, where $\Pr[i_x=0]=(2\ceil{q/r}-1)p$,
   2) the $r-a-1$ indices $j$ where $\Pr[i_x=j]=2\ceil{q/r}p$,
   and 3) the $a$ indices $j$ st. $\Pr[i_x=j]=2\floor{q/r}p$.
   \begin{align*}
      \Pr[i_x=i_y]
   &=
   (2p\ceil{ q/r}-p)^2 + (r-a-1) (2p \lceil q/r\rceil)^2 + (r-a) (2p \lfloor q/r \rfloor)^2
 \\&= ((4a+1)r+4(q+a)(q-a-1))p^2/r
 \\&\le (1 + (r^2-r)/(2q-1)^2) / r.
   \end{align*}
   The last inequality comes from maximizing over $a$, which yields $a=(r-1)/2$.

   The result now follows from
   \begin{align}
      (r^2-r)/(2q-1)^2
      \le
      (r-1/2)^2/(2q-1)^2
      \le
      (r/(2q))^2,
   \end{align}
   which holds exactly when $r\le q$.



\end{proof}
Lemma \ref{lem:ix-r-dist} above is all we need to know about the
marginal distribution of $i_x$. However, we also need a replacement
for Lemma \ref{lem:remove-si} for handling the sign-bit $s_x$.
\begin{lemma}\label{lem:remove-si-r-dist} Consider distinct keys
   $x_0,\ldots,x_{j - 1}$, $j\leq k$ and an expression $B=s_{x_0}A$ where $A$
   depends on $i_{x_0},\ldots,i_{x_{j - 1}}$ and $s_{x_1},\ldots,s_{x_{j - 1}}$ but not
   $s_{x_0}$. 
   Then
   \begin{equation}\label{eq:remove-si-r-dist}
      \E[s_xA]=\E[A \mid i_x=0]/p.
   \end{equation}
\end{lemma}
\begin{proof}
   The proof follows the same idea as that for Lemma \ref{lem:remove-si}.
   First we have
   \[\E[B]=\E_{h(x_0) \sim \unif([2^b]\setminus\{0\})}[B]=\E_{h(x_0) \sim \unif[2^b]}[B]2^b/p-\E[B \mid h(x_0)=0]/p.\]
   With $h(x_0)\sim \unif[2^b]$, the bit $a_{x_0}$ is uniform and 
   independent of $j_{x_0}$, so $s_{x_1}\in\{-1,1\}$ is uniform and 
   independent of $i_{x_0}$, and therefore 
   \[\E_{h(x_0) \sim \unif[2^b]}[s_{x_0}A]=0.\]
   Moreover, $h(x_0)=0$ implies $j_x={x_0}$, $i_{x_0}=0$, $a_{x_0}=0$,
   and $s_{x_0}=-1$,
   so 
   \[\E[s_{x_0}A]=-\E[s_{x_0}A \mid h(x_0)=0]/p=\E[A \mid i_{x_0}=0].\]
\end{proof}

From \Cref{lem:remove-si-r-dist} and \eqref{eq:ix=0} we have \nameref{prop:near-independence}
with $\Pr[i_x = 0] \le (1 + r/2^b)/r$, and \eqref{eq:ix=iy} implies that we have
$(1 + (r/2^b)^2)/r$-low collision probability. We can then use
\Cref{lem:count-mersenne} to prove the following result.
\begin{theorem}\label{thm:h-and-s-p-arb-r}
   Let $u$ be a power of two, $1 < r \le u/2$, and let $p=2^b-1>u$ be a
   Mersenne prime.
   Suppose with have a 4-universal hash function $h:[u]\to[2^b-1]$, e.g.,
   generated using \Cref{alg:Mersenne}. Suppose
   $i:[u]\to[r]$ and
   $s:[u]\to\{-1,1\}$ are constructed from $h$ as described in
   \Cref{alg:Mersenne-arb-r}. Using this $i$ and $s$ 
   in the Count Sketch Algorithm \ref{alg:count-sketch}, the second moment 
   estimate $X=\sum_{i\in[k]} C_i^2$ satisfies:
   \begin{align}
      \E[X] &= F_2+(F_1^2-F_2)/p^2, \label{eq:E-F2-p-arb-r}\\
      | \E[X] - F_2 | &\le F_2 (n - 1)/p^2, \label{eq:E-F2-p-com-arb-r}\\
      \Var[X]&< 2(1 + (r/2^b)^2)F_2^2/r.\label{eq:V-F2-p-arb-r}
   \end{align}
\end{theorem}

Now \Cref{lem:count-mersenne} give us \eqref{eq:E-F2-p-arb-r}
and \eqref{eq:E-F2-p-com-arb-r}. Furthermore, we have that
\begin{align*}
    \Var[X] 
        &\le 2F_2^2/r + F_2^2 (2\eps/r + 4(1 + \delta)n / (rp^2) + n^2/p^4 - 2 /(rn))
        \\&= 2(1 + (r/2^b)^2)F_2^2/r + F_2^2(4(1 + r/2^b)n/(r p^2) + n^2/p^4 - 2/(rn)) 
        \\&\le 2(1 + (r/2^b)^2)F_2^2/r + F_2^2(4(1 + r/p)n/(r p^2) + n^2/p^4 - 2/(rn)) .
\end{align*}

We know that $2 \le r \le u/2 \le (p + 1)/4$ and $n \le u$.
This implies that $p \ge 7$ and that $n/p \le u/p \le 4/7$.
If we can prove that $4(1 + r/p)n / (rp^2) + n^2/p^4 - 2 / (rn) \le 0$ then
we have the result. We have that
\begin{align*}
   4(1 + r/p)n / (rp^2) + n^2/p^4 - 2 / (rn)
      &= 4 n/(rp^2) + 4 n /(p^3) + n^2/p^4 - 2/(rn)
      \\&\le 4u/(rp^2) + 4u/(p^3) + u^2/p^4 - 2/(ru) .
\end{align*}
Again we note that $4u/(r p^2) - 2/(ru) = (2u^2 - p^2)/(u p^2 r) \le 0$
since $u \le (p + 1)/2$ so it maximized when $r = u/2$. We then get
that
\begin{align*}
   4u/(r p^2) + 4u/(p^3) + u^2/p^4 - 2/(ru)
      \le 8/p^2 + 4u/(p^3) + u^2/p^4 - 4/u^2 .
\end{align*}
We now use that $u/p \le (4/7)^2$ and get that
\begin{align*}
   8/p^2 + 4u/(p^3) + u^2/p^4 - 4/u^2 .
      \le (8 + 4 (4/7) + (4/7)^2 - 4 (7/4)^2)/p^2
      \le 0 .
\end{align*}
This finishes the proof of \eqref{eq:V-F2-p-arb-r} and thus also of \Cref{thm:h-and-s-p-arb-r}.



% \section{Generalized analysis}

The general analysis use the following strong assumption.
\begin{assumption}\label{ass:independence}
    For distinct keys $x_0, \ldots x_{j - 1}$, $j \le k$
    and an expression $A(i_{x_0}, \ldots, i_{x_{j - 1}}, s_{x_1}, \ldots, s_{x_{j - 1}})$,
    which depends on $i_{x_0}, \ldots, i_{x_{j - 1}}$ and $s_{x_1}, \ldots, s_{x_{j - 1}}$
    but not on $s_{x_0}$. Then
    \begin{align}
        \E[s_{x_0} A(i_{x_0}, \ldots, i_{x_{j - 1}}, s_{x_1}, \ldots, s_{x_{j - 1}})] = 0\; .
    \end{align}
\end{assumption}
We will show our result using the following weaker assumption.
\begin{assumption}\label{ass:near-independence}
    There exists $l \in [r]$ such that for distinct keys $x_0, \ldots x_{j - 1}$, $j \le k$
    and an expression $A(i_{x_0}, \ldots, i_{x_{j - 1}}, s_{x_1}, \ldots, s_{x_{j - 1}})$,
    which depends on $i_{x_0}, \ldots, i_{x_{j - 1}}$ and $s_{x_1}, \ldots, s_{x_{j - 1}}$
    but not on $s_{x_0}$. Then
    \begin{align}\label{eq:near-independence}
        \E[s_{x_0} A(i_{x_0}, \ldots, i_{x_{j - 1}}, s_{x_1}, \ldots, s_{x_{j - 1}})]
            = \E[A(i_{x_0}, \ldots, i_{x_{j - 1}}, s_{x_1}, \ldots, s_{x_{j - 1}}) \mid i_{x_0} = l]/p \; .
    \end{align}
    Furthermore, there exists $\delta$ such that for any key $x$, then
    \begin{align}\label{eq:prob-special-value}
        \Pr[i_x = l] \le (1 + \delta)/r \; .
    \end{align}
\end{assumption}

For both of the analyses we will need the following assumption
\begin{assumption}\label{ass:collision}
    There exists a value $a$ such that for distinct keys $x \neq y$,
    then
    \begin{align}\label{eq:collision}
        \Pr[i_x = i_y] \le (1 + \eps)/r\; .
    \end{align}
\end{assumption}

\paragraph{Uniform bits to arbitrary bins.}
Then \Cref{ass:independence} is satisfied and \Cref{ass:collision} is
satisfied with $\eps = (r/(2q))^2$.

\paragraph{Mersenne to power of two.}
Then \Cref{ass:near-independence} is satisfied with $l = 2^l - 1$ and
$\delta = -(r - 1)/p$ and \Cref{ass:collision} is satisfied
with $\eps = (r - 1)/p^2$.

\paragraph{Mersenne to arbitrary bins.}
Then \Cref{ass:near-independence} is satisfied with $l = 0$ and
$\delta = r/2^b$ and \Cref{ass:collision} is satisfied
with $\eps = (r/2^b)^2$.

\begin{lemma}
    If \Cref{ass:near-independence} is satisfied with $l$
    and $\delta$ and \Cref{ass:collision} is satisfied with
    $\eps$, then
    \begin{align}
        \E[X] &= F_2^2 + (F_1^2 - F_2^2)/p^2 \le (1 + (n - 1)/p^2)F_2^2\\
        \Var[X] &\le 2F_2^4/r + F_2^4 (2\eps/r + 4(1 + \delta)n / (rp^2) + n^2/p^4 - 2 /(rn))
    \end{align}
\end{lemma}
\begin{proof}
    We first bound $\E[s_x s_y f_x f_y [i_x = i_y]]$ for distinct keys
    $x \neq y$. Using \cref{eq:near-independence} twice we get that
    \begin{equation}\begin{split}\label{eq:twice-split}
        \E[s_x s_y f_x f_y [i_x = i_y]]
            &= \E[s_x f_x f_y [i_x = i_y] \mid i_y = l]/p
            \\&= \E[s_x f_x f_y [i_x = l]]/p
            \\&= \E[f_x f_y [i_x = l] \mid i_x = l]/p
            \\&= f_x f_y / p^2 \; .
    \end{split}\end{equation}
    From this we can calculate $\E[X]$.
    \begin{align*}
        \E[X]
            = F_2^2 + \sum_{x \neq y} \E[s_x s_y f_x f_y [i_x = i_y]]
            = F_2^2 + (F_1^2 - F_2^2)/p^2
            \le (1 + (n - 1)/p^2)F_2^2 \; ,
    \end{align*}
    where the last inequality follows from Cauchy-Schwartz. This shows the first result.

    Same method is applied to the analysis of the variance, which is
    \[
        \Var[X]
            = \Var[Y]
            \le \E[Y^2]
            = \sum_{x,y,x',y' \in [u], x \neq y, x' \neq y'} \E[(s_x s_y f_x f_y [i_x = i_y]) (s_{x'} s_{y'} f_{x'} f_{y'}[i_{x'} = i_{y'}])]
        \; .
    \] 
    Consider any term in the sum. Suppose some key, say $x$, is unique in the
    sense that $x \not \in \{y,x',y'\}$. Then we can apply \cref{eq:near-independence}.
    Given that $x \neq y$ and $x'\neq y'$, we have either $2$ or $4$ such unique keys.
    If all 4 keys are distinct, as in \cref{eq:twice-split}, we get
    \begin{align*}
        \E[(s_x s_y f_x f_y [i_x = i_y]) &(s_{x'} s_{y'} f_{x'} f_{y'}[i_{x'} = i_{y'}])]
            \\&= \E[(s_x s_y f_x f_y [i_x = i_y])] \E[s_{x'} s_{y'} f_{x'} f_{y'}[i_{x'} = i_{y'}])]
            \\&= (f_x f_y/p^2)(f_{x'} f_{y'}/p^2)
            \\&= f_x f_y f_{x'} f_{y'}/p^4
        \; .
    \end{align*}
    The expected sum over all such terms is thus bounded
    as 
    \begin{equation}\begin{split}
        \sum_{{\rm distinct}\, x,y,x',y'\in[u]}& \E[(s_x s_y f_x f_y [i_x = i_y]) (s_{x'} s_{y'} f_{x'} f_{y'}[i_{x'} = i_{y'}])]
            \\&= \sum_{{\rm distinct}\,x,y,x',y'\in[u]} f_xf_yf_{x'}f_{y'}/p^4
            \\&\le F_1^4 /p^4
            \\&\le F_2^2 n^2/p^4.\label{eq:distinct}
    \end{split}\end{equation}
    Where the last inequality used Cauchy-Schwartz. We also have to consider all the cases with
    two unique keys, e.g., $x$ and $x'$ unique while $y=y'$. Then using \cref{eq:near-independence}
    and \cref{eq:prob-special-value}, we get
    \begin{align*}
        \E[(s_x s_y f_x f_y [i_x = i_y]) &(s_{x'} s_{y'} f_{x'} f_{y'}[i_{x'} = i_{y'}])]
            \\&= f_x f_{x'} f_y^2 \E[s_x s_{x'} [i_x = i_{x'} = i_y]]
            \\&= f_x f_{x'} f_y^2 \E[s_{x'} [l = i_{x'} = i_y]]/p
            \\&= f_x f_{x'} f_y^2 \E[l = i_y]/p^2
            \\&\le f_x f_{x'} f_y^2(1 + \delta)/(rp^2).
    \end{align*}    
    Summing over all terms with $x$ and $x'$ unique while $y=y'$, and
    using Cauchy-Schwartz and $u\leq p$, we get 
    \begin{align*}
        \sum_{{\rm distinct}\,x,x',y} f_xf_{x'}f_y^2 (1 + \delta) /(rp^2) 
            \le F_1^2F_2^2 (1 + \delta)/(rp^2)
            \le F_2^4 n(1 + \delta)/(rp^2).
    \end{align*}
    There are four ways we can pick the two unique keys $a\in \{x,y\}$
    and $b\in \{x',y'\}$, so we conclude that
    \begin{equation}\label{eq:one-pair}
        \sum_{\begin{array}{c}
            x,y,x',y'\in[u], x\neq y, x'\neq y',\\
            {\rm two\ keys\ are\ unique}
        \end{array}}
        \E[(s_x s_y f_x f_y [i_x = i_y]) (s_{x'} s_{y'} f_{x'} f_{y'}[i_{x'} = i_{y'}])]
            \le 4 F_2^4 n(1 + \delta)/(rp^2) .
    \end{equation}
    Finally, we need to reconsider the terms with two pairs, that
    is where $(x,y)=(x',y')$ or $(x,y)=(y',x')$. In
    this case, $(s_x s_y f_x f_y [i_x = i_y]) (s_{x'} s_{y'} f_{x'} f_{y'}[i_{x'} = i_{y'}]) = f_x^2 f_y^2 [i_x = i_y]$.
    By \cref{eq:collision}, we get 
    \begin{equation}\begin{split}    
        \sum_{\begin{array}{c}
            x,y,x',y'\in[u], x\neq y, x'\neq y',\\
            (x,y)=(x',y')\,\vee\,(x,y)=(y',x')
        \end{array}}&
            \E[(s_x s_y f_x f_y [i_x = i_y]) (s_{x'} s_{y'} f_{x'} f_{y'}[i_{x'} = i_{y'}])]
            \\&=2\sum_{x,y\in[u],x\neq y} f_x^2f_y^2 \Pr[i_x=i_y]
            \\&=2\sum_{x,y\in[u],x\neq y} f_x^2f_y^2 (1 + \eps)/r
            \\&=2(F_2^4 - F_4^4)(1 + \eps)/r .\label{eq:two-pairs}
    \end{split}\end{equation}
    Adding up add \req{eq:distinct}, \req{eq:one-pair}, and
    \req{eq:two-pairs}, we get 
    \begin{align*}
        \Var[Y]
            &\le 2(1 + \eps)(F_2^4 - F_4^4)/r + F_2^4(4(1 + \delta) n / (rp^2) + n^2/p^4)
            \\&\le 2F_2^4/r + F_2^4 (2\eps/r + 4(1 + \delta)n / (rp^2) + n^2/p^4 - 2 /(rn)) .
    \end{align*}
    This finishes the proof.
\end{proof}

\begin{corollary}
    We get the following results
    \begin{itemize}
        \item \textbf{Mersenne to power of two:}
            \[
                \Var[X] \le 2 F_2^4/r .
            \]
        \item \textbf{Mersenne to arbitrary number of bins:}
            \[
                \Var[X] \le 2 (1 + (r/(p + 1))^2) F_2^4/r .
            \]
    \end{itemize}
\end{corollary}
\begin{proof}
    We will use that we know that $2 \le r \le u/2 \le (p + 1)/4$ and $n \le u$.
    This implies that $p \ge 7$ and that $n/p \le u/p \le 4/7$.

    \paragraph{Mersenne to power of two.} We then know that $\delta = -(r - 1)/p \le 0$ 
    and $\eps = (r - 1)/p^2 \le r/p^2$. We want to prove that
    $2\eps/r + 4(1 + \delta)n / (rp^2) + n^2/p^4 - 2/(rn) \le 0$ which would
    prove our result. We get that
    \begin{align*}
        2\eps/r + 4(1 + \delta)n / (rp^2) + n^2/p^4 - 2/(rn) 
            &\le 2/p^2 + 4n/(r p^2) + n^2/p^4 - 2/(rn)
            \\&\le 2/p^2 + 4u/(r p^2) + u^2/p^4 - 2/(ru) .
    \end{align*}
    Now we note that $4u/(r p^2) - 2/(ru) = (2u^2 - p^2)/(u p^2 r) \le 0$
    since $u \le (p + 1)/2$ so it maximized when $r = u/2$. We then get
    that
    \begin{align*}
        2/p^2 + 4u/(r p^2) + u^2/p^4 - 2/(ru)
            \le 2/p^2 + 8/p^2 + u^2 / p^4 - 4/u^2 .
    \end{align*}
    We now use that $u/p \le (4/7)^2$ and get that
    \begin{align*}
        2/p^2 + 8/p^2 + u^2 / p^4 - 4/u^2
            \le (10 + (4/7)^2 - 4 (7/4)^2)/p^2
            \le 0 .
    \end{align*}
    This finishes the first part.
    
    \paragraph{Mersenne to arbitrary number of bins.} We then know that
    $\delta = r/(p + 1) \le r/p$ and $\eps = r^2/(p + 1)^2$.
    We have that
    \begin{align*}
        \Var[X]
            &= 2F_2^4/r + F_2^4 (2\eps/r + 4(1 + \delta)n / (rp^2) + n^2/p^4 - 2 /(rn))
            \\&= 2(1 + \eps)F_2^4/r + F_2^4(4(1 + \delta)n / (rp^2) + n^2/p^4 - 2 / (rn))
            \\&\le 2(1 + r^2/(p + 1)^2) F_2^4/r + F_2^4(4(1 + r/p)n / (rp^2) + n^2/p^4 - 2 / (rn))
        .
    \end{align*}
    If we can prove that $4(1 + r/p)n / (rp^2) + n^2/p^4 - 2 / (rn) \le 0$ then
    we have the result. We have that
    \begin{align*}
        4(1 + r/p)n / (rp^2) + n^2/p^4 - 2 / (rn)
            &= 4 n/(rp^2) + 4 n /(p^3) + n^2/p^4 - 2/(rn)
            \\&\le 4u/(rp^2) + 4u/(p^3) + u^2/p^4 - 2/(ru) .
    \end{align*}
    Again we note that $4u/(r p^2) - 2/(ru) = (2u^2 - p^2)/(u p^2 r) \le 0$
    since $u \le (p + 1)/2$ so it maximized when $r = u/2$. We then get
    that
    \begin{align*}
       4u/(r p^2) + 4u/(p^3) + u^2/p^4 - 2/(ru)
            \le 8/p^2 + 4u/(p^3) + u^2/p^4 - 4/u^2 .
    \end{align*}
    We now use that $u/p \le (4/7)^2$ and get that
    \begin{align*}
        8/p^2 + 4u/(p^3) + u^2/p^4 - 4/u^2 .
            \le (8 + 4 (4/7) + (4/7)^2 - 4 (7/4)^2)/p^2
            \le 0 .
    \end{align*}
    This finishes the second part.
\end{proof}


\section{Division and Modulo with Generalized Mersenne Primes}

%%! TEX root = ../../mersenne.tex

\title{
   %Mersenne Division
   A General Algorithm for Generalized Mersenne Primes
}
\author{Thomas Dybdahl Ahle, Jakob Tejs Bæk Knudsen}
\maketitle

\begin{abstract}
   We give a simple algorithm for division and modulo calculation with
   Generalized and Pseudo-Mersenne Primes, that is prime numbers on the form $q-c$.
   Our algorithm is competitive with the state of the art, while implementable in two lines of code.

   This includes important cryptographic primes such as $p_{192}=2^{192} - 2^{64} - 1$.
   For $x < (q/c)^n$ our algorithm times time $n$, and uses no more registers than it takes to store $x$.
\end{abstract}

%\tableofcontents
%! TEX root = ../merdiv.tex

\subsection{Introduction}

Primes on the form $2^n-1$ are known as Mersenne primes, and are used all over Cryptography and Computer Science because they allow quicker algorithms than primes in general.
This allows speeding up finite field arithmetic,
since $c\bmod (2^n-1) = (c\bmod 2^n) + \lfloor k/n\rfloor$ if the value is smaller than $p=2^n-1$ (otherwise just subtract $p$.)

In this paper we investigate division by such primes, as well as two generalized classes:
\begin{description}
   \item[Generalized Mersenne Primes]
      also sometimes known as Solinas primes~\cite{Solinas2011}, are sparse numbers, that is $f(2^m)$ where $f(x)$ is a low-degree polynomial.
      Examples are the primes in NIST's document "Recommended Elliptic Curves for Federal Government Use": $p_{192} = 2^{192} - 2^{64} - 1$ and $p_{384} = 2^{384}-2^{128}-2^{96}+2^{32}-1$.
   \item[Pseudo-Mersenne Primes]
      also sometimes known as Crandall primes~\cite{crandall1992method}, are numbers on the form $2^n - c$ for a small odd $c$.
      It is sometimes required that $c < 2^{\lfloor n/2\rfloor}$~\cite{van2014encyclopedia}.
\end{description}

We also give a simple algorithm for taking the modulo, which works efficiently for all Pseudo-Mersenne Primes.


%! TEX root = ../merdiv.tex

\subsubsection{Applications}

\paragraph{$k$-wise hashing without modulus}

A very common algorithm is to define
$h(x) = \sum_{i=0}^k a_i x^i \mod p$
for some prime $p$.

This is efficient if $p$ supports fast modulus.
However now the range of $h$ is $[p]$, which may not be what we are looking for.
Hence it is common to take $h(x)\mod m$ for some $m<p$.
However $m$ is a general number, and can't be expected to support fast modulus.

An alternative is the following:
\begin{align}
   \left\lfloor\frac{h(x)m}{p}\right\rfloor.
\end{align}
This only requires division with $p$, which we have shown can be done fast.

An alternative is to compute $q = \lfloor2^n/p\rfloor$ and take
\begin{align}
\left\lfloor
   \frac{h(x)mq}2^n
\right\rfloor
\end{align}
but that requires using larger words, and has some other drawbacks\todo{Which ones?}

%! TEX root = ../merdiv.tex

\subsection{Algorithm}

The proposed algorithm takes integers $n>c>0$ and $x$, and returns
$\left\lfloor\frac{x}{2^n-c}\right\rfloor$.
The algorithm performs two additions, a shift, and a multiplication with $c$ for each memory word occupied by $x$.
Thus, in the case of Generalized Mersenne Primes with sparsity $k$, we can use a total of $3+k$ simple operations.
\begin{algorithm}
   %\caption{}\label{euclid}
   \begin{algorithmic}[1]
      \Procedure{Divide}{x, n, c}
         \State \textbf{let} $m$ \text{such that} $x \le (2^n/c)^m$.
         \State $v \gets 0$
         %\For{$i\gets 1$ \textbf{to} $m$}
         \For{ $m$ times}
            \State $v \gets (v + 1)c+x$ \textbf{shift-right} $n$.
         \EndFor
         %\EndFor
         \State\textbf{return} $v$
      \EndProcedure
   \end{algorithmic}
\end{algorithm}

%Note that for many Pseudo Mersenne Primes, multiplication by $c$ can be replaced by a single shift, making the algorithm completely additions and shifts.

%Note that we use the same number of operations independent of $x$.


To compute modulo we use the fact that
\begin{align}
   x \bmod p
   = x - p\left\lfloor\frac{x}{p}\right\rfloor
   = x - (2^n - c)\left\lfloor\frac{x}{2^n-c}\right\rfloor,
\end{align}
which is only two additions, a shift, and a multiplication with $c$ on top of the division algorithm.
As $\left\lfloor\frac{x}{p}\right\rfloor p \le x$ there is no danger of overflow.

In fact our algorithm does something stronger, namely gives an efficient division algorithm for any number on the form $q-c$ where $q$ supports fast division.
The intuition for the algorithm is the expansion
\begin{align}
   \frac{x}{q-c}
   = \frac{x}{q}\sum_{i=0}^\infty \left(\frac{c}{q}\right)^i
   = x\frac{1+\frac{c+\frac{c^2 + \dots}{q}}{q}}{q}
   = \frac{x+c\frac{x+c\frac{x + \dots}{q}}{q}}{q},
\end{align}
however curiously we need to add a $+1$ to get efficient convergence.

\begin{theorem}
   %Given integers $q>c>0$, $n\ge 0$ and
   \todo{Do we support negative $c$? Can we instead say $q>|c|>0$?}
   $0\le x \le (q/c)^{n-1}(q-c)$,
   \todo{Given the Corollary, we can also write this as $x < (q/c)^n$, which may be easier to understand?}
   then
   \begin{align}
      \left\lfloor\frac{x}{q-c}\right\rfloor = v_n,
   \end{align}
   where $v_0 = 0$ and
   $v_{i+1} = \left\lfloor\frac{(v_i+1)c+x}{q}\right\rfloor$.
\end{theorem}
\begin{proof}
   Write $x = m(q-c)+h$ for non-negative integers $m$ and $h$ with $h<q-c$.
   Then we get
   \begin{align}
      \left\lfloor\frac{x}{q-c}\right\rfloor = m.
      \label{eq:floor}
   \end{align}

   Let $u_0=0$, $u_{i+1} = \lfloor\frac{q}{c}u_i+1\rfloor$.
   By induction $u_i \ge (q/c)^{i-1}$ for $i>0$.
   This is trivial for $i=1$ and $u_{i+1}=\lfloor \frac qc u_i +1\rfloor \ge \lfloor (q/c)^i + 1 \rfloor \ge (q/c)^i$.

   Now define $E_i\in\mathbb Z$ such that $v_i = m - E_i$.
   We will show by induction that $0\le E_{i} \le u_{n-i}$ for $0\le i\le n$ such that $E_n = 0$, which gives the theorem.
   For a start $E_0=m\ge 0$ and $E_0 = \lfloor x/(q-c)\rfloor \le (q/c)^{n-1} \le u_n$.

   For the induction step we plug in our expressions for $x$ and $v_i$:
   \begin{align}
      v_{i+1}
      &= \left\lfloor \frac{(m-E_i+1)c+m(q-c)+h}{q}\right\rfloor
    =
    m
    +
    \left\lfloor \frac{(- E_i+1)c +h}{q}\right\rfloor
    =
    m
    - \left\lceil \frac{(E_i-1)c - h}{q}\right\rceil.
   \end{align}
   Now by the induction hypothesis,
   \begin{align}
      E_{i+1}
      = \left\lceil \frac{(E_i-1)c - h}{q}\right\rceil
      \le\left\lceil (u_{n-i}-1)\frac{c}{q}\right\rceil
      = \left\lceil \left\lfloor \frac{q}{c}u_{n-i-1} \right\rfloor \frac{c}{q}\right\rceil
      \le \left\lceil u_{n-i-1}\right\rceil
      = u_{n-i-1}.
   \end{align}
   and similarly since $E_i \ge 0$ and $h\le q-c-1$,
   we have
   $\left\lceil \frac{E_ic - h - c}{q}\right\rceil \ge
   \left\lceil \frac{- q + 1}{q}\right\rceil = 0$, which completes the proof.
\end{proof}
\begin{corollary}
   For $c=1$ we can be slightly more specific, and support $x < q^n$.
   In particular, for $c=1$ and $x < q^2$ we have
   $
   \left\lfloor\frac{x}{q-1}\right\rfloor
   = v_2
   = \left\lfloor\frac{\left\lfloor\frac{x+1}{q}\right\rfloor+x+1}{q}\right\rfloor
   $.
\end{corollary}
\begin{proof}
   This follows by repeating the proof, but noting that $u_i = \frac{q^i-1}{q-1}$ since all the $q/c$ terms are integral.
\end{proof}

%! TEX root = ../../mersenne.tex

\subsection{Experiments}

\todo{Jakob made some promising experiments.}

%! TEX root = ../../mersenne.tex

\subsection{Related Work}

Simple Power Analysis on Fast Modular Reduction with
Generalized Mersenne Prime for Elliptic Curve Cryptosystems

https://pdfs.semanticscholar.org/3f84/39b357a8331f1cc6f8de68a19223ff027f2c.pdf

\begin{algorithm}
   \caption{Fast reduction modulo $p_{192} = 2^{192} - 2^{64} - 1$}
   \begin{algorithmic}
      \State \textbf{input} $c \gets (c_5, c_4, c_3, c_2, c_1, c_0)$, where each $c_i$ is a 64-bit word, and $0 \le c < p^2_{192}$.
      \State $s_0 = (c_2, c_1, c_0)$
      \State $s_0 = (0, c_3, c_3)$
      \State $s_0 = (c_4, c_4, 0)$
      \State $s_0 = (c_5, c_5, c_5)$
      \State \textbf{return} $s_0 + s_1 + s_2 + s_3 \mod p_{192}$.
   \end{algorithmic}
\end{algorithm}
That's weird.

In~\cite{granger2013generalised} the authors defined a different family of Generalised Mersenne numbers and showed various fast multiplication and reduction schemes.

Other methods:
 - Classical one using divisions
 - Montgomery's method
 - ! Barrett reduction

Modified Crandall Algorithm:
https://ieeexplore.ieee.org/stamp/stamp.jsp?arnumber=4016496

Does modulo in parallel with division (which we have shown is unnecessary), but taking only the division parts the algorithm is
\begin{algorithm}
   \begin{algorithmic}[1]
      \Procedure{Divide}{x, n, c}
         \State $q_0 \gets \lfloor x/2^n\rfloor$
         \State $r_0 \gets x \bmod 2^n$
         \State $q \gets q_0$
         \State $r \gets r_0$
         \State $i \gets 0$
         \While{$q_i>0$}
            \State $t \gets q_i c$
            \State $q_{i+1} \gets \lfloor t / 2^n\rfloor$
            \State $r_{i+1} \gets t \bmod 2^n$
            \State $q\gets q+q_{i+1}$
            \State $r\gets r+r_{i+1}$
            \State $i\gets i+1$
         \EndWhile
         \State $t \gets 2^n-c$
         \While{$r\ge t$}
            \State $r\gets r-t$
            \State $q\gets q+1$
         \EndWhile
         \State\textbf{return} $q$
      \EndProcedure
   \end{algorithmic}
\end{algorithm}

No proof of correctness is given.
The while loop condition is different.
No guarantees on running time.
The addition is different from ours.
Doesn't add 1. Doesn't add $x$.
Has that extra weird loop for fixing things in the end.
So it actually has to do the $r$ computation?


Words:
 GM, PM reduction: Module af Generalized or Pseudo-Mersennes.

The most important case is ``Modular Multiplication'', in which we are given two $w$-word numbers, $x,y$ and want to compute $xy\mod p$ which is again a $w$-word number.

Montgomery method:
\begin{align}
   (aR\mod N)(bR\mod N) \mod N = (abR)R \mod N
\end{align}
We then need to remove the factor of $R$ by multiplying with its inverse $\mod N$.



Unfortunately there are only 45 of them known.
The most useful one perhaps being.

Heuristically there are $O(\log x)$ Mersenne primes up to $x$.


Trivia:
Euler proved that an even number $n$ is perfect if and only if it is on the form $n=2^{q-1}M_q$, where $M_q=2^q-1$ is prime.
(Usually we know a number is perfect if its divisors sum to the number itself, e.g. $6=1+2+3$ or $28=1+2+4+7+14$.)

Four of the recommended primes in NIST's document "Recommended Elliptic Curves for Federal Government Use" are Solinas primes:


%[[Curve448]] uses the Solinas prime <math>2^{448} - 2^{224} - 1</math>

The classical Mersenne mod algorithm.
Write $n = a2^p + b$, then
\begin{align}
   n \mod M_p = \begin{cases}
      a + b & \text{if } a+b<M_p\\
      a + b + 1 - 2^p & \text{otherwise}.
   \end{cases}
\end{align}

Thus reduction modulo a Mersenne prime requires an
integer addition, as opposed to an integer division for
modular reduction in the general case. There are two
drawbacks to this method of modular reduction:
\begin{itemize}
   \item Finding the integers a and b is easiest when p is a multiple of word size of the machine, since then there is
no actual shifting of bits needed to align a and b for
the modular addition. But word sizes are in practice
powers of two, whereas p must be an odd prime.
   \item The Mersenne primes are so rare, with none between
      $M_{127}$ and $M_{521}$ hat usually there will be none of the desired magnitude.
\end{itemize}
For these reasons~\cite{van2014encyclopedia}, cryptographers tend not to use Mersenne primes, preferring similar moduli such as pseudo-Mersenne primes and generalized Mersenne primes.

\paragraph{Pseudo-Mersenne primes}
On the form $p = 2^n-k$.
If $x < p^2$, write $x = a 2^{2n} + b 2^m + c$,
where $a\in\{0,1\}$.
(Note, if $k > 0$, we always have $a=0$.)
Then $n \mod p = ak^2+bk+c$.
Repeating this a few times gives the calculation.


\paragraph{Generalised Mersenne primes}
Solinas algorithm

%! TEX root = ../../mersenne.tex
\subsection{Mess}

Words:
 GM, PM reduction: Module af Generalized or Pseudo-Mersennes.

The most important case is ``Modular Multiplication'', in which we are given two $w$-word numbers, $x,y$ and want to compute $xy\mod p$ which is again a $w$-word number.

Montgomery method:
\begin{align}
   (aR\mod N)(bR\mod N) \mod N = (abR)R \mod N
\end{align}
We then need to remove the factor of $R$ by multiplying with its inverse $\mod N$.



Unfortunately there are only 45 of them known.
The most useful one perhaps being.

Heuristically there are $O(\log x)$ Mersenne primes up to $x$.


Trivia:
Euler proved that an even number $n$ is perfect if and only if it is on the form $n=2^{q-1}M_q$, where $M_q=2^q-1$ is prime.
(Usually we know a number is perfect if its divisors sum to the number itself, e.g. $6=1+2+3$ or $28=1+2+4+7+14$.)

Four of the recommended primes in NIST's document "Recommended Elliptic Curves for Federal Government Use" are Solinas primes:


%[[Curve448]] uses the Solinas prime <math>2^{448} - 2^{224} - 1</math>

The classical Mersenne mod algorithm.
Write $n = a2^p + b$, then
\begin{align}
   n \mod M_p = \begin{cases}
      a + b & \text{if } a+b<M_p\\
      a + b + 1 - 2^p & \text{otherwise}.
   \end{cases}
\end{align}

Thus reduction modulo a Mersenne prime requires an
integer addition, as opposed to an integer division for
modular reduction in the general case. There are two
drawbacks to this method of modular reduction:
\begin{itemize}
   \item Finding the integers a and b is easiest when p is a multiple of word size of the machine, since then there is
no actual shifting of bits needed to align a and b for
the modular addition. But word sizes are in practice
powers of two, whereas p must be an odd prime.
   \item The Mersenne primes are so rare, with none between
      $M_{127}$ and $M_{521}$ hat usually there will be none of the desired magnitude.
\end{itemize}
For these reasons~\cite{van2014encyclopedia}, cryptographers tend not to use Mersenne primes, preferring similar moduli such as pseudo-Mersenne primes and generalized Mersenne primes.

\paragraph{Pseudo-Mersenne primes}
On the form $p = 2^n-k$.
If $x < p^2$, write $x = a 2^{2n} + b 2^m + c$,
where $a\in\{0,1\}$.
(Note, if $k > 0$, we always have $a=0$.)
Then $n \mod p = ak^2+bk+c$.
Repeating this a few times gives the calculation.


\paragraph{Generalised Mersenne primes}
Solinas algorithm



\bibliographystyle{alpha}
\bibliography{general}

\end{document}




