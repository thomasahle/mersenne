\documentclass[11pt]{article}  
%\renewcommand\baselinestretch{0.95}
\usepackage{amsmath}
\usepackage{amsthm}
\usepackage{amsfonts}
\usepackage{amssymb}

\usepackage{times}
\usepackage{fullpage}
\usepackage{epsfig}
\usepackage{graphicx}
\usepackage{epstopdf}
\usepackage{todonotes}
\usepackage{hyperref}
\usepackage{cleveref}
\usepackage{algorithm}
\usepackage[noend]{algpseudocode}
%\usepackage[ruled,vlined,commentsnumbered,titlenotnumbered]{algorithm2e}
\newcommand{\suchthat}{\mathrel{}\mathclose{}\ifnum\currentgrouptype=16\middle\fi\vert\mathopen{}\mathrel{}}
%\newcommand{\E}{{\sf E}}

\DeclareMathOperator*{\E}{E}
\DeclareMathOperator*{\Var}{Var}

\newcommand{\ppmod}{\rule{-1.5ex}{0ex}\pmod}
\newcommand{\floor}[1]{\lfloor {#1} \rfloor}
\newcommand{\ceil}[1]{\lceil {#1}\rceil}
\newcommand{\Prp}[1]{\Pr\left[{#1} \right]}
\newcommand{\Ep}[1]{{\E}\left[{#1} \right]}
\newcommand{\req}[1]{(\ref{#1})}
\newcommand\eps\varepsilon
\newcommand\Z{\mathbb Z}

\newtheorem {lemma} {Lemma}[section]
\newtheorem {fact} [lemma] {Fact}
\newtheorem {assumption} {Assumption}
\newtheorem {definition} {Definition}
\newtheorem {corollary} [lemma] {Corollary}
\newtheorem {theorem}[lemma] {Theorem}
\newtheorem {observation}[lemma] {Observation}
\newtheorem {question}{Question}[section]
\newtheorem {exercise}[question]{Exercise}

\newcommand{\unif}{\mathcal{U}}


% Eva's comments in a different colour
\usepackage{color}
\usepackage{xcolor}
\newcommand{\er}[1]{\textcolor{blue}{#1}}
\newcommand{\erdel}[1]{\textcolor{LightGreen}{#1}}

% Drawing
\newcommand{\andtt}{ \mathbin{\texttt{\&}} }
\newcommand{\xor}{\oplus}
\newcommand{\ls}{ \mathbin{\texttt{<\!<}} }
\newcommand{\rs}{ \mathbin{\texttt{>\!>}} }


\title{The Power of Hashing with Mersenne Primes}
\author{Thomas Dybdahl Ahle, Jakob Tejs Bæk Knudsen, Mikkel Thorup}

\begin{document}
\maketitle

\begin{abstract}
The classic way of computing a $k$-universal hash function is to use a random degree-$(k-1)$ polynomial over a prime field $\mathbb Z_p$.
For a fast computation of the polynomial, the prime $p$ is often chosen as a Mersenne prime $p=2^b-1$.

In this paper, we show that there are other nice advantages to using Mersenne primes.
Our view is that the output of the hash function is a $b$-bit integer that is uniformly distributed in $[2^b]$, except that $p$ (the all \texttt1s value) is missing.
Uniform bit strings have many nice properties, such as splitting into substrings, which
%Thinking of the hash values as almost uniform $b$-bit integers
leads to simple efficient code with strong theoretical qualities.
We will demonstrate this with focus on the 4-universal hashing in the classic count-sketch for second moment estimation.

From an algorithmic perspective we provide a new algorithm for division and modulus with Pseudo-Mersenne primes
$p=2^b-c$ for small $c$,
which improves upon a classical algorithm of Crandall, and 
expands the availability and speed of Mersenne based techniques.
\end{abstract}

\tableofcontents

%! TEX root = ../mersenne.tex
\section{Introduction}

\begin{figure}
   \centering
   \begin{tikzpicture}[darkstyle/.style={circle,draw,fill=gray!40,minimum size=20}]
      \newcommand*{\figb}{8}
      % The red box
      \draw[pattern=north west lines, pattern color=red] (0,0) rectangle (\figb,1);
      % Horizontal lines
      \foreach \y in {0,...,6}
         \draw (0, \y) -- (\figb, \y);
      % Vertical lines
      \foreach \x in {0,...,\figb}
         \draw (\x, 0) -- (\x, 6);
      % Most of the numbers
      \pgfmathsetmacro{\figbthree}{\figb - 3}
      \pgfmathsetmacro{\figbtwo}{\figb - 2}
      \pgfmathsetmacro{\figbone}{\figb - 1}
      \foreach \y in {0,...,2}
         \foreach \x in {0,...,\figbthree}
            \node [draw=none] at (.5+\x,.5+\y) {1};
      \foreach \x in {\figbthree,...,\figbone}
         \node [draw=none] at (.5+\x,.5+3) {.};
      \foreach \y in {4,...,5}
         \foreach \x in {0,...,\figbtwo}
            \node [draw=none] at (.5+\x,.5+\y) {0};
      % The rest of the numbers
      \node [draw=none] at (.5+\figb-1,.5+0) {1};
      \node [draw=none] at (.5+\figb-2,.5+0) {1};
      \node [draw=none] at (.5+\figb-1,.5+1) {0};
      \node [draw=none] at (.5+\figb-2,.5+1) {1};
      \node [draw=none] at (.5+\figb-1,.5+2) {1};
      \node [draw=none] at (.5+\figb-2,.5+2) {0};
      \node [draw=none] at (.5+\figb-1,.5+4) {1};
      \node [draw=none] at (.5+\figb-1,.5+5) {0};
   \end{tikzpicture}
   \caption{The output of a random polynomial modulo $p=2^b-1$ is uniformly distributed in $[p]$. This means that each bit has the same identical distribution, which is only $1/p$ biased towards 0.}
   \label{fig:bits}
\end{figure}

Hashing is often used in the sketching of high volume data streams, such as traffic through an Internet router, and then this speed is critical to keep up with the stream.
%
%
While hashing with Mersenne primes has been used for more than 40 years by anyone who
%, for $k>2$,
wanted to implement k-universal hashing efficiently using standard portable code.
Here we argue that uniform hash values from a Mersenne prime field with prime $p=2^b-1$
are not only fast to compute,
but have special advantages different from any other prime field.

We believe we are the first to notice that such values
can largely be treated as uniform $b$-bit strings, that is, we can use
the tool box of very simple and efficient tricks for uniform
$b$-bit strings.
From $[2^b]$ we are missing $p$, the all \texttt1s value, but a careful analysis shows that this has only minor effect on the quality of the outcome.
To put our results in perspective, suppose we were hashing $n$ keys uniformly into $b$-bit strings.
The probability that any of them hash to $p=2^b-1$ is at most $n/p$.
This means the total variation between the two distributions is $n/p$ and any error probability we might have proved assuming uniform $b$-bit hash values is off by at most $n/p$.
%This may be good if $p/n$ is sufficiently large.
\emph{In contrast, our analysis yields error probability that differs from the uniform case by $n/p^2$ or less!}
%This allows good error bounds even as we set $p$ close to $n$, which again means that half the word operations can saved.
This means that we for a desired error can reduce the bit-length of the primes to less than half. This saves not only space. It typically means that we can speed up the multiplications with at least a factor 2.

Many of the ideas presented here would not apply at all if we had a prime $2^b-a$ with $a>1$, e.g., $a=3$, so our work is very particular to Mersenne primes.
Primes on the form $2^b-a$ with $a>1$ are known as Pseudo-Mersenne Primes~\cite{van2014encyclopedia}, and are used when a Mersenne prime of a reasonable magnitude isn't available.
Supplementing our main analysis, we provide a new algorithm for division and hashing with such numbers, which improves upon classical cryptographic work by Crandall~\cite{crandall1992method}.


\subsection{Polynomial hashing using Mersenne primes}

The $k$-universal hashing with a polynomial uses $O(k)$ space and $O(k)$ time
to compute the hash value of a key. Siegel \cite{Siegel04} has proved that if we want $k$-universal hashing in time $t<k$, then we need to use space $u^{1/t}$.
Such tabulation based methods are useful in many contexts (see survey \cite{Thorup17}, but not if we need small space.

A classic example where constant space hash functions are needed is in static two-level hash functions \cite{FKS84}.
To store n keys with constant access time, they n second level hash tables, each with its own  hash function.
Another example is small sketches such as the count sketch \cite{XXX}\todo{What do we cite here?} discussed in this paper. Here we may want to store the hash function as part of the sketch, e.g., to query the value of a given key.


\subsubsection{Preliminaries: Implementation of a Hash Function}
The classic definition of $k$-universal hashing
goes back to Carter and Wegman~\cite{wegman81kwise}.
\begin{definition}
   A random hash function $h:U\to R$ is $k$-universal if for $k$
   distinct keys $x_0,\ldots,x_{k-1}\in U$, the values
   $h(x_0),\ldots,h(x_{k-1})$ are uniform in $R^k$.
\end{definition}
\noindent
Note that the definition also implies the values 
$h(x_0),\ldots,h(x_{k-1})$ are independent.
A very similar concept is that of $k$-independence, which has only this requirement, but doesn't include that values must be uniform.

The classic example of $k$-universal
hash function is uniformly random degree-$(k-1)$ polynomial over a prime field
$\Z_p$, that is, we pick a uniformly random vector
$\vec a=(a_0,\ldots,a_{k-1})\in \Z_p^k$ of $k$ coefficients, and define
$h_{\vec a}:[p]\to[p]$,
\footnote{ We use the notation $[s]=\{0,\ldots,s-1\}$.  }
   by 
\[h_{\vec a}(x)=\sum_{x\in[k]}a_i x^i \mod p.\]
%
Generally, we are given a key domain $[u]$ and a smaller range $[r]$ for the hash values.
It is natural to define
$h^r_{\vec a}:[u]\to[r]$ by
\[h^r_{\vec a}(x)=h_{\vec a}(x)\bmod r.\]
Assuming $p\geq \max\{u,r\}$, the hash values of $k$ distinct keys remain independent,
while staying as close as possible to the uniform distribution on $[r]$.
(This will turn out to be very important.)

In terms of speed, the main bottleneck in the above approach is the mod operations.
If we assume $r=2^\ell$, the $\bmod r$ operation above can be replaced by a binary AND: $x \bmod r = x \andtt r-1$.
In the same vein, an old idea by 
Carter and Wegmen \cite{carter77universal} is to use a
Mersenne prime for $p=2^b-1$,\footnote{e.g., $p=2^{61}-1$ for hashing 32-bit keys or
$p=2^{89}-1$ for hashing 64-bit keys.}
to speed up the computation of the (mod $p$) operations.
The point is that
\begin{equation}
   y \bmod (2^b-1)
   \equiv (y\bmod 2^{b}) + \floor{y/2^b}
   \equiv (y \andtt p) + (y \rs b)
   \pmod {p}.
   \label{eq:Mersenne}
\end{equation}
%This leads to efficient computations because 
%\[y\bmod 2^{b}=y\andtt p\quad\textnormal{and}\quad\floor{y/2^b}=y\rs b.\]
Again allowing us to use the very fast bit-wise {\sc and} ($\andtt$) and the right-shift ($\rs$),
instead of the expensive modulo operation.


\vspace{1em}

The above completes our description of how Mersenne primes are
normally used for fast computation of $k$-universal hash functions.
We show an implementation in \cref{alg:Mersenne} with one further improvement:
By assuming that $p=2^b-1\geq 2u-1$
(which is automatically satisfied in the typical case where $u$ is a power
of two, e.g., $2^{32}$ or $2^{64}$)
we can get away with only testing a possible off-by-one in \cref{eq:Mersenne} once, rather that at every loop.
Note the proof by loop invariant in the comments.

\begin{algorithm}[H]
   \caption{
      For $x\in [u]$, prime $p=2^b-1\geq 2u-1$,
      and $\vec a=(a_0,\ldots,a_{k-1})\in[p]^k$,
      computes $y=h_{\vec a}(x)=\sum_{i\in[q]}a_i x^i\mod p$.
   }\label{alg:Mersenne}
   \begin{algorithmic}
      \State $y\gets a_{k-1}$
      \For{$i=q-2,\ldots,0$}
      \Comment{Invariant: $\quad y<2p$}

      \State $y\gets y*x+a_i$
      \Comment{$\quad y<2p(u-1)+(p-1)<(2u-1)p\leq p^2$}

      \State $y\gets (y\andtt p)+(y\rs b)$
      \Comment{$\quad y<p+p^2/2^b<2p$}
      \EndFor
      \If{$y\geq p$}
      \State $y\gets y-p$
      \Comment{$y<p$}
      \EndIf
      %\State \Return $y$
   \end{algorithmic}
\end{algorithm}


In \Cref{subsec:intro-division} we will give one further improvement to \Cref{alg:Mersenne}.
Mostly the description above is a fairly standard description of state-of-the-art hashing.
\footnote{We note that $k=2$, we do have the fast multiply-shift scheme of Dietzfelbinger~\cite{dietzfel96universal}, that directly gives 2-universal
hashing from $b$-bit strings to $\ell$-bit strings, but for $k>2$,
there is no faster method that can be implemented with portable code
in a standard programming language like C.}

We stress that while this is a particularly fast implementation of Mersenne prime hashing, the main novelty of the paper will be in the analysis.

%The main point of this paper is our note from before, that the values hashed to $[r]$ are not completely uniform, as they would have been if $p=2^b$ was a prime.
%It turns out that with a novel analysis, bits from Mersenne primes are actually
%almost as good as if
%they were uniformly distributed $b$-bit strings (we are only missing
%the all \texttt{1}s value $2^b-1$). 





\subsubsection{Good bucketing with powers of two}\label{sec:power-of-two}
As a first illustration of the advantage that we get using a
Mersenne prime $p=2^b-1$, consider the case mentioned above where we
want hash values in the range $[r]$ where $r=2^\ell$ is a power of
two. We will often refer to the hash values in $[r]$ as buckets so
that we are hashing keys to buckets.
%Avoiding a degenerate case, we assume $r>1$. In particular this implies that $r$ does not divide our prime $p$.

We assume a $k$-universal hash function $h:[u]\to[p]$, e.g.,
the one from \Cref{alg:Mersenne}. To get hash values in $[r]$,
we defined $h^r:[u]\to[r]$ by
\[h^r(x)=h(x)\bmod r=h(x)\andtt (r-1).\]
As discussed previously, the hash values of up to $k$ distinct keys remain
independent with $h^r$. The issue is that hash values from 
$h^r$ are not quite uniform in $[r]$.

Recall that for any key $x$, we have $h(x)$ uniformly distributed in $[2^b-1]$.
This is the uniform distribution on $b$-bit strings except that we are
missing $p=2^b-1$. Now $p$ is the all \texttt{1}s, and 
$p \bmod r = p\andtt (r-1) = r-1$.
Therefore
\begin{align}
   \Pr[h^r(x)=i]
   &=\lceil p/r\rceil/p
   =((p+1)/r)/p
   =(1+1/p)/r
   \quad
   \text{for any $i < r-1$,}
   \label{eq:coll-ell<r-1}
   \\
   \text{while}\quad
   \Pr[h^r(x)=r-1]
   &=\lfloor p/r\rfloor/p=((p+1-r)/r)/p
   =(1-(r-1)/p)/r.
   \label{eq:coll-ell=r-1}
\end{align}
Thus $\Pr[h^r(x)=i]\leq (1+1/p)/r$ for all $i\in[r]$. This upper-bound
only has a relative error of $1/p$ from the uniform $1/r$. For
comparison, if we had used a prime of the form $p=2^b-a$ and $a<r$, then
we would only get an upper bound of $(1+a/p)/r$.
%Below we return to a Mersenne prime $p=2^b-1$

Combining \req{eq:coll-ell<r-1} and \req{eq:coll-ell=r-1} with
pairwise independence, for any distinct keys $x,y\in [u]$, we show that the
collision probability is bounded
\begin{equation}\begin{split}  
   \Pr[h^r(x)=h^r(y)]
      =(r-1)((1+1/p)/r)^2+((1-(r-1)r/p)/r)^2
      %\\&= \frac{r +(r^2-r)/p^2}{r^2}
      =(1+(r-1)/p^2)/r
      %\\& <(1+r/p^2)/r
   .\label{eq:coll}
\end{split}\end{equation}
We note that relative error $r/p^2$ is small as long as $p$ is
large.

\subsubsection{Selecting arbitrary bits from the hash value}
Interestingly, the above analysis holds no matter which $\ell$ bits we use when mapping the hash values from $[2^b-1]$ to $[2^\ell]$.
Let $\mu:[2^b]\to[2^\ell]$ be any map defined by selecting $\ell$ bits from a $b$-bit string.
Above we used $\mu(y)=y\bmod 2^\ell=y\andtt  (2^\ell-1)$, selecting the $\ell$ least significant bits, but we could also use $\mu(y)=\floor{y/2^{b-\ell}}=y\rs (b-\ell)$ selecting the $\ell$ most significant bits.
The basic point is that a uniform distribution on $[2^b]$ maps to a uniform distribution on $[2^\ell]$.
We are only missing the all \texttt1s value $p=2^b-1$ which maps to $2^\ell-1$ regardless of which $\ell$ bits we select, so our equations \req{eq:coll-ell<r-1}--\req{eq:coll} hold no matter which $\ell$ bits we select for $h^r$.

The fact that it doesn't matter which $\ell$ bits we select is only
true because we use a Mersenne prime $p=2^b-1$. Suppose we used some
other $b$-bit prime $p=2^b-a$ where $2^{b-\ell}<a<2^{b-1}$. If we
select the $\ell$ most signifiant bits, then $0$ elements from $[p]$
map to $2^\ell-1$ while $2^{b-\ell}$ elements from $[p]$ map to $0$. However,
with the $\ell$ least significant bits, we have $\floor{p/2^\ell}$ or
$\ceil{p/2^\ell}$ elements from $[p]$ mapping to each element in
$[2^\ell]$, so the maximal difference is 1.


\subsection{Two-for-one hash functions in second moment estimation}
In this section we discuss how we can get several hash functions for
the price of one, and apply the idea to second moment estimation using
count sketches \cite{charikar04count-sketch}.

Suppose we had a $k$-universal hash function into $b$-bit strings.
We note that using standard programming languages such as C, we have
no simple and efficient method computing such hash
functions when $k>2$. However, later we will argue that polynomial
hashing using a Mersenne prime $2^b-1$ delivers a better-than-expected
approximation.

Let $h:U\to [2^b]$ be $k$-universal. By definition this
means that if we have $j\leq k$ distinct keys $x_1,\ldots,x_j$, then
$(h(x_1),\ldots,h(x_j))$ is uniform in $[2^b]^j\equiv [2]^{bj}$,
so this means that \emph{all} the bits in $h(x_1),\ldots,h(x_j)$ are
independent and uniform. We can use this to split our $b$-bit hash
values into smaller segments, and sometimes use them as if
they were the output of universally computed hash functions.
We illustrate this idea below in the context of the second moment estimation.

\subsubsection{Second moment estimation}
We now review the second moment estimation of streams based on count
sketches \cite{charikar04count-sketch} (which are based on the
celebrated second moment AMS-estimator from \cite{alon96frequency})

The basic set-up is as follows.  For keys in $[u]$ and integer values in $\Z$, we are given a stream of key/value $(x_0,\Delta_0),\ldots, (x_{n-1},\Delta_{n-1})\in [u]\times\Z$. The
total value of key $x\in[u]$ is
\[f_x=\sum_{i\in[n],x_i=x} \Delta_i.\]
We let $n\leq u$ be  the number of non-zero values
$f_x\neq 0$, $x\in [u]$. Often $n$ is much smaller than $u$.
We define the $m$th moment to be the $m$-norm to $m$th power,
$F_m^m = \sum_{x\in [u]}f_y^m$. The goal here is to
estimate the second moment $F_2^2 = \sum_{x\in [u]}f_x^2$. 

\begin{algorithm}[H]
   \caption{\label{alg:count-sketch} Count Sketch. Uses a
      vector/array $C$ of $r$ integers and two independent
      4-universal hash functions $i:[u]\to[r]$ and $s:[u]\to\{-1,1\}$.
   .}
   \begin{algorithmic}
      \Procedure{Initialize}{}
         \State For $i\in[t]$, set $C[i]\gets 0$.
      \EndProcedure
      \Procedure{Process}{$x, \Delta$}
         \State $C[i(x)]\gets C[i(x)]+s(x) \Delta$. 
      \EndProcedure
      \Procedure{Output}{}
         \State \Return $\sum_{i\in[t]} C[i]^2$.
      \EndProcedure
   \end{algorithmic}
\end{algorithm}
The standard analysis \cite{charikar04count-sketch} shows that 
\begin{align}
   \E[X]&= \| f\|_2^2 \label{eq:E-F2}\\
   \Var[X]&=2(\| f\|_2^4 - \| f\|_4^4)/r<2\| f\|_2^4/r \label{eq:V-F2}
\end{align}
By Chebyshev's inequality, this implies
\[\Pr[|X-\| f\|_2^2|\geq \eps \| f\|_2^2]\leq \Var[X]/(\eps \| f\|_2^2)^2<
2/(k\eps^2).\]
With $t=8/\eps^2$, the error probability is below 1/4.
To reduce the error probability, we can use the standard trick of making $r$ independent experiments and return the median estimate.
Using Chernoff bounds, the probability that the median deviates by more than $\eps \|f\|_2^2$ is bounded by $\exp(-r/12)$.

\subsubsection{Two-for-one hash functions with \texorpdfstring{$b$}{b}-bit hash values}
As the count sketch is described above,
it uses two independent 4-universal hash functions
$i:[u]\to[r]$ and $s:[u]\to\{-1,1\}$, but 4-universal hash functions
are generally slow to compute, so, aiming to save roughly a factor 2
in speed, a tempting idea is to compute them both using a single hash
function.

The analysis behind \req{eq:E-F2} and \req{eq:V-F2} does not quite
require $i:[u]\to[r]$ and $s:[u]\to\{-1,1\}$ to be independent.
It suffices that the hash values are uniform and that for any
given set of $j\leq 4$ distinct keys $x_1,\ldots,x_j$, the $2j$ hash
values $i(x_1),\ldots,i(x_j),s(x_1),\ldots,s(x_j)$ are independent.
A critical step in the analysis is that if
$A$ depends on $i(x_1),\ldots,i(x_j),s(x_2),\ldots,s(x_j)$, but
not on $s(x_1)$, then
\begin{equation}\label{eq:E-0}
  \E[s(x_0) A] = 0 .
\end{equation}
This follows because $\E[s(x_1)]=0$ by uniformity of $s(x_1)$ and because $s(x_1)$ is independent of $A$.


Assuming that $t=2^\ell$ is a power of two, we can easily construct
$i:[u]\to[t]$ and $s:[u]\to\{-1,1\}$ using a single $4$-universal
hash function $h:[u]\to[2^b]$ where $b>\ell$. Recall that all the bits in
$h(x_1),\ldots,h(x_4)$ are independent. We can therefore use the
$\ell$ least significant bits of $h(x)$ for $i(x)$ and the most
significant bit of $h(x)$ for a bit $a(x)\in[2]$, and finally set
$s(x)=1-2a(x)$. It is then easy to show that if $h$ is $k$-universal
than $h$ satisfies \cref{eq:E-0}.
\begin{algorithm}[H]
   \caption{For key $x\in [u]$, compute $i(x)=i_x\in[2^\ell]$ and
      $s(x)=s_x\in\{-1,+1\}$,\rule{5ex}{0ex}
   using $h:[u]\to [2^b]$ where $b>\ell$.}
   \label{alg:h-and-s}
   \begin{algorithmic}
      \State $h_x\gets h(x)$
      \Comment $h_x$ uses $b$ bits
      \State $i_x\gets h_x \andtt (2^\ell-1)$
      \Comment $i_x$ gets $\ell$ least significant bits of $h_x$
      \State $a_x\gets h_x\rs (b-1)$
      \Comment $a_x$ gets the most significant bit of $h_x$
      \State $s_x\gets 1-(a_x\ls1)$
      \Comment $a_x\in[2]$ is converted to a sign $s_x\in\{-1,+1\}$
   \end{algorithmic}
\end{algorithm}
% \begin{lemma}\label{lem:b-bit-hashing} Suppose $h:[u]\to[2^b]$ is $k$-universal. Let
%    $i:[u]\to[2^\ell]$ and
%    $s:[u]\to\{-1,1\}$ be constructed from $h$ as described in Algorithm \ref{alg:h-and-s}. Then $h$ and $s$ are both $k$-universal. Moreover, for
%    any $j\leq k$ distinct keys $x_1,\ldots,x_j$, the $2j$ hash
%    values $i(x_1),\ldots,i(x_j),s(x_1),\ldots,s(x_j)$ are universal.
%    In particular, if $A$ depends on
%    $i(x_1),\ldots,i(x_j),s(x_2),\ldots,s(x_j)$, but not on $s(x_1)$, then
%    \begin{equation}\label{eq:E-0}
%       \E[s(x_1)A]=0
%    \end{equation}
% \end{lemma}
Note that Algorithm \ref{alg:h-and-s} is well defined as long as 
$h$ returns a $b$-bit integer. However, \cref{eq:E-0} requires
that $h$ is $k$-universal into $[2^b]$, which in particular implies that
the hash values are uniform in $[2^b]$.


\subsubsection{Two-for-one hashing with  Mersenne primes}\label{sec:two-for-one}
Above we discussed how useful it would be with $k$-universal hashing
mapping uniformly into $b$-bit strings. The issue was that the lack of
efficient implementations with standard portable code if
$k>2$. However, when $2^b-1$ is a Mersenne prime $p\geq u$, then we do
have have the efficient computation from Algorithm \ref{alg:Mersenne}
of a $k$-universal hash function $h:[u]\to[2^b-1]$. The hash values
are $b$-bit integers, and they are uniformly distributed, except that
we are missing the all \texttt{1}s value $p=2^b-1$. We want to
understand how this missing value affects us if we try to split the
hash values as in Algorithm \ref{alg:h-and-s}. Thus, we assume a
$k$-universal hash function $h:[u]\to[2^b-1]$ from which we construct
$i:[u]\to[2^\ell]$ and $s:[u]\to\{-1,1\}$ as
described in Algorithm \ref{alg:h-and-s}. As usual, we assume $2^\ell>1$.
Since $i_x$ and $s_x$ are
both obtained by selection of bits from $h_x$, we know from Section
\ref{sec:power-of-two} that each of them have close to uniform
distributions. However, we need a good replacement for \req{eq:E-0}
which besides uniformity, requires $i_x$ and $s_x$ to be independent,
and this is certainly not the case.

Before getting into the analysis, we argue that we really do get two
hash functions for the price of one. The point is that our efficient
computation in Algorithm \ref{alg:Mersenne} requires that we use a
Mersenne prime $2^b-1$ such that $u\leq 2^{b-1}$, and this is even if
our final target is to produce just a single bit for the sign function
$s:[u]\to\{-1,1\}$. We also know that $2^\ell<u$, for otherwise we
get perfect results implementing $i:[u]\to[2^\ell]$ as the identifty
function (perfect because it is collision free).  Thus we can assume
$\ell<b$, hence that $h$ provides enough bits for both $s$ and $i$.


We now consider the effect of the hash values from $h$ being uniform
in $[2^b-1]$ instead of in $[2^b]$. Suppose we want to compute the
expected value of an expression $B$ depending only on the independent
hash values $h(x_1),\ldots,h(x_j)$ of $j\leq k$ distinct keys
$x_1,\ldots,x_j$.

Our generic idea is to play with the distribution of $h(x_1)$ while
leaving the distributions of the other independent hash values
$h(x_2)\ldots,h(x_j)$ unchanged, that is, they remain uniform in
$[2^b-1]$. We will consider having $h(x_1)$ uniformly distributed in
$[2^b]$, denoted $h(x_1) \sim \unif[2^b]$, but then we later have to
subtract the ``fake'' case where $h(x_1)=p=2^b-1$.  Making the
distribution of $h(x_1)$ explicit, we get
\begin{equation}\begin{split}
  \E_{h(x_1) \sim \unif[p]}[B]&=\sum_{y\in[p]}\E[B \mid h(x_1)=y]/p
  =\sum_{y\in[2^b]}\E[B \mid h(x_1)=y]/p - \E[B \mid h(x_1)=p]/p \\
  &=\E_{h(x_1) \sim \unif[2^b]}[B](p+1)/p - \E[B \mid h(x_1)=p]/p.\label{eq:play-with-dist}
\end{split}\end{equation}
Let us now apply this idea our situation where $i:[u]\to[2^\ell]$ and
$s:[u]\to\{-1,1\}$ are constructed from $h$ as described in Algorithm
\ref{alg:h-and-s}. We will prove
\begin{lemma}\label{lem:remove-si}  Consider distinct keys $x_1,\ldots,x_j$, $j\leq k$ and an expression $B=s(x_1)A$ where $A$
   depends on $i(x_1),\ldots,i(x_j)$ and $s(x_2),\ldots,s(x_j)$ but not
   $s(x_1)$. Then
   \begin{equation}\label{eq:remove-si}
      \E[s(x_1)A]=\E[A\mid i(x_1)=2^\ell-1]/p.
   \end{equation}
\end{lemma}
\begin{proof}
When $h(x_1) \sim \unif[2^b]$, then $s(x_1)$ is uniform
in $\{-1,+1\}$ and independent of $i(x_1)$. The remaining
$(i(x_i),s(x_i))$, $i>1$, are independent of $s(x_1)$ because they
are functions of $h(x_i)$ which is independent of $h(x_1)$, so
we conclude that 
\[\E_{h(x) \sim \unif[2^b]}[s(x_1)A]=0\]
Finally, when $h(x_1)=p$, we get $s(x_1)=-1$ and $i(x_1)=2^\ell-1$, 
so applying \req{eq:play-with-dist}, we conclude
that 
\[\E[s(x_1)A] = -\E[s(x_1) A \mid h(x_1) = p]/p = \E[A \mid i(x)=2^\ell-1]/p.\]
\end{proof}
Above \req{eq:remove-si} is our replacement for \req{eq:E-0}, that is,
when the hash values from $h$ are uniform in $[2^b-1]$ instead of
in $[2^b]$, then $\E[s(x_1)B]$ is reduced by $\E[B \mid i(x)=2^\ell-1]/p$.
For large $p$, this is a small additive error. Using this in a careful
analysis, we will show that our fast second moment estimation 
based on Mersenne primes performs almost perfectly:

\begin{theorem}\label{thm:h-and-s-p}
   Let $r>1$ and $u>r$ be powers of two and let $p=2^b-1>u$ be a
   Mersenne prime.
   Suppose with have a 4-universal hash function $h:[u]\to[2^b-1]$, e.g.,
   generated using Algorithm \ref{alg:Mersenne}. Suppose
   $i:[u]\to[r]$ and
   $s:[u]\to\{-1,1\}$ are constructed from $h$ as described in
   Algorithm \ref{alg:h-and-s}. Using this $i$ and $s$ 
   in the Count Sketch Algorithm \ref{alg:count-sketch}, the second moment 
   estimate $X=\sum_{i\in[k]} C_i^2$ satisfies:
   \begin{align}
      \E[X]&=F_2+(F_1^2-F_2)/p^2 < (1+n/p^2)\,F_2\textnormal,\label{eq:E-F2-p}\\
      | \E[X] - F_2 | &\le F_2 (n - 1)/p^2, \label{eq:E-F2-p-com}\\
      \Var[X]&< 2(F_2^2-F_4)/r+F_2^2 (2.33+4 n/r)/p^2<2F_2^2/r.\label{eq:V-F2-p}
   \end{align}
\end{theorem}
The difference from \req{eq:E-F2} and \req{eq:V-F2} 
is negligible when $p$ is large. Theorem \ref{thm:h-and-s-p} will be
proved in Section \ref{sec:analysis-two-for-one}.


\subsection{Arbitrary number of buckets}\label{sec:most-uniform}
We now consider the general case where we want to hash into a set of buckets $R$ whose size is not a power of two.
Suppose we have a $2$-universal hash function $h:U\to Q$.
We will compose $h$ with a map $\mu:Q\to R$, and use $\mu\circ h$ as a hash function from $U$ to $R$.
Let $q=|Q|$ and $r=|R|$.
We want the map $\mu$ to be \emph{most uniform} in the sense that for bucket $i\in R$, the number of elements from $Q$ mapping to $i$ is either $\floor{q/r}$ or $\ceil{q/r}$.
Then the uniformity of hash values with $h$ implies for any key $x$ and bucket $i\in R$ \[\floor{q/r}/q\leq \Pr[\mu(h(x))=i]\leq \ceil{q/r}/q.\]
Below we typically have $Q=[q]$ and $R=[r]$.
A standard example of a most uniform map $\mu:[q]\to[r]$ is $\mu(x)=x\bmod r$ which the one used above when we defined $h^r:[u]\to[r]$, but as we mentioned before, the modulo operation is quite slow unless $r$ is a power of two.

Another example of a most uniform map $\mu:[q]\to[r]$ 
is $\mu(x)=\floor{xr/q}$,
which is also quite slow in general, but if $q=2^b$ is a power of two,
it can be implemented as $\mu(x)=(xr)\rs\,b$ where 
$\rs$ denotes right-shift. This would be yet another advantage 
of of having $k$-universal hashing into $[2^b]$.\todo{Add comment about it being efficient for Mersenne aswell using our result!}

Now, our interest is the case where $q$ is a Mersenne prime $p=2^b-1$. We want
an efficient and most uniform map $\mu:[2^b-1]$ into any given $[r]$.
Our simple solution is to define
\begin{equation}\label{eq:most-uniform}
   \mu(x)=\floor{(x+1)r/2^b}=((x+1)r)\rs b.
\end{equation}
Lemma \ref{lem:most-uniform} (iii) below 
states that \req{eq:most-uniform} indeed
gives a most uniform map. 
\begin{lemma}\label{lem:most-uniform} Let $r$ and $b$ be positive integers.
   %, and let $x\in [2^b-1]$.
   Then
   \begin{itemize}
      \item[(i)] $x\mapsto (xr)\rs\,b$ is a most
         uniform map from $[2^b]$ to $[r]$.
      \item[(ii)] $x\mapsto (xr)\rs\,b$ is a most
         uniform map from $[2^b]\setminus\{0\}=\{1,\ldots,2^b-1\}$ to $[r]$.
      \item[(iii)] $x\mapsto ((x+1)r)\rs \, b$ is a most
         uniform map from $[2^b-1]$ to $[r]$.
   \end{itemize}
\end{lemma}
\begin{proof}
   Trivially (ii) implies (iii). 
   The statement (i) is folklore and easy to prove, so we know that every
   $i\in[r]$ gets hit by $\floor {2^b/r}$ or $\ceil{2^b/r}$ elements from
   $[2^b]$. It is also clear that $\ceil{2^b/r}$ elements, including $0$,
   maps to $0$. To prove (ii), we remove $0$ from $[2^b]$, 
   implying that only
   $\ceil{2^b/r}-1$ elements map to $0$. For all positive integers $q$
   and $r$, $\ceil{(q+1)/r}-1=\floor{q/r}$, and we use this here with 
   $q=2^b-1$. It follows that all buckets from $[r]$ gets $\floor{q/r}$
   or $\floor{q/r}+1$ elements from $Q=\{1,\ldots,q\}$. If $r$ does
   not divide $q$ then $\floor{q/r}+1=\ceil{q/r}$, as desired. However,
   if $r$ divides $q$, then $\floor{q/r}=q/r$, and this
   is the least number of elements from $Q$ hitting any bucket in $[r]$. Then 
   no bucket from $[r]$ can get hit by more than $q/r=\ceil{q/r}$ 
   elements from $Q$. This completes the proof of (ii), and hence of (iii).
\end{proof}
We note that our trick does not work when $q=2^b-a$ for $a\geq 2$, that is,
using $x\mapsto ((x+a)r)\rs  b$, for in this general case, 
the number of elements hashing to $0$ is $\ceil {2^b/r}-a$, or $0$ if
$a\geq \floor {2^b/r}$.
One may try many other hash functions $(c_1 x r+ c_2 x+ c_3 r + c_4) \rs b$ similarly without any luck.
Our new uniform map from \req{eq:most-uniform} is thus very specific to Mersenne prime fields.
For general $a\ge 2$ we provide a scheme using two shifts in Section \ref{sec:division}.

We will see in Section \ref{sec:two-for-one} that our new uniform map
works very well in conjunction with the idea of splitting of hash values.




\subsection{Even faster division and modulo with (Pseudo) Mersenne Primes}\label{subsec:intro-division}
%
\begin{algorithm}[H]
   \caption{For Mersenne prime $p=2^b-1$ and $x< 2^{2b}$, computes
   \label{alg:div-simple}
   $y=x\bmod p$ and $z=\floor{x/p}$}
   \begin{algorithmic}
      \State $z \gets x+1 \rs b$,
      \State $z \gets z+x+1 \rs b$,
      %\State $y \gets x - (z \ls b) + z$.
      \State $y \gets (x + z) \andtt p $.
   \end{algorithmic}
\end{algorithm}
%
We even suggest a speed-up the computation of $\pmod p$ for Mersenne primes
$p=2^b-1$. The issue in Algorithm \ref{alg:Mersenne} is that
if-statement can be slow because of branch prediction.
From the standard Algorithm \ref{alg:Mersenne} this means that we can
replace the last three lines statements
\begin{algorithmic}
   \State $y \gets (y\andtt p)+(y\rs b)$
   \If{$y\ge p$}
      \State $y\gets y-p$
   \EndIf
\end{algorithmic}
with Algorithm \ref{alg:div-simple} above.

\vspace{1em}

In fact our Algorithm \ref{alg:div-simple} does much more than just division and modulo computation with Mersenne primes.
The full algorithm in Section \ref{sec:division} gives a fast, branch-less algorithm for division and modulo by any \emph{generalized} Mersenne Prime.
There are in general two kinds of such primes:
%Primes on the form $2^n-1$ are known as Mersenne primes, and are used all over Cryptography and Computer Science because they allow quicker algorithms than primes in general.
%This allows speeding up finite field arithmetic,
%since $c\bmod (2^n-1) = (c\bmod 2^n) + \lfloor k/n\rfloor$ if the value is smaller than $p=2^n-1$ (otherwise just subtract $p$.)

\begin{description}
   \item[Pseudo-Mersenne Primes]
      are primes on the form $2^b-c$, where is usually required that $c < 2^{\lfloor n/2\rfloor}$~\cite{van2014encyclopedia}.
      Crandal patented a method for working with Pseudo-Mersenne Primes in 1992~\cite{crandall1992method},
      why those primes are also sometimes called ``Crandal-primes''.
      The method was formalized and extended by Jaewook Chung and Anwar Hasan in 2003~\cite{chung2003more}.
      Our method, while simpler, has both stronger guarantees and better practical performance.
      We provide a comparison with the Crandal-Chung-Hansan method in Section 4.
      %also sometimes known as Crandall primes, are numbers on the form $2^n - c$ for a small odd $c$.
   \item[Generalized Mersenne Primes]
      also sometimes known as Solinas primes~\cite{Solinas2011}, are sparse numbers, that is $f(2^m)$ where $f(x)$ is a low-degree polynomial.
      Examples are the primes in NIST's document ``Recommended Elliptic Curves for Federal Government Use'':
         $p_{192} = 2^{192} - 2^{64} - 1$
      and
         $p_{384} = 2^{384}-2^{128}-2^{96}+2^{32}-1$.
      We simply note that Solinas primes are also Pseudo-Mersenne Primes, which allow a particularly fast multiplication with $c$.
      Hence our algorithms are also efficient for Generalised Mersenne primes.
\end{description}

Our generalized algorithm is shown in Algorithm \ref{alg:division-generalized}.
\begin{algorithm}[H]
   \caption{For Pseudo-Mersenne prime $p=2^b-c$ and $x,m$ such that $x< (2^b/c)^m$, computes
      $z=\floor{x/p}$}
   \label{alg:division-generalized}
   \begin{algorithmic}
      %\Procedure{Divide}{x, n, c}
         \State $x' \gets x + c$
         \State $z \gets 0$
         %\For{$i\gets 1$ \textbf{to} $m$}
         \For{ $m$ times}
            \State $z \gets z * c + x'$
            \State $z \gets z \rs n$
         \EndFor
         %\EndFor
         %\State \Return $v$
      %\EndProcedure
   \end{algorithmic}
\end{algorithm}
The full proof of this is given in Theorem \ref{thm:simple-div} in Section \ref{sec:division}.

In the Crandall case $x\le 2^b$ and $c\le 2^{\floor n/2}$, as described by the Encyclopedia of Cryptography and Security~\cite{van2014encyclopedia}, we can let $m=2$, unroll the loop, and the algorithm runs in just two steps for a total of five operations:
   $$
   \left\lfloor\frac{x}{2^b-c}\right\rfloor
   = \left\lfloor\frac{\lfloor\frac{x'}{2^b}\rfloor c + x'}{2^b}\right\rfloor
      = (x' \rs n)*c + x' \rs n
   ,$$
   including $x'=x+c$.

%In fact our algorithm does something stronger, namely gives an efficient division algorithm for any number on the form $q-c$ where $q$ supports fast division.
The intuition for the algorithm is the expansion
\begin{align}
   \frac{x}{q-c}
   = \frac{x}{q}\sum_{i=0}^\infty \left(\frac{c}{q}\right)^i
   = x\frac{1+\frac{c+\frac{c^2 + \dots}{q}}{q}}{q}
   = \frac{\frac{\frac{\dots+x}{q}c+x}{q}c+x}{q},
\end{align}
however curiously we need to add $c$ to $x$ to compensate for the rounding down.




%The proposed algorithm takes integers $n>c>0$ and $x$, and returns
%$\left\lfloor\frac{x}{2^n-c}\right\rfloor$.
%The algorithm performs two additions, a shift, and a multiplication with $c$ for each memory word occupied by $x$.
%Thus, in the case of Generalized Mersenne Primes with sparsity $k$, we can use a total of $3+k$ simple operations.


%Note that for many Pseudo Mersenne Primes, multiplication by $c$ can be replaced by a single shift, making the algorithm completely additions and shifts.

%Note that we use the same number of operations universal of $x$.


To compute modulo we use the fact that if $z=\floor{x/p}$, then
\begin{align}
   x \bmod p
   = x - pz
   = x - (2^b-c)z
   = x - (z\ls b) - c*z,
\end{align}
which is only two additions, a shift, and a multiplication with $c$ on top of the division algorithm.
As $pz = \floor{x/p}p \le x$ there is no danger of overflow.
We can save one operation by noting
that if $x = z (2^b-c) + r$, then
$$x\bmod p = \left(x+c*z \right) \bmod 2^b,$$
which is fast to compute using $(\andtt p)$.
This is the method is shown in Algorithm \ref{alg:mod-generalized} and applied with $c=1$ in Algorithm \ref{alg:div-simple}.

\begin{algorithm}[H]
   \caption{For Pseudo-Mersenne prime $p=2^b-c$ and $z=\floor{x/p}$ computes
      $r=x \bmod p$.}
   \label{alg:mod-generalized}
   \begin{algorithmic}
      \State $r \gets (x + z*c) \andtt (2^b-1)$
   \end{algorithmic}
\end{algorithm}



%In the case $x\le 2^{2b}$ and $c=1$, we get the simplified Algorithm \ref{alg:div-simple} described above: $ \left\lfloor\frac{x}{2^n-1}\right\rfloor = (x+1 \rs n)+x+1 \rs n$.

\paragraph{Application to arbitrary number of buckets}
In Subsection~\ref{sec:most-uniform} we discussed how $\floor{\frac{h(x)r}{2^b-1}}$ provides a most uniform map from $[2^b-1]\to[r]$.
To avoid the division step, we instead considered the map
$\floor{\frac{(h(x)+1)r}{2^b}}$.
However, for primes on the form $2^b-c$, $c>1$ this approach doesn't provide a most-uniform map.
%
Instead we may use Algorithm \ref{alg:division-generalized} to compute
$$\left\lfloor\frac{h(x)r}{2^b-c}\right\rfloor$$
directly, getting a perfect most-uniform map.
%(Another alternative was to pre-compute $q = \lfloor2^b/p\rfloor$ and take
%$\floor{\frac{h(x)rq}{2^b}}$, however that requires larger words to store the product $h(x)rq$.)


\subsubsection{Related Algorithms}

Modulus computation by Generalized Mersenne primes is widely studied in the Cryptography community.
For example, four of the recommended primes in NIST's document ``Recommended Elliptic Curves for Federal Government Use'' are Generalized Mersenne.
Naturally, much work has been done on making computations with those primes fast.
Articles like ``Simple Power Analysis on Fast Modular Reduction with Generalized Mersenne Prime for Elliptic Curve Cryptosystems''~\cite{sakai2006simple}
give very specific algorithms such as Algorithm \ref{alg:solina}, for each of a number of well known such primes.

\begin{algorithm}[H]
   \caption{Fast reduction modulo $p_{192} = 2^{192} - 2^{64} - 1$}
   \label{alg:solina}
   \begin{algorithmic}
      \State \textbf{input} $c \gets (c_5, c_4, c_3, c_2, c_1, c_0)$, where each $c_i$ is a 64-bit word, and $0 \le c < p^2_{192}$.
      \State $s_0 \gets (c_2, c_1, c_0)$
      \State $s_1 \gets (0, c_3, c_3)$
      \State $s_2 \gets (c_4, c_4, 0)$
      \State $s_3 \gets (c_5, c_5, c_5)$
      \State \textbf{return} $s_0 + s_1 + s_2 + s_3 \mod p_{192}$.
   \end{algorithmic}
\end{algorithm}

Division by Mersenne primes is a less common task, but a number of well known division algorithms can be specialized, such as 
 classical trial division, Montgomery's method and Barrett reduction.


%Montgomery method:
%\begin{align}
%   (aR\mod N)(bR\mod N) \mod N = (abR)R \mod N
%\end{align}
%We then need to remove the factor of $R$ by multiplying with its inverse $\mod N$.


The state of the art appears to be the modified Crandall Algorithm by Chung and Hasan~\cite{chung2006low}.
This algorithm, given in Algorithm \ref{alg:cch} modifies Crandall's algorithm~\cite{crandall1992method} from 1992 to compute division as well as modulo for generalized $2^b-c$ Mersenne primes.
\begin{algorithm}[H]
   \caption{Crandall, Chung, Hassan algorithm. For $p=2^b-c$, computes $q, r$ such that $x = qp+r$ and $r<p$.}
   \label{alg:cch}
   \begin{algorithmic}
      %\Procedure{Divide}{x, n, c}
         \State $q_0 \gets x \rs n $
         \State $r_0 \gets x \andtt 2^b-1$
         \State $q \gets q_0, r\gets r_0$
         \State $i \gets 0$
         \While{$q_i>0$}
            \State $t \gets q_i*c$
            \State $q_{i+1} \gets t \rs n$
            \State $r_{i+1} \gets t \andtt {2^b -1}$
            \State $q\gets q+q_{i+1}$
            \State $r\gets r+r_{i+1}$
            \State $i\gets i+1$
         \EndWhile
         \State $t \gets 2^b-c$
         \While{$r\ge t$}
            \State $r\gets r-t$
            \State $q\gets q+1$
         \EndWhile
         \State\textbf{return} $q$
      %\EndProcedure
   \end{algorithmic}
\end{algorithm}
The authors state that for $2n+\ell$ bit input, Algorithm \ref{alg:cch}
requires at most $s$ iterations of the first loop, if $c < 2^{((s-1)n-\ell)/s}$.
This corresponds roughly to the requirement $x < 2^b (2^b/c)^s$, similar to ours.

Unfortunately the algorithm ends up doing double work, by computing the quotient and remainder concurrently.
The algorithm also suffers from the extra while loop for ``cleaning up'' the computations after the main loop.



%No proof of correctness is given.
%The while loop condition is different.
%No guarantees on running time.
%The addition is different from ours.
%Doesn't add 1. Doesn't add $x$.
%Has that extra weird loop for fixing things in the end.
%So it actually has to do the $r$ computation?


Chung and Hasan also has an earlier, simpler algorithm from 2003~\cite{chung2003more},
but it appears to give the wrong result for many simple cases.
This appears to be because of a lack of the ``clean up'' while loop at the end of Algorithm \ref{alg:cch}.

%listed in Algorithm \ref{alg:cch2}.
%\begin{algorithm}\label{alg:cch2}
%   \caption{For $q=2^n-c$, computes $t, r$ such that $x = tq+r$ and $r<q$. (Broken.)}
%   \begin{algorithmic}[1]
%      \State $r \gets x$
%      \State $t \gets 0$
%      \While{$r > q$}
%         \State $A \gets r \rs n$
%         \State $B \gets r \andtt 2^n-1$
%         \State $t \gets t + A$
%         \State $r \gets B + A*c$
%      \EndWhile
%      \State\textbf{return} $(t, r)$
%   \end{algorithmic}
%\end{algorithm}
%Chung and Hasan show that Algorithm \ref{alg:cch2} runs in $O(n)$ time.
%The algorithm however seems to be broken, as one can see from the example
%$n=4, c=2, x=15$.
%In this case we let $r\gets 15$ and we have $r\ge q = 2^4-2=14$,
%but $A = r \ns n = 0$, so the algorithm never makes progress.


%In~\cite{granger2013generalised} the authors defined a different family of Generalised Mersenne numbers and showed various fast multiplication and reduction schemes.



% On the number of Mersenne Primes:
%Unfortunately there are only 45 of them known.
%The most useful one perhaps being.
%Heuristically there are $O(\log x)$ Mersenne primes up to $x$.
%Trivia:
%Euler proved that an even number $n$ is perfect if and only if it is on the form $n=2^{q-1}M_q$, where $M_q=2^q-1$ is prime.
%(Usually we know a number is perfect if its divisors sum to the number itself, e.g. $6=1+2+3$ or $28=1+2+4+7+14$.)


%[[Curve448]] uses the Solinas prime <math>2^{448} - 2^{224} - 1</math>



%! TEX root = ../mersenne.tex


\section{Analysis of second moment estimation using  Mersenne primes}\label{sec:analysis-two-for-one}
In this section, we will prove Theorem \ref{thm:h-and-s-p}.  Thus we
have $r=2^\ell>1$ and $u>r$ both powers of two and $p=2^b-1>u$ a
Mersenne prime.

Now let us set up the relevant variables for the basic count sketch.
For each key $x\in [u]$, we have a value $f_x\in \Z$, and the
goal was to estimate the second moment $F_2 = \sum_{x\in u}f_x^2$.

We had two functions $i:[u]\to[r]$ and $s:[u]\to\{-1,+1\}$. 
For notational convenience, we define $i_x=i(x)$ and $s_x=s(x)$.

For each $i\in [r]$, we have a counter 
$C_i=\sum_{x\in[u]} s_x f_x[i_x=i]$, and we define the 
estimator $X=\sum_{i\in[k]} C_i^2$. We want to study how
well it approximates $F_2$.
We have 
\begin{align*}
X&=\sum_{i\in[r]}\left( \sum_{x\in[u]}s_x f_x[i_x=i]\right)^2\\
&=\sum_{i\in[r]}\sum_{x,y\in[u]}s_x s_y f_x f_y [i_x = i_y = i]\\
&=\sum_{x,y\in[u]}s_x s_y f_x f_y[i_x=i_y]\\
&=\sum_{x\in[u]} f_x^2+\sum_{x,y\in[u],x\neq y} s_x s_y f_x f_y[i_x=i_y] .
\end{align*}
Thus 
\begin{equation}\label{eq:decomp}
   X=F_2 + Y \mbox{ where } Y=\sum_{x,y\in[u],x\neq y} s_x s_y f_x f_y [i_x = i_y] .
\end{equation}
We thus want to provide bounds on the error $Y$.

As discussed in the introduction one of the critical steps in the analysis
of count sketch in the classical case is \cref{eq:E-0}. We formalize this into
the following property.
\begin{property}\label{prop:independence}
    For distinct keys $x_0, \ldots x_{j - 1}$, $j \le k$
    and an expression $A(i_{x_0}, \ldots, i_{x_{j - 1}}, s_{x_1}, \ldots, s_{x_{j - 1}})$,
    which depends on $i_{x_0}, \ldots, i_{x_{j - 1}}$ and $s_{x_1}, \ldots, s_{x_{j - 1}}$
    but not on $s_{x_0}$. Then
    \begin{align}
        \E[s_{x_0} A(i_{x_0}, \ldots, i_{x_{j - 1}}, s_{x_1}, \ldots, s_{x_{j - 1}})] = 0\; .
    \end{align}
\end{property}

The case where we use a Mersenne prime for our hash function we
have that $h$ is uniform in $[2^b - 1]$ and not in $[2^b]$, hence \Cref{prop:independence}
is not satisfied. Instead we have \cref{eq:E} which is almost as good
and will replace \Cref{prop:independence} in the analysis for count sketch.
Again, we have formalized this into an property.
\begin{property}\label{prop:near-independence}
    There exists $t \in [r]$ such that for distinct keys $x_0, \ldots x_{j - 1}$, $j \le k$
    and an expression \\$A(i_{x_0}, \ldots, i_{x_{j - 1}}, s_{x_1}, \ldots, s_{x_{j - 1}})$,
    which depends on $i_{x_0}, \ldots, i_{x_{j - 1}}$ and $s_{x_1}, \ldots, s_{x_{j - 1}}$
    but not on $s_{x_0}$. Then
    \begin{align}\label{eq:near-independence}
        \E[s_{x_0} A(i_{x_0}, \ldots, i_{x_{j - 1}}, s_{x_1}, \ldots, s_{x_{j - 1}})]
            = \frac1p \E[A(i_{x_0}, \ldots, i_{x_{j - 1}}, s_{x_1}, \ldots, s_{x_{j - 1}}) \mid i_{x_0} = t].
    \end{align}
    Furthermore, there exists $\delta$ such that for any key $x$, then
    \begin{align}\label{eq:prob-special-value}
        \Pr[i_x = l] \le (1 + \delta)/r \; .
    \end{align}
\end{property}


Something something collisions.\todo{Write it properly}
\begin{property}\label{prop:collision}
    There exists a value $a$ such that for distinct keys $x \neq y$,
    then
    \begin{align}\label{eq:collision}
        \Pr[i_x = i_y] \le (1 + \eps)/r\; .
    \end{align}
\end{property}

First, as a warm-up for later comparison, we analyse the simple classic
case where \Cref{prop:independence} is satisfied.
\begin{lemma}
    If \Cref{prop:independence} is satisfied for $k = 4$ and \Cref{prop:collision}
    is satisfied with $\eps$, then
    \begin{align}
        \E[X] &= F_2 \\
        \Var[X] &\le 2(1 + \eps)(F_2^2 - F_4)/r \le 2(1 + \eps)F_2^2/r .
    \end{align}
\end{lemma}
\begin{proof}
    We will first show that $\E[Y] = 0$ and hence that $\E[X] = F_2$. By
    \Cref{prop:independence} we have that $\E[s_x s_y f_x f_y [i_x = i_y]] = 0$
    for $x \neq y$ and thus $\E[Y] = \sum_{x,y\in[u],x\neq y} \E[s_x s_y f_x f_y [i_x = i_y]] = 0$.

    Now we want to bound the variance of $X$. We note that since $\E[Y] = 0$ and $X = F_2 + Y$ then
    \begin{align*}
        \Var[X] = \Var[Y] = \E[Y^2]
            = \sum_{\substack{x, y, x', y' \in [u]\\ x \neq y, x' \neq y'}} \E[(s_x s_y f_x f_y [i_x = i_y])(s_{x'} s_{y'} f_{x'} f_{y'} [i_{x'} = i_{y'}])] .
    \end{align*}
    Now we consider one of the terms $\E[(s_x s_y f_x f_y [i_x = i_y])(s_{x'} s_{y'} f_{x'} f_{y'} [i_{x'} = i_{y'}])]$.
    Suppose that one of the keys, say $x$, is unique, i.e. $x \not\in \{y, x', y'\}$.
    Then \Cref{prop:independence} implies that 
    \[
        \E[(s_x s_y f_x f_y [i_x = i_y])(s_{x'} s_{y'} f_{x'} f_{y'} [i_{x'} = i_{y'}])] = 0 .
    \]
    Thus we can now assume that there are no unique keys. Since $x \neq y$ and $x' \neq y'$, we conclude
    that $(x, y) = (x', y')$ or $(x, y) = (y', x')$. Therefore
    \begin{align*}
       \Var[X] &= \sum_{\substack{x, y, x', y' \in [u]\\ x \neq y, x' \neq y'}}
                \E[(s_x s_y f_x f_y [i_x = i_y])(s_{x'} s_{y'} f_{x'} f_{y'} [i_{x'} = i_{y'}])]
            \\&= 2 \sum_{\substack{x, y, x', y' \in [u]\\ x \neq y, (x', y') = (x, y)}}
                \E[(s_x s_y f_x f_y [i_x = i_y])(s_{x'} s_{y'} f_{x'} f_{y'} [i_{x'} = i_{y'}])]
            \\&= 2\sum_{x,y\in[u],x\neq y} \E[(s_x s_y f_x f_y[i_x=i_y])^2]
            \\&= 2\sum_{x,y\in[u],x\neq y} \E[(f_x^2f_y^2[i_x=i_y])]
            \\&\le 2\sum_{x,y\in[u],x\neq y} (f_x^2f_y^2)(1 + \eps)/r
            \\&= 2(1 + \eps) (F_2^2-F_4)/r.
    \end{align*}
    The inequality follows by \Cref{prop:collision}.
\end{proof}
Something something finish this part\todo{Write this properly}

This completes the proof of \req{eq:V-F2}. In the above analysis, we
did not need $s$ and $i$ to be completely independent. All we needed
was that for any $j\leq 4$ distinct keys $x_1,\ldots,x_j$, the hash
values $s(x_1),\ldots,s(x_j)$ and $i(x_1),\ldots,i(x_j)$ are all
independent and uniform in the desired domain. This was why we could
use a single $k$-universal hash function $h:[u]\to[2^b]$ with
$b>\ell$, and use it to construct $s:[u]\to\{-1,+1\}$ and
$i:[u]\to[2^\ell]$ as described in Algorithm \ref{alg:h-and-s}
(c.f. Lemma \ref{lem:b-bit-hashing}).


\begin{lemma}
    If \Cref{prop:near-independence} is satisfied with $t$
    and $\delta$ and \Cref{prop:collision} is satisfied with
    $\eps$, then
    \begin{align}
        \E[X] &= F_2 + (F_1^2 - F_2)/p^2 \\
        | \E[X] - F_2 | &\le F_2 (n - 1)/p^2 \\
        \Var[X] &\le 2F_2^2/r + F_2^2 (2\eps/r + 4(1 + \delta)n / (rp^2) + n^2/p^4 - 2 /(rn))
    \end{align}
\end{lemma}
\begin{proof}
    We first bound $\E[s_x s_y f_x f_y [i_x = i_y]]$ for distinct keys
    $x \neq y$. Using \cref{eq:near-independence} twice we get that
    \begin{equation}\begin{split}\label{eq:twice-split}
        \E[s_x s_y f_x f_y [i_x = i_y]]
            &= \E[s_x f_x f_y [i_x = i_y] \mid i_y = t]/p
            \\&= \E[s_x f_x f_y [i_x = t]]/p
            \\&= \E[f_x f_y [i_x = t] \mid i_x = t]/p
            \\&= f_x f_y / p^2 \; .
    \end{split}\end{equation}
    From this we can calculate $\E[X]$.
    \begin{align*}
        \E[X]
            = F_2 + \sum_{x \neq y} \E[s_x s_y f_x f_y [i_x = i_y]]
            = F_2 + (F_1^2 - F_2)/p^2 .
    \end{align*}
    Now we note that $0 \le F_1^2 \le n F_2$ by Cauchy-Schwartz, hence we get that
    $| \E[X] - F_2 | \le (n - 1)/p^2$.

    Same method is applied to the analysis of the variance, which is
    \[
        \Var[X]
            = \Var[Y]
            \le \E[Y^2]
            = \sum_{x,y,x',y' \in [u], x \neq y, x' \neq y'} \E[(s_x s_y f_x f_y [i_x = i_y]) (s_{x'} s_{y'} f_{x'} f_{y'}[i_{x'} = i_{y'}])]
        \; .
    \] 
    Consider any term in the sum. Suppose some key, say $x$, is unique in the
    sense that $x \not \in \{y,x',y'\}$. Then we can apply \cref{eq:near-independence}.
    Given that $x \neq y$ and $x'\neq y'$, we have either $2$ or $4$ such unique keys.
    If all 4 keys are distinct, as in \cref{eq:twice-split}, we get
    \begin{align*}
        \E[(s_x s_y f_x f_y [i_x = i_y]) &(s_{x'} s_{y'} f_{x'} f_{y'}[i_{x'} = i_{y'}])]
            \\&= \E[(s_x s_y f_x f_y [i_x = i_y])] \E[s_{x'} s_{y'} f_{x'} f_{y'}[i_{x'} = i_{y'}])]
            \\&= (f_x f_y/p^2)(f_{x'} f_{y'}/p^2)
            \\&= f_x f_y f_{x'} f_{y'}/p^4
        \; .
    \end{align*}
    The expected sum over all such terms is thus bounded
    as 
    \begin{equation}\begin{split}
        \sum_{{\rm distinct}\, x,y,x',y'\in[u]}& \E[(s_x s_y f_x f_y [i_x = i_y]) (s_{x'} s_{y'} f_{x'} f_{y'}[i_{x'} = i_{y'}])]
            \\&= \sum_{{\rm distinct}\,x,y,x',y'\in[u]} f_xf_yf_{x'}f_{y'}/p^4
            \\&\le F_1^4 /p^4
            \\&\le F_2^2 n^2/p^4.\label{eq:distinct}
    \end{split}\end{equation}
    Where the last inequality used Cauchy-Schwartz. We also have to consider all the cases with
    two unique keys, e.g., $x$ and $x'$ unique while $y=y'$. Then using \cref{eq:near-independence}
    and \cref{eq:prob-special-value}, we get
    \begin{align*}
        \E[(s_x s_y f_x f_y [i_x = i_y]) &(s_{x'} s_{y'} f_{x'} f_{y'}[i_{x'} = i_{y'}])]
            \\&= f_x f_{x'} f_y^2 \E[s_x s_{x'} [i_x = i_{x'} = i_y]]
            \\&= f_x f_{x'} f_y^2 \E[s_{x'} [t = i_{x'} = i_y]]/p
            \\&= f_x f_{x'} f_y^2 \E[t = i_y]/p^2
            \\&\le f_x f_{x'} f_y^2(1 + \delta)/(rp^2).
    \end{align*}    
    Summing over all terms with $x$ and $x'$ unique while $y=y'$, and
    using Cauchy-Schwartz and $u\leq p$, we get 
    \begin{align*}
        \sum_{{\rm distinct}\,x,x',y} f_xf_{x'}f_y^2 (1 + \delta) /(rp^2) 
            \le F_1^2 F_2 (1 + \delta)/(rp^2)
            \le F_2^2 n(1 + \delta)/(rp^2).
    \end{align*}
    There are four ways we can pick the two unique keys $a\in \{x,y\}$
    and $b\in \{x',y'\}$, so we conclude that
    \begin{equation}\label{eq:one-pair}
        \sum_{\substack{
            x,y,x',y'\in[u], x\neq y, x'\neq y',\\
            (x,y)=(x',y')\,\vee\,(x,y)=(y',x')
        }}
        \E[(s_x s_y f_x f_y [i_x = i_y]) (s_{x'} s_{y'} f_{x'} f_{y'}[i_{x'} = i_{y'}])]
            \le 4 F_2^2 n(1 + \delta)/(rp^2) .
    \end{equation}
    Finally, we need to reconsider the terms with two pairs, that
    is where $(x,y)=(x',y')$ or $(x,y)=(y',x')$. In
    this case, $(s_x s_y f_x f_y [i_x = i_y]) (s_{x'} s_{y'} f_{x'} f_{y'}[i_{x'} = i_{y'}]) = f_x^2 f_y^2 [i_x = i_y]$.
    By \cref{eq:collision}, we get 
    \begin{equation}\begin{split}    
        \sum_{\substack{
            x,y,x',y'\in[u], x\neq y, x'\neq y',\\
            (x,y)=(x',y')\,\vee\,(x,y)=(y',x')
        }}&
            \E[(s_x s_y f_x f_y [i_x = i_y]) (s_{x'} s_{y'} f_{x'} f_{y'}[i_{x'} = i_{y'}])]
            \\&=2\sum_{x,y\in[u],x\neq y} f_x^2f_y^2 \Pr[i_x=i_y]
            \\&=2\sum_{x,y\in[u],x\neq y} f_x^2f_y^2 (1 + \eps)/r
            \\&=2(F_2^2 - F_4)(1 + \eps)/r .\label{eq:two-pairs}
    \end{split}\end{equation}
    Adding up add \req{eq:distinct}, \req{eq:one-pair}, and
    \req{eq:two-pairs}, we get 
    \begin{align*}
        \Var[Y]
            &\le 2(1 + \eps)(F_2^2 - F_4)/r + F_2^2(4(1 + \delta) n / (rp^2) + n^2/p^4)
            \\&\le 2F_2^2/r + F_2^2 (2\eps/r + 4(1 + \delta)n / (rp^2) + n^2/p^4 - 2 /(rn)) .
    \end{align*}
    This finishes the proof.
\end{proof}


We will use that we know that $2 \le r \le u/2 \le (p + 1)/4$ and $n \le u$.
This implies that $p \ge 7$ and that $n/p \le u/p \le 4/7$.

\paragraph{Mersenne to power of two.} We then know that $\delta = -(r - 1)/p \le 0$ 
and $\eps = (r - 1)/p^2 \le r/p^2$. We want to prove that
$2\eps/r + 4(1 + \delta)n / (rp^2) + n^2/p^4 - 2/(rn) \le 0$ which would
prove our result. We get that
\begin{align*}
    2\eps/r + 4(1 + \delta)n / (rp^2) + n^2/p^4 - 2/(rn) 
        &\le 2/p^2 + 4n/(r p^2) + n^2/p^4 - 2/(rn)
        \\&\le 2/p^2 + 4u/(r p^2) + u^2/p^4 - 2/(ru) .
\end{align*}
Now we note that $4u/(r p^2) - 2/(ru) = (2u^2 - p^2)/(u p^2 r) \le 0$
since $u \le (p + 1)/2$ so it maximized when $r = u/2$. We then get
that
\begin{align*}
    2/p^2 + 4u/(r p^2) + u^2/p^4 - 2/(ru)
        \le 2/p^2 + 8/p^2 + u^2 / p^4 - 4/u^2 .
\end{align*}
We now use that $u/p \le (4/7)^2$ and get that
\begin{align*}
    2/p^2 + 8/p^2 + u^2 / p^4 - 4/u^2
        \le (10 + (4/7)^2 - 4 (7/4)^2)/p^2
        \le 0 .
\end{align*}

\subsection{Two-for-one hashing from uniform bits to arbitrary number of buckets}
We have a hash function $h:U\to Q$, but we want hash values in $R$, so
we need a map $\mu:Q\to R$, and then use $\mu\circ h$ as
our hash function from $U$ to $R$. We normally assume that the hash values 
with $h$ are pairwise independent, that is, for any distinct $x$ and $y$,
the hash values $h(x)$ and $h(y)$ are independent, but then 
$\mu(h(x))$ and $\mu(h(y))$ are also independent. This means
that the collision probability can be calculated
as 
\[\Pr[\mu(h(x))=\mu(h(y))]=\sum_{i\in R}\Pr[\mu(h(x))=\mu(h(y))=i]=\sum_{i\in R}\Pr[\mu(h(x)=i)]^2.\]
This sum of squared probabilities attains is minimum value $1/|R|$
exactly when $\mu(h(x))$ is uniform in $R$. 

Let $q=|Q|$ and $r=|R|$. Suppose that $h$ is $2$-universal. Then
$h(x)$ is uniform in $Q$, and then we get the lowest collision
probability with $\mu\circ h$ if $\mu$ is most uniform as defined in
Section \ref{sec:most-uniform}, that is, the number of elements from
$Q$ mapping to any $i\in[r]$ is either $\floor{q/r}$ or
$\ceil{q/r}$. To calculate the collision probability,
Let $a\in[r]$ be such that $r$ divides $q+a$. Then the map $\mu$ maps
$\ceil{q/r}=(q+a)/r$ balls to $r-a$ buckets and
$\floor{q/r}=(q+a-r)/r$ balls to $a$ buckets. For a key $x\in [u]$, we
thus have $r-a$ buckets hit with probability $(1+a/q)/r$ and
$a$ buckets hit with probability $(1-(r-a)/q)/r$.
The collision probability is then
\begin{equation}\begin{split}
    \Pr[\mu(h(x))=&\mu(h(y))]= (r-a)((1+a/q)/r)^2+a((1-(r-a)r/q)/r)^2
        \\&=\frac{(r-a)+(r-a)2a/q+(r-a)a^2/q^2+ a-a^2(r-a)/p+a(r-a)^2/q^2}{r^2}
        \\&=\frac{r +r a (r-a)/q^2}{r^2}
        \\&=(1+a(r-a)/q^2)/r
        \\&\le \left(1+(r/(2q))^2\right)/r.\label{eq:coll-a}
\end{split}\end{equation}
Note that the above calculation generalizes the one for \req{eq:coll} which
had $a=1$. We will think of $(r/(2q))^2$ as the general relative rounding
cost when we do not have any information about how $r$ divides $q$.



%! TEX root = ../mersenne.tex
\section{Algorithms and analysis with arbitrary number of buckets}
We now consider the case where we want to hash into
a number of buckets. We will analyze the collision probability
with most uniform maps introduced in Section \ref{sec:most-uniform},
and later we will show how it can be used in connection with the
two-for-one hashing from Section \ref{sec:two-for-one}.

\subsection{Collision probability with most uniform distributions}
We have a hash function $h:U\fct Q$, but we want hash values in $R$, so
we need a map $\mu:Q\fct R$, and then use $\mu\circ h$ as
our hash function from $U$ to $R$. We normally assume that the hash values 
with $h$ are pairwise independent, that is, for any distinct $x$ and $y$,
the hash values $h(x)$ and $h(y)$ are independent, but then 
$\mu(h(x))$ and $\mu(h(y))$ are also independent. This means
that the collision probability can be calculated
as 
\[\Pr[\mu(h(x))=\mu(h(y))]=\sum_{i\in R}\Pr[\mu(h(x))=\mu(h(y))=i]=\sum_{i\in R}\Pr[\mu(h(x)=i)]^2.\]
This sum of squared probabilities attains is minimum value $1/|R|$
exactly when $\mu(h(x))$ is uniform in $R$. 

Let $q=|Q|$ and $r=|R|$. Suppose that $h$ is $2$-universal. Then
$h(x)$ is uniform in $Q$, and then we get the lowest collision
probability with $\mu\circ h$ if $\mu$ is most uniform as defined in
Section \ref{sec:most-uniform}, that is, the number of elements from
$Q$ mapping to any $i\in[r]$ is either $\floor{q/r}$ or
$\ceil{q/r}$. To calculate the collision probability,
Let $a\in[r]$ be such that $r$ divides $q+a$. Then the map $\mu$ maps
$\ceil{q/r}=(q+a)/r$ balls to $r-a$ buckets and
$\floor{q/r}=(q+a-r)/r$ balls to $a$ buckets. For a key $x\in [u]$, we
thus have $r-a$ buckets hit with probability $(1+a/q)/r$ and
$a$ buckets hit with probability $(1-(r-a)/q)/r$.
The collision probability is then
\begin{align}
\Pr[\mu(h(x))=&\mu(h(y))]= (r-a)((1+a/q)/r)^2+a((1-(r-a)r/q)/r)^2\nonumber\\
  &=\frac{(r-a)+(r-a)2a/q+(r-a)a^2/q^2+ a-a2(r-a)/p+a(r-a)^2/q^2}{r^2}\nonumber\\
  &=\frac{r +r a (r-a)/q^2}{r^2}=(1+a(r-a)/q^2)/r\leq \left(1+(r/(2q))^2\right)/r.\label{eq:coll-a}
  \end{align}
Note that the above calculation generalizes the one for \req{eq:coll} which
had $a=1$. We will think of $(r/(2q))^2$ as the general relative rounding
cost when we do not have any information about how $r$ divides $q$.

\subsection{Two-for-one hashing from uniform bits to arbitrary number of buckets}
We will now briefly discuss how would get the two-for-one hash
functions in count sketches with an arbitrary number $r$ of bins based
on a single $4$-universal hash function $h:[u]\fct [2^b]$.  We want to
construct the two hash functions $s:[u]\fct\{-1,+1\}$ and
$i:[u]\fct[r]$. As usual the results with uniform $b$-bit strings will
set the bar that we later compare with when from $h$ we get hash values that
are only uniform in $[2^b-1]$.

The construction of $s$ and $i$ is presented in 
Algorithm \ref{alg:b-bit-arb-r}.
\begin{algorithm}\label{alg:b-bit-arb-r}
  \caption{For key $x\in [u]$, compute $i(x)=i_x\in[r]$ and
    $s(x)=s_x\in\{-1,+1\}$.\rule{5ex}{0ex}
    Uses 4-universal $h:[u]\fct [2^b]$.}
  $h_x\gets h(x)$\hfill $\rhd\quad h_x$ has $b$ uniform bits\\
  $j_x\gets h_x\&(2^{b-1}-1)$\hfill $\rhd\quad j_x$ gets $b-1$ least 
   significant bits of $h_x$\\
  $i_x\gets (rj_x)\texttt{>>}(b-1)$\hfill $\rhd\quad i_x$ is most uniform
  in $[r]$\\
$a_x\gets h_x\texttt{>>}(b-1)$\hfill $\rhd\quad a_x$ gets the most significant bit of $h_x$\\
$s_x\gets 2b_x-1$\hfill $\rhd\quad s_x$ is uniform in $\{-1,+1\}$ and
independent of $i_x$.\\
\end{algorithm}  
The difference relative to Algorithm \ref{alg:h-and-s} is the computation
of $i_x$ where we now first pick out the $(b-1)$-bit string $j_x$ from
$h_x$, and then apply the most uniform map $(rj_x)\texttt{>>}(b-1)$
to get $i_x$. This does not affect $s_x$ which remains independent
of $i_x$. 

We now have to study the effect on our estimate error
\[Y=X-F_2\sum_{x,y\in[u],x\neq y} s_x f_x[i_x=i_y]s_y f_y.\]
The expectation is exactly as before. The only
change is in the analysis of the variance where
we before had each $i_x$ uniform in $[r]=[2^\ell]$, hence
$\Pr[i_x=i_y]=1/r$. Now have values uniform in $[2^{b-1}]$ mapped
most uniformly to $[r]$ for an arbitrary $r$. Then by \req{eq:coll-a},
we get $\Pr[i_x=i_y]=\left(1+(r/(2q))^2\right)/r$ where $q=2^{b-1}$. 
This increases
the variance bound accordingly from $2F_2^2/r$ to
\begin{equation}\label{eq:Var-b-bit-arb-r}
\Var[X]=\Var[Y]\leq 2F_2^2\left(1+(r/2^b)^2\right)/r.
\end{equation}

\subsection{Two-for-one hashing from Mersenne primes to arbitrary number of buckets}
We will now show how wan get the two-for-one hash functions in count
sketches with an arbitrary number $r$ of bins based on a single
$4$-universal hash function $h:[u]\fct [2^b-1]$.  Again we want to
construct the two hash functions $s:[u]\fct\{-1,+1\}$ and
$i:[u]\fct[r]$.  The construction will be the same as we had in
Algorithm \ref{alg:b-bit-arb-r} when $h$ returned uniform values in
$[2^b]$ with the change that we set $h_x\gets h(x)+1$, so that it
becomes uniform in $[2^b]\setminus\{0\}$. It is also convinient to
swap the sign of the signbit $s_x$ setting $s_x\gets 2a_x+1$ instead
of $s_x\gets 1-2a_x$. The basic reason is that we have swapped the
role of \texttt{0} and \texttt{1} in $a_x$.  The resulting algorithm
is presented as Algorithm \ref{alg:Mersenne-arb-r}.
\begin{algorithm}\label{alg:Mersenne-arb-r}
  \caption{For key $x\in [u]$, compute $i(x)=i_x\in[r]$ and
    $s(x)=s_x\in\{-1,+1\}$.\rule{5ex}{0ex}
    Uses 4-universal $h:[u]\fct [p]$ for Mersenne prime $p=2^b-1\geq u$.}
  $h_x\gets h(x)+1$\hfill $\rhd\quad h_x$ uses $b$ bits uniformly except $h_x\neq 0$\\
  $j_x\gets h_x\&(2^{b-1}-1)$\hfill $\rhd\quad j_x$ gets $b-1$ least 
   significant bits of $h_x$\\
  $i_x\gets (rj_x)\texttt{>>}b-1$\hfill $\rhd\quad i_x$ is quite uniform
  in $[r]$\\
$a_x\gets h_x\texttt{>>}(b-1)$\hfill $\rhd\quad a_x$ gets the most significant bit of $h_x$\\
$s_x\gets 1-2b_x$\hfill $\rhd\quad s_x$ is quite uniform in $\{-1,+1\}$ and
quite independent of $i_x$.\\
\end{algorithm}  
The rest of Algorithm \ref{alg:Mersenne-arb-r} is exactly like 
Algorithm \ref{alg:b-bit-arb-r}, and we will now discuss the new
distributions of the resulting variables. We had
$h_x$ uniform in $[2^b]\setminus\{0\}$, and then we set
$j_x \gets h_x\&(2^{b-1}-1)$. Then $j_x\in[2^{b-1}]$ with 
$\Pr[j_x=0]=1/(2^{b}-1)$ while  $\Pr[j_x=j]=2/(2^{b}-1)$ for all $j>0$.

Next we set $i_x\gets (rj_x)\texttt{>>}b-1$. We know from Lemma
\ref{lem:most-uniform} (i) that this is a most uniform map from
$[2^{b-1}]$ to $[r]$.  It maps a maximal number of elements from
$[2^{b-1}]$ to $0$, including $0$ which had half probability for
$j_x$.
We conclude
\begin{align}
   \Pr[i_x=0] &= (\ceil{2^{b-1}/r}2-1)/(2^{b}-1)
   \label{eq:prix0}
   \\
   \Pr[i_x\neq 0] &\in
   \{\floor{2^{b-1}/r}2/(2^{b}-1), \ceil{2^{b-1}/r}2/(2^{b}-1)\}
   \label{eq:prixneq0}
   .
\end{align}
%all $i\in[r]\setminus\{0\}$ have probability
%while
%$0$ has probability $(\ceil{2^{b-1}/r}2-1)/(2^{b}-1)$. 
We note
that the probability for $0$ is in the middle of the two other
bounds and often this yields a more uniform distribution on $[r]$ than
the most uniform distribution we could get from the
uniform distribution on $[2^{b-1}]$.

With
more careful calculations, we can get some nicer bounds
that we shall later.
\begin{lemma}\label{lem:ix-r-dist} For any distinct $x,y\in [u]$, 
\begin{align}
\Pr[i_x=0]&\le(1+r/2^b)/r\label{eq:ix=0}\\
\Pr[i_x=i_y]&\leq \left(1+(r/2^b)^2\right)/r.\label{eq:ix=iy}
\end{align}
\end{lemma}
\begin{proof}
The proof of \req{eq:ix=0} is a simple calculation.
Using \req{eq:prix0} and the fact $\ceil{2^{b-1}/r}\le(2^{b-1}+r-1)/r$ we have
\begin{align*}
\Pr[i_x=0]&\le (2(2^{b-1}+r-1)/r)-1)/(2^{b}-1)\\
&=((2^b+r-2)/r)/(2^b-1)\\
&=\left(1+(r-1)/(2^b-1)\right)/r\\
&\le\left(1+r/2^b\right)/r.
\end{align*}
The last inequality follows because $r<u<2^b$.

% The proof of \ref{eq:ix=iy} follows from {\color{red}Thomas}.
For \ref{eq:ix=iy},
let $q=2^{b-1}$ and $p=1/(2q-1)$.
We define $a\ge 0$ to be the smallest integer, such that $r\setminus q+a$.
In particular this means 
$\lceil q/r\rceil = (q+a)/r$ and
$\lfloor q/r\rfloor = (q-r+a)/r$.

We bound the sum
$$
   \Pr[i_x=i_y]
   = \sum_{k=0}^{r-1} \Pr[i_x = k]^2
$$
by splitting into three cases:
1) The case $i_x=0$, where $\Pr[i_x=0]=(2\ceil{q/r}-1)p$,
2) the $r-a-1$ indices $j$ where $\Pr[i_x=j]=2\ceil{q/r}p$,
and 3) the $a$ indices $j$ st. $\Pr[i_x=j]=2\floor{q/r}p$.
\begin{align*}
   \Pr[i_x=i_y]
   &=
   (2p\ceil{ q/r}-p)^2 + (r-a-1) (2p \lceil q/r\rceil)^2 + (r-a) (2p \lfloor q/r \rfloor)^2
    \\&= ((4a+1)r+4(q+a)(q-a-1))p^2/r
   \\&\le (1 + (r^2-r)/(2q-1)^2) / r.
\end{align*}
The last inequality comes from maximizing over $a$, which yields $a=(r-1)/2$.

The result now follows from
\begin{align}
   (r^2-r)/(2q-1)^2
   \le
   (r-1/2)^2/(2q-1)^2
   \le
   (r/(2q))^2
\end{align}
which holds exactly when $r\le q$.



\end{proof}
Lemma \ref{lem:ix-r-dist} above is all we need to know about the
marginal distribution of $i_x$. However, we also need a replacement
for Lemma \ref{lem:remove-si} for handling the signbit $s_x$.
\begin{lemma}\label{lem:remove-si-r-dist} Consider distinct keys
$x_1,\ldots,x_j$, $j\leq k$ and an expression $B=s_{x_1}A$ where $A$
depends on $i_{x_1},\ldots,i_{x_j}$ and $s_{x_2},\ldots,s_{x_j}$ but not
$s_{x_1}$. 
Then
\begin{equation}\label{eq:remove-si-r-dist}
\E[s_xA]=\E[A\suchthat i_x=0]/p.
\end{equation}
\end{lemma}
\begin{proof}
The proof follows the same idea as that for Lemma \ref{lem:remove-si}.
First we have
\[\E[B]=\E_{h_{x_1}\gets U([2^b]\setminus\{0\})}[B]=\E_{h_{x_1}\gets U[2^b]}[B]2^b/p-\E_{h_{x_1}=0}[B]/p.\]
With $h_{x_1}\gets U[2^b]$, the bit $a_{x_1}$ is uniform and 
independent of $j_{x_1}$, so $s_{x_1}\in\{-1,+1\}$ is uniform and 
independent of $i_{x_1}$, and therefore 
\[\E_{h_{x_1}\gets U[2^b]}[s_{x_1}A]=0.\]
Moreover, $h_{x_1}=0$ implies $j_x={x_1}$, $i_{x_1}=0$, $a_{x_1}=0$,
and $s_{x_1}=-1$,
so 
\[\E[s_{x_1}A]=-\E_{h_{x_1}=0}[s_{x_1}A]/p=\E_{i_{x_1}=0}[A].\]
\end{proof}
We now consider our $F_2$ extimator
\[X=\sum_{i\in[r]}\left( \sum_{x\in[u]}s_x f_x[i_x=i]\right)^2\!
=F_2+Y\mbox{ where }Y=\sum_{x,y\in[u],x\neq y}
s_x f_x[i_x=i_y]s_y f_y.\]
Using Lemma \ref{lem:remove-si-r-dist} instead of Lemma
\ref{lem:remove-si-r-dist} we get an almost identical calculation 
of the expection $\E[Y]$ as
in Section \ref{sec:two-for-one} and with the same
conclusion that 
\begin{equation}\label{eq:E-Mersenne-arb-r}
\E[Y]=(F_1^2-F_2)/p^2\leq (n-1)F_2/p^2.
\end{equation}
Therfore we still have the same bound \req{eq:E-F2-p} as in Theorem \ref{thm:h-and-s-p}.

Concerning the variance, we have some more changes to the 
calculations in Section \ref{sec:two-for-one}.
We still start with
\[\Var[X]=\Var[Y]\leq \E[Y^2]=
\sum_{x,y,x',y'\in[u],x\neq y,x'\neq y'}\E[(s_x f_x[i_x=i_y]s_y f_y)
  (s_{x'}f_{x'}[i_{x'}=i_{y'}]s_{y'} f_{y'})].\] 
For terms with 4 distinct keys, we still get the same expectation
as in Section \ref{sec:two-for-one}, so their
total contribution to the variance is still the $F_2^2 n^2/p^4$ from
\req{eq:distinct}

However, for the terms with two unique keys and one pair of identical
keys, the factor $\E[i_x=r-1]<1/r$ gets replaced with $\E[i_x=0]$
which by \req{eq:ix=0} in Lemma \ref{lem:ix-r-dist} is bound by
$(1+r/2^b)/r$. As a result, for the total contribution of these terms,
we have to multiply the $4 F_2^2 n/(rp^2)$ from \req{eq:one-pair} by
$(1+r/2^b)$, so they now sum up to
\begin{equation}\label{eq:one-pair'}
4 (1+r/2^b) F_2^2 n/(rp^2)
\end{equation}
Finally, for the terms with two pairs of identical keys, $\Pr[i_x=i_y]$ was bounded
by $(1+1/p)/r$, which is replaced by the bound $\left(1+(r/2^b)^2\right)/r$, so
so 
\begin{align}
2\sum_{x,y\in[u],x\neq y}&(f_x^2f_y^2)\Pr[i_x=i_y]\nonumber\\
&=2\sum_{x,y\in[u],x\neq y}(f_x^2f_y^2)(1+(r/2^b)^2)/k\nonumber\\
&=2(F_2^2-F_4)(1+(r/2^b)^2)/k\nonumber\\
&\leq 2(1-1/n)F_2^2(1+(r/2^b)^2)/k\label{eq:two-pairs'}
\end{align}
Adding it all up, we have proved that 
\[\Var[Y]\leq F_2^2 n^2/p^4+2(1-1/n)F_2^2(1+(r/2^b)^2)/r+4(1+r/2^b)F_2^2 n/(rp^2).\]
We want the argue that we get the same variance bound as we had in 
\req{eq:Var-b-bit-arb-r} with uniform $b$-bit hash values; namely that
\begin{equation}\label{eq:Var-Mersenne-arb-r}
\Var[Y]\leq 2\left(1+(r/2^b)^2\right)F_2^2/r.
\end{equation}
which follows if 
\[ru^3/p^4+4(1+r/2^b)n^2/p^2\leq 2.\]
Recall that $2\leq k<u\leq (p+1)/2$. Since $u$ is a power of two,
$u\geq 4$. We conclude $n/p\leq u/p\leq 4/7$, $r/p\leq 3/7$, and
$p+1\geq 8$. Therefore the left-hand side is bounded by
$(3/7)(4/7)^3+4(1+2/8)(4/7)^2<1.8<2$.  This completes the proof of
\req{eq:Var-Mersenne-arb-r}. The basic conclusion is that both with
$r$ is a power of two and when $r$ is arbitrary, we get the same
variance bounds with Mersenne primes as we did with uniform $b$-bit
strings. The only difference is the small error in expectation from
\req{eq:E-Mersenne-arb-r}. Relative to the correct value $F_2$, the
relative error in the expectation is by a factor of at most
$n/p^2$ which is insignificant when $p$ is large.




% \section{Generalized analysis}

The general analysis use the following strong assumption.
\begin{assumption}\label{ass:independence}
    For distinct keys $x_0, \ldots x_{j - 1}$, $j \le k$
    and an expression $A(i_{x_0}, \ldots, i_{x_{j - 1}}, s_{x_1}, \ldots, s_{x_{j - 1}})$,
    which depends on $i_{x_0}, \ldots, i_{x_{j - 1}}$ and $s_{x_1}, \ldots, s_{x_{j - 1}}$
    but not on $s_{x_0}$. Then
    \begin{align}
        \E[s_{x_0} A(i_{x_0}, \ldots, i_{x_{j - 1}}, s_{x_1}, \ldots, s_{x_{j - 1}})] = 0\; .
    \end{align}
\end{assumption}
We will show our result using the following weaker assumption.
\begin{assumption}\label{ass:near-independence}
    There exists $l \in [r]$ such that for distinct keys $x_0, \ldots x_{j - 1}$, $j \le k$
    and an expression $A(i_{x_0}, \ldots, i_{x_{j - 1}}, s_{x_1}, \ldots, s_{x_{j - 1}})$,
    which depends on $i_{x_0}, \ldots, i_{x_{j - 1}}$ and $s_{x_1}, \ldots, s_{x_{j - 1}}$
    but not on $s_{x_0}$. Then
    \begin{align}\label{eq:near-independence}
        \E[s_{x_0} A(i_{x_0}, \ldots, i_{x_{j - 1}}, s_{x_1}, \ldots, s_{x_{j - 1}})]
            = \E[A(i_{x_0}, \ldots, i_{x_{j - 1}}, s_{x_1}, \ldots, s_{x_{j - 1}}) \mid i_{x_0} = l]/p \; .
    \end{align}
    Furthermore, there exists $\delta$ such that for any key $x$, then
    \begin{align}\label{eq:prob-special-value}
        \Pr[i_x = l] \le (1 + \delta)/r \; .
    \end{align}
\end{assumption}

For both of the analyses we will need the following assumption
\begin{assumption}\label{ass:collision}
    There exists a value $a$ such that for distinct keys $x \neq y$,
    then
    \begin{align}\label{eq:collision}
        \Pr[i_x = i_y] \le (1 + \eps)/r\; .
    \end{align}
\end{assumption}

\paragraph{Uniform bits to arbitrary bins.}
Then \Cref{ass:independence} is satisfied and \Cref{ass:collision} is
satisfied with $\eps = (r/(2q))^2$.

\paragraph{Mersenne to power of two.}
Then \Cref{ass:near-independence} is satisfied with $l = 2^l - 1$ and
$\delta = -(r - 1)/p$ and \Cref{ass:collision} is satisfied
with $\eps = (r - 1)/p^2$.

\paragraph{Mersenne to arbitrary bins.}
Then \Cref{ass:near-independence} is satisfied with $l = 0$ and
$\delta = r/2^b$ and \Cref{ass:collision} is satisfied
with $\eps = (r/2^b)^2$.

\begin{lemma}
    If \Cref{ass:near-independence} is satisfied with $l$
    and $\delta$ and \Cref{ass:collision} is satisfied with
    $\eps$, then
    \begin{align}
        \E[X] &= F_2^2 + (F_1^2 - F_2^2)/p^2 \le (1 + (n - 1)/p^2)F_2^2\\
        \Var[X] &\le 2F_2^4/r + F_2^4 (2\eps/r + 4(1 + \delta)n / (rp^2) + n^2/p^4 - 2 /(rn))
    \end{align}
\end{lemma}
\begin{proof}
    We first bound $\E[s_x s_y f_x f_y [i_x = i_y]]$ for distinct keys
    $x \neq y$. Using \cref{eq:near-independence} twice we get that
    \begin{equation}\begin{split}\label{eq:twice-split}
        \E[s_x s_y f_x f_y [i_x = i_y]]
            &= \E[s_x f_x f_y [i_x = i_y] \mid i_y = l]/p
            \\&= \E[s_x f_x f_y [i_x = l]]/p
            \\&= \E[f_x f_y [i_x = l] \mid i_x = l]/p
            \\&= f_x f_y / p^2 \; .
    \end{split}\end{equation}
    From this we can calculate $\E[X]$.
    \begin{align*}
        \E[X]
            = F_2^2 + \sum_{x \neq y} \E[s_x s_y f_x f_y [i_x = i_y]]
            = F_2^2 + (F_1^2 - F_2^2)/p^2
            \le (1 + (n - 1)/p^2)F_2^2 \; ,
    \end{align*}
    where the last inequality follows from Cauchy-Schwartz. This shows the first result.

    Same method is applied to the analysis of the variance, which is
    \[
        \Var[X]
            = \Var[Y]
            \le \E[Y^2]
            = \sum_{x,y,x',y' \in [u], x \neq y, x' \neq y'} \E[(s_x s_y f_x f_y [i_x = i_y]) (s_{x'} s_{y'} f_{x'} f_{y'}[i_{x'} = i_{y'}])]
        \; .
    \] 
    Consider any term in the sum. Suppose some key, say $x$, is unique in the
    sense that $x \not \in \{y,x',y'\}$. Then we can apply \cref{eq:near-independence}.
    Given that $x \neq y$ and $x'\neq y'$, we have either $2$ or $4$ such unique keys.
    If all 4 keys are distinct, as in \cref{eq:twice-split}, we get
    \begin{align*}
        \E[(s_x s_y f_x f_y [i_x = i_y]) &(s_{x'} s_{y'} f_{x'} f_{y'}[i_{x'} = i_{y'}])]
            \\&= \E[(s_x s_y f_x f_y [i_x = i_y])] \E[s_{x'} s_{y'} f_{x'} f_{y'}[i_{x'} = i_{y'}])]
            \\&= (f_x f_y/p^2)(f_{x'} f_{y'}/p^2)
            \\&= f_x f_y f_{x'} f_{y'}/p^4
        \; .
    \end{align*}
    The expected sum over all such terms is thus bounded
    as 
    \begin{equation}\begin{split}
        \sum_{{\rm distinct}\, x,y,x',y'\in[u]}& \E[(s_x s_y f_x f_y [i_x = i_y]) (s_{x'} s_{y'} f_{x'} f_{y'}[i_{x'} = i_{y'}])]
            \\&= \sum_{{\rm distinct}\,x,y,x',y'\in[u]} f_xf_yf_{x'}f_{y'}/p^4
            \\&\le F_1^4 /p^4
            \\&\le F_2^2 n^2/p^4.\label{eq:distinct}
    \end{split}\end{equation}
    Where the last inequality used Cauchy-Schwartz. We also have to consider all the cases with
    two unique keys, e.g., $x$ and $x'$ unique while $y=y'$. Then using \cref{eq:near-independence}
    and \cref{eq:prob-special-value}, we get
    \begin{align*}
        \E[(s_x s_y f_x f_y [i_x = i_y]) &(s_{x'} s_{y'} f_{x'} f_{y'}[i_{x'} = i_{y'}])]
            \\&= f_x f_{x'} f_y^2 \E[s_x s_{x'} [i_x = i_{x'} = i_y]]
            \\&= f_x f_{x'} f_y^2 \E[s_{x'} [l = i_{x'} = i_y]]/p
            \\&= f_x f_{x'} f_y^2 \E[l = i_y]/p^2
            \\&\le f_x f_{x'} f_y^2(1 + \delta)/(rp^2).
    \end{align*}    
    Summing over all terms with $x$ and $x'$ unique while $y=y'$, and
    using Cauchy-Schwartz and $u\leq p$, we get 
    \begin{align*}
        \sum_{{\rm distinct}\,x,x',y} f_xf_{x'}f_y^2 (1 + \delta) /(rp^2) 
            \le F_1^2F_2^2 (1 + \delta)/(rp^2)
            \le F_2^4 n(1 + \delta)/(rp^2).
    \end{align*}
    There are four ways we can pick the two unique keys $a\in \{x,y\}$
    and $b\in \{x',y'\}$, so we conclude that
    \begin{equation}\label{eq:one-pair}
        \sum_{\begin{array}{c}
            x,y,x',y'\in[u], x\neq y, x'\neq y',\\
            {\rm two\ keys\ are\ unique}
        \end{array}}
        \E[(s_x s_y f_x f_y [i_x = i_y]) (s_{x'} s_{y'} f_{x'} f_{y'}[i_{x'} = i_{y'}])]
            \le 4 F_2^4 n(1 + \delta)/(rp^2) .
    \end{equation}
    Finally, we need to reconsider the terms with two pairs, that
    is where $(x,y)=(x',y')$ or $(x,y)=(y',x')$. In
    this case, $(s_x s_y f_x f_y [i_x = i_y]) (s_{x'} s_{y'} f_{x'} f_{y'}[i_{x'} = i_{y'}]) = f_x^2 f_y^2 [i_x = i_y]$.
    By \cref{eq:collision}, we get 
    \begin{equation}\begin{split}    
        \sum_{\begin{array}{c}
            x,y,x',y'\in[u], x\neq y, x'\neq y',\\
            (x,y)=(x',y')\,\vee\,(x,y)=(y',x')
        \end{array}}&
            \E[(s_x s_y f_x f_y [i_x = i_y]) (s_{x'} s_{y'} f_{x'} f_{y'}[i_{x'} = i_{y'}])]
            \\&=2\sum_{x,y\in[u],x\neq y} f_x^2f_y^2 \Pr[i_x=i_y]
            \\&=2\sum_{x,y\in[u],x\neq y} f_x^2f_y^2 (1 + \eps)/r
            \\&=2(F_2^4 - F_4^4)(1 + \eps)/r .\label{eq:two-pairs}
    \end{split}\end{equation}
    Adding up add \req{eq:distinct}, \req{eq:one-pair}, and
    \req{eq:two-pairs}, we get 
    \begin{align*}
        \Var[Y]
            &\le 2(1 + \eps)(F_2^4 - F_4^4)/r + F_2^4(4(1 + \delta) n / (rp^2) + n^2/p^4)
            \\&\le 2F_2^4/r + F_2^4 (2\eps/r + 4(1 + \delta)n / (rp^2) + n^2/p^4 - 2 /(rn)) .
    \end{align*}
    This finishes the proof.
\end{proof}

\begin{corollary}
    We get the following results
    \begin{itemize}
        \item \textbf{Mersenne to power of two:}
            \[
                \Var[X] \le 2 F_2^4/r .
            \]
        \item \textbf{Mersenne to arbitrary number of bins:}
            \[
                \Var[X] \le 2 (1 + (r/(p + 1))^2) F_2^4/r .
            \]
    \end{itemize}
\end{corollary}
\begin{proof}
    We will use that we know that $2 \le r \le u/2 \le (p + 1)/4$ and $n \le u$.
    This implies that $p \ge 7$ and that $n/p \le u/p \le 4/7$.

    \paragraph{Mersenne to power of two.} We then know that $\delta = -(r - 1)/p \le 0$ 
    and $\eps = (r - 1)/p^2 \le r/p^2$. We want to prove that
    $2\eps/r + 4(1 + \delta)n / (rp^2) + n^2/p^4 - 2/(rn) \le 0$ which would
    prove our result. We get that
    \begin{align*}
        2\eps/r + 4(1 + \delta)n / (rp^2) + n^2/p^4 - 2/(rn) 
            &\le 2/p^2 + 4n/(r p^2) + n^2/p^4 - 2/(rn)
            \\&\le 2/p^2 + 4u/(r p^2) + u^2/p^4 - 2/(ru) .
    \end{align*}
    Now we note that $4u/(r p^2) - 2/(ru) = (2u^2 - p^2)/(u p^2 r) \le 0$
    since $u \le (p + 1)/2$ so it maximized when $r = u/2$. We then get
    that
    \begin{align*}
        2/p^2 + 4u/(r p^2) + u^2/p^4 - 2/(ru)
            \le 2/p^2 + 8/p^2 + u^2 / p^4 - 4/u^2 .
    \end{align*}
    We now use that $u/p \le (4/7)^2$ and get that
    \begin{align*}
        2/p^2 + 8/p^2 + u^2 / p^4 - 4/u^2
            \le (10 + (4/7)^2 - 4 (7/4)^2)/p^2
            \le 0 .
    \end{align*}
    This finishes the first part.
    
    \paragraph{Mersenne to arbitrary number of bins.} We then know that
    $\delta = r/(p + 1) \le r/p$ and $\eps = r^2/(p + 1)^2$.
    We have that
    \begin{align*}
        \Var[X]
            &= 2F_2^4/r + F_2^4 (2\eps/r + 4(1 + \delta)n / (rp^2) + n^2/p^4 - 2 /(rn))
            \\&= 2(1 + \eps)F_2^4/r + F_2^4(4(1 + \delta)n / (rp^2) + n^2/p^4 - 2 / (rn))
            \\&\le 2(1 + r^2/(p + 1)^2) F_2^4/r + F_2^4(4(1 + r/p)n / (rp^2) + n^2/p^4 - 2 / (rn))
        .
    \end{align*}
    If we can prove that $4(1 + r/p)n / (rp^2) + n^2/p^4 - 2 / (rn) \le 0$ then
    we have the result. We have that
    \begin{align*}
        4(1 + r/p)n / (rp^2) + n^2/p^4 - 2 / (rn)
            &= 4 n/(rp^2) + 4 n /(p^3) + n^2/p^4 - 2/(rn)
            \\&\le 4u/(rp^2) + 4u/(p^3) + u^2/p^4 - 2/(ru) .
    \end{align*}
    Again we note that $4u/(r p^2) - 2/(ru) = (2u^2 - p^2)/(u p^2 r) \le 0$
    since $u \le (p + 1)/2$ so it maximized when $r = u/2$. We then get
    that
    \begin{align*}
       4u/(r p^2) + 4u/(p^3) + u^2/p^4 - 2/(ru)
            \le 8/p^2 + 4u/(p^3) + u^2/p^4 - 4/u^2 .
    \end{align*}
    We now use that $u/p \le (4/7)^2$ and get that
    \begin{align*}
        8/p^2 + 4u/(p^3) + u^2/p^4 - 4/u^2 .
            \le (8 + 4 (4/7) + (4/7)^2 - 4 (7/4)^2)/p^2
            \le 0 .
    \end{align*}
    This finishes the second part.
\end{proof}


%! TEX root = ../mersenne.tex

\section{Division and Modulo with Generalized Mersenne Primes}
\label{sec:division}

The purpose of this section is to prove the correctness of Algorithm \ref{alg:division-generalized}.
In particular we will prove the following equivalent mathematical statement:

\begin{theorem}\label{thm:simple-div}
   Given integers $q>c>0$, $n\ge 0$ and
   %\todo{Do we support negative $c$? Can we instead say $q>|c|>0$?}
   $$0\le x \le \begin{cases}
      c (q/c)^{n} - c &\quad\text{if } c \setminus q \\
      (q/c)^{n-1}(q-c) &\quad\text{otherwise}
      %q^{n} - 1 &\quad\text{if } c = 1 \\
   \end{cases}.$$
   Define the sequence $(v_i)_{i\in[n]}$ by
   $
      v_0 = 0$ and
      $v_{i+1} = \left\lfloor\frac{(v_i+1)c+x}{q}\right\rfloor$.
   Then
   $$
      \left\lfloor\frac{x}{q-c}\right\rfloor = v_n.$$
   %\todo{Given the Corollary, we can also write this as $x < (q/c)^n$, which may be easier to understand?}
\end{theorem}
We note that when $c<q-1$ a sufficient requirement is that $x< (q/c)^n$.
For $c=q-1$ we are computing $\floor{x/1}$ so we do not need to run the algorithm at all.

To be more specific, the error $E_i = \floor{\frac{x}{q-c}} - v_i$ at each step
%The value at each step satisfy $v_i = \floor{\frac{x}{q-c}} - E_i$ where
%the errors $E_i$ are never negative.
is bounded by $0\le E_i\le u_{n-i}$,
where $u_i$ is a sequence defined by
$u_0=0$ and $u_{i+1} = \lfloor\frac{q}{c}u_i+1\rfloor$.
For example, this means that if we stop the algorithm after $n-1$ steps, the error will be at most $u_1=1$.
\begin{proof}
   Write $x = m(q-c)+h$ for non-negative integers $m$ and $h$ with $h<q-c$.
   Then we get
   \begin{align*}
      \left\lfloor\frac{x}{q-c}\right\rfloor = m.
      \label{eq:floor}
   \end{align*}

   Let $u_0=0$, $u_{i+1} = \lfloor\frac{q}{c}u_i+1\rfloor$.
   By induction $u_i \ge (q/c)^{i-1}$ for $i>0$.
   This is trivial for $i=1$ and $u_{i+1}=\lfloor \frac qc u_i +1\rfloor \ge \lfloor (q/c)^i + 1 \rfloor \ge (q/c)^i$.

   Now define $E_i\in\mathbb Z$ such that $v_i = m - E_i$.
   We will show by induction that $0\le E_{i} \le u_{n-i}$ for $0\le i\le n$ such that $E_n = 0$, which gives the theorem.
   For a start $E_0=m\ge 0$ and $E_0 = \lfloor x/(q-c)\rfloor \le (q/c)^{n-1} \le u_n$.

   For $c\setminus q$ we can be slightly more specific, and support $x \le c (q/c)^n-c$.
   This follows by noting that $u_i = \frac{(q/c)^i-1}{q/c-1}$ for $i>0$, since all the $q/c$ terms are integral.
   Thus for $E_0=\floor{x/(q-c)}\le u_n$ it suffices to require $x\le c q^n-c$.

   For the induction step we plug in our expressions for $x$ and $v_i$:
   \begin{align*}
      v_{i+1}
      &= \left\lfloor \frac{(m-E_i+1)c+m(q-c)+h}{q}\right\rfloor
    \\&=
    m
    +
    \left\lfloor \frac{(- E_i+1)c +h}{q}\right\rfloor
    \\&=
    m
    - \left\lceil \frac{(E_i-1)c - h}{q}\right\rceil.
   \end{align*}
   The lower bound follows easily from $E_i \ge 0$ and $h\le q-c-1$:
   $$E_{i+1} = \left\lceil \frac{E_ic - h - c}{q}\right\rceil \ge
   \left\lceil \frac{- q + 1}{q}\right\rceil = 0.$$
   For the upper bound we use the inductive hypothesis as well as the bound $h\ge 0$:
   \begin{align*}
      E_{i+1}
      &= \left\lceil \frac{(E_i-1)c - h}{q}\right\rceil
    \\&\le\left\lceil (u_{n-i}-1)\frac{c}{q}\right\rceil
    \\&= \left\lceil \left\lfloor \frac{q}{c}u_{n-i-1} \right\rfloor \frac{c}{q}\right\rceil
    \\&\le \left\lceil u_{n-i-1}\right\rceil
    \\&= u_{n-i-1}.
   \end{align*}
   The last equality comes from $u_{n-i-1}$ being integer.
   Having thus bounded the errors, the proof is complete.
\end{proof}
We can also note that if the algorithm is repeated more than $n$ times, the error stays at 0, since 
$\lceil (u_{n-i}-1)\frac{c}{q}\rceil = \lceil -\frac{c}{q}\rceil = 0$.
   %In particular, for $c=1$ and $x < q^2$ we have
   %$
   %\left\lfloor\frac{x}{q-1}\right\rfloor
   %= v_2
   %= \left\lfloor\frac{\left\lfloor\frac{x+1}{q}\right\rfloor+x+1}{q}\right\rfloor
   %$.


\bibliographystyle{alpha}
\bibliography{general}

\end{document}




