%! TEX root = ../../mersenne.tex

\subsection{Algorithm}

The proposed algorithm takes integers $n>c>0$ and $x$, and returns
$\left\lfloor\frac{x}{2^n-c}\right\rfloor$.
The algorithm performs two additions, a shift, and a multiplication with $c$ for each memory word occupied by $x$.
Thus, in the case of Generalized Mersenne Primes with sparsity $k$, we can use a total of $3+k$ simple operations.
\begin{algorithm}
   %\caption{}\label{euclid}
   \begin{algorithmic}[1]
      \Procedure{Divide}{x, n, c}
         \State \textbf{let} $m$ \text{such that} $x \le (2^n/c)^m$.
         \State $v \gets 0$.
         %\For{$i\gets 1$ \textbf{to} $m$}
         \For{ $m$ times}
            \State $v \gets (v + 1)c+x$ \textbf{shift-right} $n$.
         \EndFor
         %\EndFor
         \State \Return $v$
      \EndProcedure
   \end{algorithmic}
\end{algorithm}

%Note that for many Pseudo Mersenne Primes, multiplication by $c$ can be replaced by a single shift, making the algorithm completely additions and shifts.

%Note that we use the same number of operations independent of $x$.


To compute modulo we use the fact that
\begin{align}
   x \bmod p
   = x - p\left\lfloor\frac{x}{p}\right\rfloor
   = x - (2^n - c)\left\lfloor\frac{x}{2^n-c}\right\rfloor,
\end{align}
which is only two additions, a shift, and a multiplication with $c$ on top of the division algorithm.
As $\left\lfloor\frac{x}{p}\right\rfloor p \le x$ there is no danger of overflow.

In fact our algorithm does something stronger, namely gives an efficient division algorithm for any number on the form $q-c$ where $q$ supports fast division.
The intuition for the algorithm is the expansion
\begin{align}
   \frac{x}{q-c}
   = \frac{x}{q}\sum_{i=0}^\infty \left(\frac{c}{q}\right)^i
   = x\frac{1+\frac{c+\frac{c^2 + \dots}{q}}{q}}{q}
   = \frac{x+c\frac{x+c\frac{x + \dots}{q}}{q}}{q},
\end{align}
however curiously we need to add a $+1$ to get efficient convergence.

\begin{theorem}
   %Given integers $q>c>0$, $n\ge 0$ and
   \todo{Do we support negative $c$? Can we instead say $q>|c|>0$?}
   $0\le x \le (q/c)^{n-1}(q-c)$,
   \todo{Given the Corollary, we can also write this as $x < (q/c)^n$, which may be easier to understand?}
   then
   \begin{align}
      \left\lfloor\frac{x}{q-c}\right\rfloor = v_n,
   \end{align}
   where $v_0 = 0$ and
   $v_{i+1} = \left\lfloor\frac{(v_i+1)c+x}{q}\right\rfloor$.
\end{theorem}
\begin{proof}
   Write $x = m(q-c)+h$ for non-negative integers $m$ and $h$ with $h<q-c$.
   Then we get
   \begin{align}
      \left\lfloor\frac{x}{q-c}\right\rfloor = m.
      \label{eq:floor}
   \end{align}

   Let $u_0=0$, $u_{i+1} = \lfloor\frac{q}{c}u_i+1\rfloor$.
   By induction $u_i \ge (q/c)^{i-1}$ for $i>0$.
   This is trivial for $i=1$ and $u_{i+1}=\lfloor \frac qc u_i +1\rfloor \ge \lfloor (q/c)^i + 1 \rfloor \ge (q/c)^i$.

   Now define $E_i\in\mathbb Z$ such that $v_i = m - E_i$.
   We will show by induction that $0\le E_{i} \le u_{n-i}$ for $0\le i\le n$ such that $E_n = 0$, which gives the theorem.
   For a start $E_0=m\ge 0$ and $E_0 = \lfloor x/(q-c)\rfloor \le (q/c)^{n-1} \le u_n$.

   For the induction step we plug in our expressions for $x$ and $v_i$:
   \begin{align}
      v_{i+1}
      &= \left\lfloor \frac{(m-E_i+1)c+m(q-c)+h}{q}\right\rfloor
    =
    m
    +
    \left\lfloor \frac{(- E_i+1)c +h}{q}\right\rfloor
    =
    m
    - \left\lceil \frac{(E_i-1)c - h}{q}\right\rceil.
   \end{align}
   Now by the induction hypothesis,
   \begin{align}
      E_{i+1}
      = \left\lceil \frac{(E_i-1)c - h}{q}\right\rceil
      \le\left\lceil (u_{n-i}-1)\frac{c}{q}\right\rceil
      = \left\lceil \left\lfloor \frac{q}{c}u_{n-i-1} \right\rfloor \frac{c}{q}\right\rceil
      \le \left\lceil u_{n-i-1}\right\rceil
      = u_{n-i-1}.
   \end{align}
   and similarly since $E_i \ge 0$ and $h\le q-c-1$,
   we have
   $\left\lceil \frac{E_ic - h - c}{q}\right\rceil \ge
   \left\lceil \frac{- q + 1}{q}\right\rceil = 0$, which completes the proof.
\end{proof}
\begin{corollary}
   For $c=1$ we can be slightly more specific, and support $x < q^n$.
   In particular, for $c=1$ and $x < q^2$ we have
   $
   \left\lfloor\frac{x}{q-1}\right\rfloor
   = v_2
   = \left\lfloor\frac{\left\lfloor\frac{x+1}{q}\right\rfloor+x+1}{q}\right\rfloor
   $.
\end{corollary}
\begin{proof}
   This follows by repeating the proof, but noting that $u_i = \frac{q^i-1}{q-1}$ since all the $q/c$ terms are integral.
\end{proof}
