%! TEX root = ../../mersenne.tex

\subsection{Introduction}

Primes on the form $2^n-1$ are known as Mersenne primes, and are used all over Cryptography and Computer Science because they allow quicker algorithms than primes in general.
This allows speeding up finite field arithmetic,
since $c\bmod (2^n-1) = (c\bmod 2^n) + \lfloor k/n\rfloor$ if the value is smaller than $p=2^n-1$ (otherwise just subtract $p$.)

In this paper we investigate division by such primes, as well as two generalized classes:
\begin{description}
   \item[Generalized Mersenne Primes]
      also sometimes known as Solinas primes~\cite{Solinas2011}, are sparse numbers, that is $f(2^m)$ where $f(x)$ is a low-degree polynomial.
      Examples are the primes in NIST's document "Recommended Elliptic Curves for Federal Government Use": $p_{192} = 2^{192} - 2^{64} - 1$ and $p_{384} = 2^{384}-2^{128}-2^{96}+2^{32}-1$.
   \item[Pseudo-Mersenne Primes]
      also sometimes known as Crandall primes~\cite{crandall1992method}, are numbers on the form $2^n - c$ for a small odd $c$.
      It is sometimes required that $c < 2^{\lfloor n/2\rfloor}$~\cite{van2014encyclopedia}.
\end{description}

We also give a simple algorithm for taking the modulo, which works efficiently for all Pseudo-Mersenne Primes.

