%! TEX root = ../../mersenne.tex
\subsection{Mess}

Words:
 GM, PM reduction: Module af Generalized or Pseudo-Mersennes.

The most important case is ``Modular Multiplication'', in which we are given two $w$-word numbers, $x,y$ and want to compute $xy\mod p$ which is again a $w$-word number.

Montgomery method:
\begin{align}
   (aR\mod N)(bR\mod N) \mod N = (abR)R \mod N
\end{align}
We then need to remove the factor of $R$ by multiplying with its inverse $\mod N$.



Unfortunately there are only 45 of them known.
The most useful one perhaps being.

Heuristically there are $O(\log x)$ Mersenne primes up to $x$.


Trivia:
Euler proved that an even number $n$ is perfect if and only if it is on the form $n=2^{q-1}M_q$, where $M_q=2^q-1$ is prime.
(Usually we know a number is perfect if its divisors sum to the number itself, e.g. $6=1+2+3$ or $28=1+2+4+7+14$.)

Four of the recommended primes in NIST's document "Recommended Elliptic Curves for Federal Government Use" are Solinas primes:


%[[Curve448]] uses the Solinas prime <math>2^{448} - 2^{224} - 1</math>

The classical Mersenne mod algorithm.
Write $n = a2^p + b$, then
\begin{align}
   n \mod M_p = \begin{cases}
      a + b & \text{if } a+b<M_p\\
      a + b + 1 - 2^p & \text{otherwise}.
   \end{cases}
\end{align}

Thus reduction modulo a Mersenne prime requires an
integer addition, as opposed to an integer division for
modular reduction in the general case. There are two
drawbacks to this method of modular reduction:
\begin{itemize}
   \item Finding the integers a and b is easiest when p is a multiple of word size of the machine, since then there is
no actual shifting of bits needed to align a and b for
the modular addition. But word sizes are in practice
powers of two, whereas p must be an odd prime.
   \item The Mersenne primes are so rare, with none between
      $M_{127}$ and $M_{521}$ hat usually there will be none of the desired magnitude.
\end{itemize}
For these reasons~\cite{van2014encyclopedia}, cryptographers tend not to use Mersenne primes, preferring similar moduli such as pseudo-Mersenne primes and generalized Mersenne primes.

\paragraph{Pseudo-Mersenne primes}
On the form $p = 2^n-k$.
If $x < p^2$, write $x = a 2^{2n} + b 2^m + c$,
where $a\in\{0,1\}$.
(Note, if $k > 0$, we always have $a=0$.)
Then $n \mod p = ak^2+bk+c$.
Repeating this a few times gives the calculation.


\paragraph{Generalised Mersenne primes}
Solinas algorithm
