%! TEX root = ../mersenne.tex
\section{Introduction}

\begin{figure}
   \centering
   \begin{tikzpicture}[darkstyle/.style={circle,draw,fill=gray!40,minimum size=20}]
      \newcommand*{\figb}{8}
      % The red box
      \draw[pattern=north west lines, pattern color=red] (0,0) rectangle (\figb,1);
      % Horizontal lines
      \foreach \y in {0,...,6}
         \draw (0, \y) -- (\figb, \y);
      % Vertical lines
      \foreach \x in {0,...,\figb}
         \draw (\x, 0) -- (\x, 6);
      % Most of the numbers
      \pgfmathsetmacro{\figbthree}{\figb - 3}
      \pgfmathsetmacro{\figbtwo}{\figb - 2}
      \pgfmathsetmacro{\figbone}{\figb - 1}
      \foreach \y in {0,...,2}
         \foreach \x in {0,...,\figbthree}
            \node [draw=none] at (.5+\x,.5+\y) {1};
      \foreach \x in {\figbthree,...,\figbone}
         \node [draw=none] at (.5+\x,.5+3) {.};
      \foreach \y in {4,...,5}
         \foreach \x in {0,...,\figbtwo}
            \node [draw=none] at (.5+\x,.5+\y) {0};
      % The rest of the numbers
      \node [draw=none] at (.5+\figb-1,.5+0) {1};
      \node [draw=none] at (.5+\figb-2,.5+0) {1};
      \node [draw=none] at (.5+\figb-1,.5+1) {0};
      \node [draw=none] at (.5+\figb-2,.5+1) {1};
      \node [draw=none] at (.5+\figb-1,.5+2) {1};
      \node [draw=none] at (.5+\figb-2,.5+2) {0};
      \node [draw=none] at (.5+\figb-1,.5+4) {1};
      \node [draw=none] at (.5+\figb-1,.5+5) {0};
   \end{tikzpicture}
   \caption{The output of a random polynomial modulo $p=2^b-1$ is uniformly distributed in $[p]$. This means that each bit has the same identical distribution, which is only $1/p$ biased towards 0.}
   \label{fig:bits}
\end{figure}

The classic way to implement $k$-universal hashing is to use a random
degree $(k-1)$-polynomial over some prime field
\cite{wegman81kwise}. Mersenne primes has been used for more than 40
years by anyone who
%, for $k>2$,
wanted an effient implementation
using standard portable code \cite{carter77universal}.

The speed of hashing is important because it is often an inner-loop
bottle-neck in data analysis. A good example is when hashing is used
in the sketching of high volume data streams, such as traffic through
an Internet router, and then this speed is critical to keep up with
the stream. A running example in this paper is the classic second
moment estimation using using 4-universal hashing in count sketches
\cite{charikar04count-sketch}. The count sketches a linear maps that
statistically preserve the Euclidean norm. They are very popular in
machine learning, where they adopted count sketches under the new name
feature hashing \cite{WDLSA09}.

In this paper, we argue that uniform hash values from a Mersenne prime
field with prime $p=2^b-1$ are not only fast to compute, but have
special advantages different from any other prime field.  We believe
we are the first to notice that such values can largely be treated as
uniform $b$-bit strings, that is, we can use the tool box of very
simple and efficient tricks for uniform $b$-bit strings.  From $[2^b]$
we are missing $p$, the all \texttt1s value, but a careful analysis
shows that the relative errors we get are in the order of $n/p^2$ or
less (this is for our example with second moment estimation, but our
analytic techniques are more general).

To put our results in perspective, suppose we were hashing $n$ keys
uniformly into $b$-bit strings.  The probability that there exists one
hashing to $p=2^b-1$ is at most $n/p$.  This means the total variation between
the two distributions is $n/p$ and any error probability we might have
proved assuming uniform $b$-bit hash values is off by at most $n/p$.
%This may be good if $p/n$ is sufficiently large.
\emph{In contrast, our analysis yields errors that differs from the uniform case by $n/p^2$ or less.}
%This allows good error bounds even as we set $p$ close to $n$, which again means that half the word operations can saved.
Loosely speaking, this means that we for a desired small error can reduce
the bit-length of the primes to less than half. This saves not only
space. It typically means that we can speed up the multiplications
with at least a factor 2.

In this paper, we also provide a new fast branch-free code for
division and modulus with Mersenne primes. Contrasting our analytic
work, this code generalizes to so-called Pseudo-Mersenne
primes~\cite{van2014encyclopedia} of the form $p=2^b-c$ for small
$c$. Our new code is simpler and faster than the classical algorithm of
Crandall~\cite{crandall1992method}. 

\subsection{Hashing uniformly into \texorpdfstring{$b$}{b} bits?}\label{sec:b-bit?}
A main point in this paper is that having hash values uniform in $[2^b-1]$
is almost as good as having uniform $b$-bit strings, but of course,
it would be even better if we just had uniform $b$-bit strings.

We do have the fast multiply-shift scheme of Dietzfelbinger~\cite{dietzfel96universal}, that directly gives 2-universal
hashing from $b$-bit strings to $\ell$-bit strings, but for $k>2$,
there is no such fast $k$-universal hashing scheme that
can be implemented with standard portable code.

More recently it has been suggested to use carry-less multiplication
for $k$-universal hashing into bit strings (see, e.g., Lamere
\cite{Lamere14}) but contrasting the hashing with Mersenne primes,
this is less standard (takes some work to get it to run on different
computers) and slower (by about 30-50\% on the computers we tested).
Morever, the code for different bit-lengths $b$ is quite different
because we need quite different irreducible polynomials.

Another alternative is to use tabulation based methods which are fast
but use a lot of space \cite{Siegel04,Tho13:simple-simple}, that is,
space $s=2^{\Omega(b)}$ to calculate $k$-universal hash function in
constant time from $b$-bit keys to $\ell$-bit hash values. The large
space can be problematic.

A classic example where constant space hash
functions are needed is in static two-level hash functions
\cite{FKS84}.  To store n keys with constant access time, they n
second level hash tables, each with its own hash function.  Another
example is small sketches such as the count sketch
\cite{charikar04count-sketch} discussed in this paper. Here we may
want to store the hash function as part of the sketch, e.g., to query
the value of a given key. Then the hash value has to be directly
computable from the small representation, ruling out tabluation based
methods (see further explanation at the end of \Cref{sec:count-sketch}).


It can thus be problematic to get efficient $k$-universal hashing directly into
$b$-bit strings, and this is why we in this paper analyze the
hash values from Mersenne prime fields that are much easier to generate.

\subsection{Polynomial hashing using Mersenne primes}

The $k$-universal hashing with a polynomial uses $O(k)$ space and $O(k)$ time
to compute the hash value of a key. Siegel \cite{Siegel04} has proved that if we want $k$-universal hashing in time $t<k$, then we need to use space $u^{1/t}$.
Such tabulation based methods are useful in many contexts (see survey \cite{Thorup17}, but not if we need small space.

A classic example where constant space hash functions are needed is in
static two-level hash functions \cite{FKS84}.  To store n keys with
constant access time, they n second level hash tables, each with its
own hash function.  Another example is small sketches such as the
count sketch \cite{charikar04count-sketch} discussed in this
paper. Here we may want to store the hash function as part of the
sketch, e.g., to query the value of a given key.


\subsubsection{Preliminaries: Implementation of a Hash Function}
The classic definition of $k$-universal hashing
goes back to Carter and Wegman~\cite{wegman81kwise}.
\begin{definition}
   A random hash function $h:U\to R$ is $k$-universal if for $k$
   distinct keys $x_0,\ldots,x_{k-1}\in U$, the $k$-touple
   $(h(x_0),\ldots,h(x_{k-1}))$ is uniform in $R^k$.
\end{definition}
\noindent
Note that the definition also implies the values 
$h(x_0),\ldots,h(x_{k-1})$ are independent.
A very similar concept is that of $k$-independence, which has only this requirement, but doesn't include that values must be uniform.

The classic example of $k$-universal
hash function is uniformly random degree-$(k-1)$ polynomial over a prime field
$\Z_p$, that is, we pick a uniformly random vector
$\vec a=(a_0,\ldots,a_{k-1})\in \Z_p^k$ of $k$ coefficients, and define
$h_{\vec a}:[p]\to[p]$,
\footnote{ We use the notation $[s]=\{0,\ldots,s-1\}$.  }
   by 
\[h_{\vec a}(x)=\sum_{x\in[k]}a_i x^i \mod p.\]
%
Given a desired key domain $[u]$ and range $[r]$ for the hash values, we pick
$p\geq \max\{u,r\}$ and define
$h^r_{\vec a}:[u]\to[r]$ by
\[h^r_{\vec a}(x)=h_{\vec a}(x)\bmod r.\]
The  hash values of $k$ distinct keys remain independent,
while staying as close as possible to the uniform distribution on $[r]$.
(This will turn out to be very important.)

In terms of speed, the main bottleneck in the above approach is the mod operations.
If we assume $r=2^\ell$, the $\bmod r$ operation above can be replaced by a binary {\sc and} (\texttt{\&}): $x \bmod r = x \andtt r-1$.
In the same vein, an old idea by 
Carter and Wegmen \cite{carter77universal} is to use a
Mersenne prime for $p=2^b-1$,\footnote{e.g., $p=2^{61}-1$ for hashing 32-bit keys or
$p=2^{89}-1$ for hashing 64-bit keys.}
to speed up the computation of the (mod $p$) operations.
The point is that
\begin{equation}
   y \bmod (2^b-1)
   \equiv (y\bmod 2^{b}) + \floor{y/2^b}
   \equiv (y \andtt p) + (y \rs b)
   \pmod {p}.
   \label{eq:Mersenne}
\end{equation}
%This leads to efficient computations because 
%\[y\bmod 2^{b}=y\andtt p\quad\textnormal{and}\quad\floor{y/2^b}=y\rs b.\]
Again allowing us to use the very fast bit-wise {\sc and} ($\andtt$) and the right-shift ($\rs$),
instead of the expensive modulo operation.


\vspace{1em}

The above completes our description of how Mersenne primes are
normally used for fast computation of $k$-universal hash functions.
We show an implementation in \Cref{alg:Mersenne} below with one further improvement:
By assuming that $p=2^b-1\geq 2u-1$
(which is automatically satisfied in the typical case where $u$ is a power
of two, e.g., $2^{32}$ or $2^{64}$)
we can get away with only testing the possible off-by-one in \Cref{eq:Mersenne} once, rather that at every loop.
Note the proof by loop invariant in the comments.

\begin{algorithm}[H]
   \caption{
      For $x\in [u]$, prime $p=2^b-1\geq 2u-1$,
      and $\vec a=(a_0,\ldots,a_{k-1})\in[p]^k$,
      computes $y=h_{\vec a}(x)=\sum_{i\in[k]}a_i x^i\mod p$.
   }\label{alg:Mersenne}
   \begin{algorithmic}
      \State $y\gets a_{k-1}$
      \For{$i=q-2,\ldots,0$}
      \Comment{Invariant: $\quad y<2p$}

      \State $y\gets y*x+a_i$
      \Comment{$\quad y<2p(u-1)+(p-1)<(2u-1)p\leq p^2$}

      \State $y\gets (y\andtt p)+(y\rs b)$
      \Comment{$\quad y<p+p^2/2^b<2p$}
      \EndFor
      \If{$y\geq p$}
      \State $y\gets y-p$
      \Comment{$y<p$}
      \EndIf
      %\State \Return $y$
   \end{algorithmic}
\end{algorithm}


In \Cref{subsec:intro-division} we will give one further improvement to \Cref{alg:Mersenne}.
Mostly the description above is a fairly standard description of state-of-the-art hashing.
\footnote{We note that $k=2$, we do have the fast multiply-shift scheme of Dietzfelbinger~\cite{dietzfel96universal}, that directly gives 2-universal
hashing from $b$-bit strings to $\ell$-bit strings, but for $k>2$,
there is no faster method that can be implemented with portable code
in a standard programming language like C.}

We stress that while this is a particularly fast implementation of Mersenne prime hashing, the main novelty of the paper will be in the analysis.

%The main point of this paper is our note from before, that the values hashed to $[r]$ are not completely uniform, as they would have been if $p=2^b$ was a prime.
%It turns out that with a novel analysis, bits from Mersenne primes are actually
%almost as good as if
%they were uniformly distributed $b$-bit strings (we are only missing
%the all \texttt{1}s value $2^b-1$). 





\subsubsection{Good bucketing with powers of two}\label{sec:power-of-two}
As a first illustration of the advantage that we get using a
Mersenne prime $p=2^b-1$, consider the case mentioned above where we
want hash values in the range $[r]$ where $r=2^\ell$ is a power of
two. We will often refer to the hash values in $[r]$ as buckets so
that we are hashing keys to buckets.
%Avoiding a degenerate case, we assume $r>1$. In particular this implies that $r$ does not divide our prime $p$.

We assume a $k$-universal hash function $h:[u]\to[p]$, e.g.,
the one from \Cref{alg:Mersenne}. To get hash values in $[r]$,
we defined $h^r:[u]\to[r]$ by
\[h^r(x)=h(x)\bmod r=h(x)\andtt (r-1).\]
As discussed previously, the hash values of up to $k$ distinct keys remain
independent with $h^r$. The issue is that hash values from 
$h^r$ are not quite uniform in $[r]$.

Recall that for any key $x$, we have $h(x)$ uniformly distributed in $[2^b-1]$.
This is the uniform distribution on $b$-bit strings except that we are
missing $p=2^b-1$. Now $p$ is the all \texttt{1}s, and 
$p \bmod r = p\andtt (r-1) = r-1$.
Therefore
\begin{align}
   \Pr[h^r(x)=i]
   &=\lceil p/r\rceil/p
   =((p+1)/r)/p
   =(1+1/p)/r
   \quad
   \text{for any $i < r-1$,}
   \label{eq:coll-ell<r-1}
   \\
   \text{while}\quad
   \Pr[h^r(x)=r-1]
   &=\lfloor p/r\rfloor/p=((p+1-r)/r)/p
   =(1-(r-1)/p)/r.
   \label{eq:coll-ell=r-1}
\end{align}
Thus $\Pr[h^r(x)=i]\leq (1+1/p)/r$ for all $i\in[r]$. This upper-bound
only has a relative error of $1/p$ from the uniform $1/r$. For
comparison, if we had used a prime of the form $p=2^b-a$ and $a<r$, then
we would only get an upper bound of $(1+a/p)/r$.
%Below we return to a Mersenne prime $p=2^b-1$

Combining \req{eq:coll-ell<r-1} and \req{eq:coll-ell=r-1} with
pairwise independence, for any distinct keys $x,y\in [u]$, we show that the
collision probability is bounded
\begin{equation}\begin{split}  
   \Pr[h^r(x)=h^r(y)]
      =(r-1)((1+1/p)/r)^2+((1-(r-1)r/p)/r)^2
      %\\&= \frac{r +(r^2-r)/p^2}{r^2}
      =(1+(r-1)/p^2)/r
      %\\& <(1+r/p^2)/r
   .\label{eq:coll}
\end{split}\end{equation}
We note that relative error $r/p^2$ is small as long as $p$ is
large.

\subsubsection{Selecting arbitrary bits from the hash value}
Interestingly, the above analysis holds no matter which $\ell$ bits we use when mapping the hash values from $[2^b-1]$ to $[2^\ell]$.
Let $\mu:[2^b]\to[2^\ell]$ be any map defined by selecting $\ell$ bits from a $b$-bit string.
Above we used $\mu(y)=y\bmod 2^\ell=y\andtt  (2^\ell-1)$, selecting the $\ell$ least significant bits, but we could also use $\mu(y)=\floor{y/2^{b-\ell}}=y\rs (b-\ell)$ selecting the $\ell$ most significant bits.
The basic point is that a uniform distribution on $[2^b]$ maps to a uniform distribution on $[2^\ell]$.
We are only missing the all \texttt1s value $p=2^b-1$ which maps to $2^\ell-1$ regardless of which $\ell$ bits we select, so our equations \req{eq:coll-ell<r-1}--\req{eq:coll} hold no matter which $\ell$ bits we select for $h^r$.

The fact that it doesn't matter which $\ell$ bits we select is only
true because we use a Mersenne prime $p=2^b-1$. Suppose we used some
other $b$-bit prime $p=2^b-a$ where $2^{b-\ell}<a<2^{b-1}$. If we
select the $\ell$ most signifiant bits, then $0$ elements from $[p]$
map to $2^\ell-1$ while $2^{b-\ell}$ elements from $[p]$ map to $0$. However,
with the $\ell$ least significant bits, we have $\floor{p/2^\ell}$ or
$\ceil{p/2^\ell}$ elements from $[p]$ mapping to each element in
$[2^\ell]$, so the maximal difference is 1.


\subsection{Two-for-one hash functions in second moment estimation}
In this section we discuss how we can get several hash functions for
the price of one, and apply the idea to second moment estimation using
count sketches \cite{charikar04count-sketch}.

Suppose we had a $k$-universal hash function into $b$-bit strings.
We note that using standard programming languages such as C, we have
no simple and efficient method computing such hash
functions when $k>2$. However, later we will argue that polynomial
hashing using a Mersenne prime $2^b-1$ delivers a better-than-expected
approximation.

Let $h:U\to [2^b]$ be $k$-universal. By definition this
means that if we have $j\leq k$ distinct keys $x_1,\ldots,x_j$, then
$(h(x_1),\ldots,h(x_j))$ is uniform in $[2^b]^j\equiv [2]^{bj}$,
so this means that \emph{all} the bits in $h(x_1),\ldots,h(x_j)$ are
independent and uniform. We can use this to split our $b$-bit hash
values into smaller segments, and sometimes use them as if
they were the output of universally computed hash functions.
We illustrate this idea below in the context of the second moment estimation.

\subsubsection{Second moment estimation}\label{sec:count-sketch}
We now review the second moment estimation of streams based on count
sketches \cite{charikar04count-sketch} (which are based on the
celebrated second moment AMS-estimator from \cite{alon96frequency})

The basic set-up is as follows.  For keys in $[u]$ and integer values in $\Z$, we are given a stream of key/value $(x_0,\Delta_0),\ldots, (x_{n-1},\Delta_{n-1})\in [u]\times\Z$. The
total value of key $x\in[u]$ is
\[f_x=\sum_{i\in[n],x_i=x} \Delta_i.\]
We let $n\leq u$ be  the number of non-zero values
$f_x\neq 0$, $x\in [u]$. Often $n$ is much smaller than $u$.
We define the $m$th moment $F_m = \sum_{x\in [u]}f_y^m$. The goal here is to
estimate the second moment $F_2 = \sum_{x\in [u]}f_x^2=\|f\|^2_2$. 

\begin{algorithm}[H]
   \caption{\label{alg:count-sketch} Count Sketch. Uses a
      vector/array $C$ of $r$ integers and two independent
      4-universal hash functions $i:[u]\to[r]$ and $s:[u]\to\{-1,1\}$.
   .}
   \begin{algorithmic}
      \Procedure{Initialize}{}
         \State For $i\in[t]$, set $C[i]\gets 0$.
      \EndProcedure
      \Procedure{Process}{$x, \Delta$}
         \State $C[i(x)]\gets C[i(x)]+s(x) \Delta$. 
      \EndProcedure
      \Procedure{Output}{}
         \State \Return $\sum_{i\in[t]} C[i]^2$.
      \EndProcedure
   \end{algorithmic}
\end{algorithm}
The standard analysis \cite{charikar04count-sketch} shows that 
\begin{align}
   \E[X]&= F_2 \label{eq:E-F2}\\
   \Var[X]&=2(F_2^2 - F_4)/r<2F_2^2/r \label{eq:V-F2}
\end{align}
As $r$ grows we see that $X$ concentrates around $F_2=\|f\|^2_2$. Here
$X=\sum_{i\in[r]} C[i]^2=\|C\|^2_2$. Now $C$ is a randomized 
function of $f$, and as $r$ grows, we get $\|C(f)\|^2_2\approx\|f\|^2_2$,
implying $\|C(f)\|_2\approx\|f\|_2$, that is, the Euclidean norm is
roughly preserved by the count sketch. However, the count sketch
is also a linear function, so Euclidean distances are preserved, that
is, for any $f,g\in \Z^u$,
\[\|f-g\|_2\approx \|C(f-g)\|_2=\|C(f)-C(g)\|_2.\]
Thus, when we want to find close vectors, we can just work with the
much smaller count sketches. This is crusial to machine learning,
where they adopted count sketches under the new name feature hashing
\cite{WDLSA09}.

In \Cref{sec:b-bit?} we mentioned that the count sketch $C$ can also
be used to estime any single value $f_x$. To do this, we use
the unbiased estimator $X_x=s(x)C[i(x)]$. This is yet another standard use
of count sketch \cite{charikar04count-sketch}. It requires
direct access to both the sketch $C$ and the two hash functions $s$ and $i$.

\subsubsection{Two-for-one hash functions with \texorpdfstring{$b$}{b}-bit hash values}
As the count sketch is described above,
it uses two independent 4-universal hash functions
$i:[u]\to[r]$ and $s:[u]\to\{-1,1\}$, but 4-universal hash functions
are generally slow to compute, so, aiming to save roughly a factor 2
in speed, a tempting idea is to compute them both using a single hash
function.

The analysis behind \req{eq:E-F2} and \req{eq:V-F2} does not quite
require $i:[u]\to[r]$ and $s:[u]\to\{-1,1\}$ to be independent.
It suffices that the hash values are uniform and that for any
given set of $j\leq 4$ distinct keys $x_1,\ldots,x_j$, the $2j$ hash
values $i(x_1),\ldots,i(x_j),s(x_1),\ldots,s(x_j)$ are independent.
A critical step in the analysis is that if
$A$ depends on $i(x_1),\ldots,i(x_j),s(x_2),\ldots,s(x_j)$, but
not on $s(x_1)$, then
\begin{equation}\label{eq:E-0}
  \E[s(x_0) A] = 0 .
\end{equation}
This follows because $\E[s(x_1)]=0$ by uniformity of $s(x_1)$ and because $s(x_1)$ is independent of $A$.


Assuming that $r=2^\ell$ is a power of two, we can easily construct
$i:[u]\to[r]$ and $s:[u]\to\{-1,1\}$ using a single $4$-universal
hash function $h:[u]\to[2^b]$ where $b>\ell$. Recall that all the bits in
$h(x_1),\ldots,h(x_4)$ are independent. We can therefore use the
$\ell$ least significant bits of $h(x)$ for $i(x)$ and the most
significant bit of $h(x)$ for a bit $a(x)\in[2]$, and finally set
$s(x)=1-2a(x)$. It is then easy to show that if $h$ is $k$-universal
than $h$ satisfies \cref{eq:E-0}.
\begin{algorithm}[H]
   \caption{For key $x\in [u]$, compute $i(x)=i_x\in[2^\ell]$ and
      $s(x)=s_x\in\{-1,1\}$,\rule{5ex}{0ex}
   using $h:[u]\to [2^b]$ where $b>\ell$.}
   \label{alg:h-and-s}
   \begin{algorithmic}
      \State $h_x\gets h(x)$
      \Comment $h_x$ uses $b$ bits
      \State $i_x\gets h_x \andtt (2^\ell-1)$
      \Comment $i_x$ gets $\ell$ least significant bits of $h_x$
      \State $a_x\gets h_x\rs (b-1)$
      \Comment $a_x$ gets the most significant bit of $h_x$
      \State $s_x\gets 1-(a_x\ls1)$
      \Comment $a_x\in[2]$ is converted to a sign $s_x\in\{-1,1\}$
   \end{algorithmic}
\end{algorithm}
% \begin{lemma}\label{lem:b-bit-hashing} Suppose $h:[u]\to[2^b]$ is $k$-universal. Let
%    $i:[u]\to[2^\ell]$ and
%    $s:[u]\to\{-1,1\}$ be constructed from $h$ as described in Algorithm \ref{alg:h-and-s}. Then $h$ and $s$ are both $k$-universal. Moreover, for
%    any $j\leq k$ distinct keys $x_1,\ldots,x_j$, the $2j$ hash
%    values $i(x_1),\ldots,i(x_j),s(x_1),\ldots,s(x_j)$ are universal.
%    In particular, if $A$ depends on
%    $i(x_1),\ldots,i(x_j),s(x_2),\ldots,s(x_j)$, but not on $s(x_1)$, then
%    \begin{equation}\label{eq:E-0}
%       \E[s(x_1)A]=0
%    \end{equation}
% \end{lemma}
Note that Algorithm \ref{alg:h-and-s} is well defined as long as 
$h$ returns a $b$-bit integer. However, \cref{eq:E-0} requires
that $h$ is $k$-universal into $[2^b]$, which in particular implies that
the hash values are uniform in $[2^b]$.


\subsubsection{Two-for-one hashing with  Mersenne primes}\label{sec:two-for-one}
Above we discussed how useful it would be with $k$-universal hashing
mapping uniformly into $b$-bit strings. The issue was that the lack of
efficient implementations with standard portable code if
$k>2$. However, when $2^b-1$ is a Mersenne prime $p\geq u$, then we do
have have the efficient computation from Algorithm \ref{alg:Mersenne}
of a $k$-universal hash function $h:[u]\to[2^b-1]$. The hash values
are $b$-bit integers, and they are uniformly distributed, except that
we are missing the all \texttt{1}s value $p=2^b-1$. We want to
understand how this missing value affects us if we try to split the
hash values as in Algorithm \ref{alg:h-and-s}. Thus, we assume a
$k$-universal hash function $h:[u]\to[2^b-1]$ from which we construct
$i:[u]\to[2^\ell]$ and $s:[u]\to\{-1,1\}$ as
described in Algorithm \ref{alg:h-and-s}. As usual, we assume $2^\ell>1$.
Since $i_x$ and $s_x$ are
both obtained by selection of bits from $h_x$, we know from Section
\ref{sec:power-of-two} that each of them have close to uniform
distributions. However, we need a good replacement for \req{eq:E-0}
which besides uniformity, requires $i_x$ and $s_x$ to be independent,
and this is certainly not the case.

Before getting into the analysis, we argue that we really do get two
hash functions for the price of one. The point is that our efficient
computation in Algorithm \ref{alg:Mersenne} requires that we use a
Mersenne prime $2^b-1$ such that $u\leq 2^{b-1}$, and this is even if
our final target is to produce just a single bit for the sign function
$s:[u]\to\{-1,1\}$. We also know that $2^\ell<u$, for otherwise we
get perfect results implementing $i:[u]\to[2^\ell]$ as the identifty
function (perfect because it is collision free).  Thus we can assume
$\ell<b$, hence that $h$ provides enough bits for both $s$ and $i$.


We now consider the effect of the hash values from $h$ being uniform
in $[2^b-1]$ instead of in $[2^b]$. Suppose we want to compute the
expected value of an expression $B$ depending only on the independent
hash values $h(x_1),\ldots,h(x_j)$ of $j\leq k$ distinct keys
$x_1,\ldots,x_j$.

Our generic idea is to play with the distribution of $h(x_1)$ while
leaving the distributions of the other independent hash values
$h(x_2)\ldots,h(x_j)$ unchanged, that is, they remain uniform in
$[2^b-1]$. We will consider having $h(x_1)$ uniformly distributed in
$[2^b]$, denoted $h(x_1) \sim \unif[2^b]$, but then we later have to
subtract the ``fake'' case where $h(x_1)=p=2^b-1$.  Making the
distribution of $h(x_1)$ explicit, we get
\begin{equation}\begin{split}
  \E_{h(x_1) \sim \unif[p]}[B]&=\sum_{y\in[p]}\E[B \mid h(x_1)=y]/p
                            \\&=\sum_{y\in[2^b]}\E[B \mid h(x_1)=y]/p - \E[B \mid h(x_1)=p]/p
                            \\ &=\E_{h(x_1) \sim \unif[2^b]}[B](p+1)/p - \E[B \mid h(x_1)=p]/p.\label{eq:play-with-dist}
\end{split}\end{equation}
Let us now apply this idea our situation where $i:[u]\to[2^\ell]$ and
$s:[u]\to\{-1,1\}$ are constructed from $h$ as described in Algorithm
\ref{alg:h-and-s}. We will prove
\begin{lemma}\label{lem:remove-si}  Consider distinct keys $x_1,\ldots,x_j$, $j\leq k$ and an expression $B=s(x_1)A$ where $A$
   depends on $i(x_1),\ldots,i(x_j)$ and $s(x_2),\ldots,s(x_j)$ but not
   $s(x_1)$. Then
   \begin{equation}\label{eq:remove-si}
      \E[s(x_1)A]=\E[A\mid i(x_1)=2^\ell-1]/p.
   \end{equation}
\end{lemma}
\begin{proof}
When $h(x_1) \sim \unif[2^b]$, then $s(x_1)$ is uniform
in $\{-1,1\}$ and independent of $i(x_1)$. The remaining
$(i(x_i),s(x_i))$, $i>1$, are independent of $s(x_1)$ because they
are functions of $h(x_i)$ which is independent of $h(x_1)$, so
we conclude that 
\[\E_{h(x) \sim \unif[2^b]}[s(x_1)A]=0\]
Finally, when $h(x_1)=p$, we get $s(x_1)=-1$ and $i(x_1)=2^\ell-1$, 
so applying \req{eq:play-with-dist}, we conclude
that 
\[\E[s(x_1)A] = -\E[s(x_1) A \mid h(x_1) = p]/p = \E[A \mid i(x)=2^\ell-1]/p.\]
\end{proof}
Above \req{eq:remove-si} is our replacement for \req{eq:E-0}, that is,
when the hash values from $h$ are uniform in $[2^b-1]$ instead of
in $[2^b]$, then $\E[s(x_1)B]$ is reduced by $\E[B \mid i(x)=2^\ell-1]/p$.
For large $p$, this is a small additive error. Using this in a careful
analysis, we will show that our fast second moment estimation 
based on Mersenne primes performs almost perfectly:

\begin{theorem}\label{thm:h-and-s-p}
   Let $r>1$ and $u>r$ be powers of two and let $p=2^b-1>u$ be a
   Mersenne prime.
   Suppose with have a 4-universal hash function $h:[u]\to[2^b-1]$, e.g.,
   generated using Algorithm \ref{alg:Mersenne}. Suppose
   $i:[u]\to[r]$ and
   $s:[u]\to\{-1,1\}$ are constructed from $h$ as described in
   Algorithm \ref{alg:h-and-s}. Using this $i$ and $s$ 
   in the Count Sketch Algorithm \ref{alg:count-sketch}, the second moment 
   estimate $X=\sum_{i\in[k]} C_i^2$ satisfies:
   \begin{align}
      \E[X]&=F_2+(F_1^2-F_2)/p^2 < (1+n/p^2)\,F_2\textnormal,\label{eq:E-F2-p}\\
      | \E[X] - F_2 | &\le F_2 (n - 1)/p^2, \label{eq:E-F2-p-com}\\
      \Var[X]&< 2(F_2^2-F_4)/r+F_2^2 (2.33+4 n/r)/p^2<2F_2^2/r.\label{eq:V-F2-p}
   \end{align}
\end{theorem}
The difference from \req{eq:E-F2} and \req{eq:V-F2} 
is negligible when $p$ is large. Theorem \ref{thm:h-and-s-p} will be
proved in Section \ref{sec:analysis-two-for-one}.


\subsection{Arbitrary number of buckets}\label{sec:most-uniform}
We now consider the general case where we want to hash into a set of buckets $R$ whose size is not a power of two.
Suppose we have a $2$-universal hash function $h:U\to Q$.
We will compose $h$ with a map $\mu:Q\to R$, and use $\mu\circ h$ as a hash function from $U$ to $R$.
Let $q=|Q|$ and $r=|R|$.
We want the map $\mu$ to be \emph{most uniform} in the sense that for bucket $i\in R$, the number of elements from $Q$ mapping to $i$ is either $\floor{q/r}$ or $\ceil{q/r}$.
Then the uniformity of hash values with $h$ implies for any key $x$ and bucket $i\in R$ \[\floor{q/r}/q\leq \Pr[\mu(h(x))=i]\leq \ceil{q/r}/q.\]
Below we typically have $Q=[q]$ and $R=[r]$.
A standard example of a most uniform map $\mu:[q]\to[r]$ is $\mu(x)=x\bmod r$ which the one used above when we defined $h^r:[u]\to[r]$, but as we mentioned before, the modulo operation is quite slow unless $r$ is a power of two.

Another example of a most uniform map $\mu:[q]\to[r]$ 
is $\mu(x)=\floor{xr/q}$,
which is also quite slow in general, but if $q=2^b$ is a power of two,
it can be implemented as $\mu(x)=(xr)\rs\,b$ where 
$\rs$ denotes right-shift. This would be yet another advantage 
of of having $k$-universal hashing into $[2^b]$.

Now, our interest is the case where $q$ is a Mersenne prime $p=2^b-1$. We want
an efficient and most uniform map $\mu:[2^b-1]$ into any given $[r]$.
Our simple solution is to define
\begin{equation}\label{eq:most-uniform}
   \mu(x)=\floor{(x+1)r/2^b}=((x+1)r)\rs b.
\end{equation}
Lemma \ref{lem:most-uniform} (iii) below 
states that \req{eq:most-uniform} indeed
gives a most uniform map. 
\begin{lemma}\label{lem:most-uniform} Let $r$ and $b$ be positive integers.
   %, and let $x\in [2^b-1]$.
   Then
   \begin{itemize}
      \item[(i)] $x\mapsto (xr)\rs\,b$ is a most
         uniform map from $[2^b]$ to $[r]$.
      \item[(ii)] $x\mapsto (xr)\rs\,b$ is a most
         uniform map from $[2^b]\setminus\{0\}=\{1,\ldots,2^b-1\}$ to $[r]$.
      \item[(iii)] $x\mapsto ((x+1)r)\rs \, b$ is a most
         uniform map from $[2^b-1]$ to $[r]$.
   \end{itemize}
\end{lemma}
\begin{proof}
   Trivially (ii) implies (iii). 
   The statement (i) is folklore and easy to prove, so we know that every
   $i\in[r]$ gets hit by $\floor {2^b/r}$ or $\ceil{2^b/r}$ elements from
   $[2^b]$. It is also clear that $\ceil{2^b/r}$ elements, including $0$,
   maps to $0$. To prove (ii), we remove $0$ from $[2^b]$, 
   implying that only
   $\ceil{2^b/r}-1$ elements map to $0$. For all positive integers $q$
   and $r$, $\ceil{(q+1)/r}-1=\floor{q/r}$, and we use this here with 
   $q=2^b-1$. It follows that all buckets from $[r]$ gets $\floor{q/r}$
   or $\floor{q/r}+1$ elements from $Q=\{1,\ldots,q\}$. If $r$ does
   not divide $q$ then $\floor{q/r}+1=\ceil{q/r}$, as desired. However,
   if $r$ divides $q$, then $\floor{q/r}=q/r$, and this
   is the least number of elements from $Q$ hitting any bucket in $[r]$. Then 
   no bucket from $[r]$ can get hit by more than $q/r=\ceil{q/r}$ 
   elements from $Q$. This completes the proof of (ii), and hence of (iii).
\end{proof}
We note that our trick does not work when $q=2^b-c$ for $c\geq 2$, that is,
using $x\mapsto ((x+c)r)\rs  b$, for in this general case, 
the number of elements hashing to $0$ is $\ceil {2^b/r}-c$, or $0$ if
$c\geq \floor {2^b/r}$.
One may try many other hash functions $(c_1 x r+ c_2 x+ c_3 r + c_4) \rs b$ similarly without any luck.
Our new uniform map from \req{eq:most-uniform} is thus very specific to Mersenne prime fields.
For general $c\ge 2$ we provide a scheme using two shifts in
Section \ref{sec:pseudo-arbitrary}.

We will see in Section \ref{sec:two-for-one} that our new uniform map
works very well in conjunction with the idea of splitting of hash values.




\subsection{Branch-free division and modulo with (Pseudo) Mersenne Primes}\label{subsec:intro-division}
%
We even suggest a fast branch-free computation of $\bmod\,p$ for
Mersenne primes $p=2^b-1$. The issue in Algorithm \ref{alg:Mersenne}
is that the if-statement can be slow because of issues with branch
prediction; for It implies that different statements are run for
different keys $x$.

More specifically, in Algorithm \ref{alg:Mersenne}, after the last
multiplication, we have a number $y<p^2$ and we want to compute the
final hash value $y\bmod p$. We obtained this using the following
statements, each of which preserve the value modulo $p$, starting from
$y<p^2$:
\begin{algorithmic}
   \State $y \gets (y\andtt p)+(y\rs b)$
   \Comment $y<2p$
   \If{$y\ge p$}
   \State $y\gets y-p$
   \Comment  $y<p$
   \EndIf
\end{algorithmic}
To avoid the if-statement, in Algorithm \ref{alg:div-simple}, we suggest
a branch-free code that starting
from $x<2^{2b}$ computes both $y=x\bmod p$ and $z=\floor{x/p}$ using
a small number of AC$^0$ instructions. 
\begin{algorithm}[H]
   \caption{For Mersenne prime $p=2^b-1$ and $x< 2^{2b}$, compute
   \label{alg:div-simple}
   $y=x\bmod p$ and $z=\floor{x/p}$}
   \begin{algorithmic}
      \State $\rhd$ First we compute $z=\floor{x/p}$
      \State $x'=x+1$
      \State $z \gets(( x' \rs b)+x')\rs b$
      %\State $y \gets x - (z \ls b) + z$.
      \State $\rhd$ Next we compute $y=x\bmod p$ given $z=\floor{x/p}$
      \State $y \gets (x + z) \andtt p $
   \end{algorithmic}
\end{algorithm}
In Algorithm \ref{alg:div-simple}, we use
$z=\floor{x/p}$ to compute $y=x\bmod p$. If we only want the
division $z=\floor{x/p}$, then we can skip the last statement.

Below we will generalize Algorithm \ref{alg:div-simple} to work for
arbitrary $x$, not only $x<2^{2b}$. Moreover, we will generalize
to work for different kinds of primes generalizing Mersenne primes:
\begin{description}
   \item[Pseudo-Mersenne Primes]
      are primes on the form $2^b-c$, where is usually required that $c < 2^{\lfloor b/2\rfloor}$~\cite{van2014encyclopedia}.
      Crandal patented a method for working with Pseudo-Mersenne Primes in 1992~\cite{crandall1992method},
      why those primes are also sometimes called ``Crandal-primes''.
      The method was formalized and extended by Jaewook Chung and Anwar Hasan in 2003~\cite{chung2003more}. The method we present is simpler with
      stronger guarantees and better practical performance.
      We provide a comparison with the Crandal-Chung-Hansan method in Section 4.
      %also sometimes known as Crandall primes, are numbers on the form $2^n - c$ for a small odd $c$.
   \item[Generalized Mersenne Primes]
      also sometimes known as Solinas primes~\cite{Solinas2011}, are sparse numbers, that is $f(2^b)$ where $f(x)$ is a low-degree polynomial.
      Examples are the primes in NIST's document ``Recommended Elliptic Curves for Federal Government Use''~\cite{nist}:
         $p_{192} = 2^{192} - 2^{64} - 1$
      and
         $p_{384} = 2^{384}-2^{128}-2^{96}+2^{32}-1$.
      We simply note that Solinas primes form a special case of
      Pseudo-Mersenne Primes, where multiplication with $c$
      can be done using a few shifts and additions.
\end{description}
We will now first generalize the division from Algorithm \ref{alg:div-simple}
to cover arbitrary $x$ and division with arbitrary Pseudo-Mersenne primes $p=2^b-c$. This is done Algorithm \ref{alg:division-generalized} below which
works also if $p=2^b-c$ is not a prime.  The
simple division in Algorithm \ref{alg:div-simple} corresponds to the case
where $c=1$ and $m=2$.
\begin{algorithm}[H]
  \caption{Given integers $p=2^b-c$ and $m$.
    For any $x< (2^b/c)^m$, compute $z=\floor{x/p}$}
   \label{alg:division-generalized}
   \begin{algorithmic}
      %\Procedure{Divide}{x, n, c}
         \State $x' \gets x + c$
         \State $z \gets x' \rs b$
         %\For{$i\gets 1$ \textbf{to} $m$}
         \For{ $m-1$ times}
            \State $z \gets (z * c + x')\rs b$
         \EndFor
         %\EndFor
         %\State \Return $v$
      %\EndProcedure
   \end{algorithmic}
\end{algorithm}
The proof that Algorithm \ref{alg:division-generalized} correctly computes
 $z=\floor{x/p}$ is provided in in Section \ref{sec:division}.
Note that $m$ can be computed in advance from $p$, and there is no requirement that it is chosen as small as possible.
For Mersenne and Solinas primes, the multiplication $z*c$ can be done very fast.

Mathematically the algorithm computes the nested division
$$
\bbfloor{\frac{x}{q-c}}
=
\bbfloor{\frac{
   \bfloor{\frac{
      \floor{\frac{
         \dots+x+c
      }{q}}c +x+c
   }{q}}c +x+c
}{q}}
\vspace{-1em} % Move the next line further up
$$
which is visually similar to the series expansion
$
   \frac{x}{q-c}
   = \frac{x}{q}\sum_{i=0}^\infty (\frac{c}{q})^i
   %= x\frac{1+\frac{c+\frac{c^2 + \dots}{q}}{q}}{q}
   = \frac{\frac{\frac{\dots+x}{q}c+x}{q}c+x}{q}.
$
It is natural to truncate this after $m$ steps for a $(c/q)^m$ approximation.
The less intuitive part is that we need to add $x+c$ rather than $x$ at each step, to compensate for rounding down the intermediate divisions.

\paragraph{Computing mod}
We will now compute the $\bmod$ operation assuming that
we have already computed $z=\floor{x/p}$. Then
\begin{align}
   x \bmod p
   = x - pz
   = x - (2^b-c)z
   = x - (z\ls b) - c*z,
\end{align}
which is only two additions, a shift, and a multiplication with $c$ on top of the division algorithm.
As $pz = \floor{x/p}p \le x$ there is no danger of overflow.
We can save one operation by noting
that if $x = z (2^b-c) + y$, then
$$x\bmod p = y=\left(x+c*z \right) \bmod 2^b.$$
This is the method presented in Algorithm \ref{alg:mod-generalized} and applied with $c=1$ in Algorithm \ref{alg:div-simple}.
\begin{algorithm}[H]
   \caption{For integers $p=2^b-c$ and $z=\floor{x/p}$ compute
      $y=x \bmod p$.}
   \label{alg:mod-generalized}
   \begin{algorithmic}
      \State $y \gets (x + z*c) \andtt (2^b-1)$
   \end{algorithmic}
\end{algorithm}



%In the case $x\le 2^{2b}$ and $c=1$, we get the simplified Algorithm \ref{alg:div-simple} described above: $ \left\lfloor\frac{x}{2^n-1}\right\rfloor = (x+1 \rs n)+x+1 \rs n$.

\subsection{Application to arbitrary number of buckets}\label{sec:pseudo-arbitrary}
In Subsection~\ref{sec:most-uniform} we discussed how $\floor{\frac{h(x)r}{2^b-1}}$ provides a most uniform map from $[2^b-1]\to[r]$.
To avoid the division step, we instead considered the map
$\floor{\frac{(h(x)+1)r}{2^b}}$.
However, for primes on the form $2^b-c$, $c>1$ this approach doesn't provide a most-uniform map.
%
Instead we may use Algorithm \ref{alg:division-generalized} to compute
$$\left\lfloor\frac{h(x)r}{2^b-c}\right\rfloor$$
directly, getting a perfect most-uniform map.
%(Another alternative was to pre-compute $q = \lfloor2^b/p\rfloor$ and take
%$\floor{\frac{h(x)rq}{2^b}}$, however that requires larger words to store the product $h(x)rq$.)


\subsubsection{Related Algorithms}

Modulus computation by Generalized Mersenne primes is widely used in the Cryptography community.
For example, four of the recommended primes in NIST's document ``Recommended Elliptic Curves for Federal Government Use''~\cite{nist} are Generalized Mersenne.
Naturally, much work has been done on making computations with those primes fast.
Articles like ``Simple Power Analysis on Fast Modular Reduction with Generalized Mersenne Prime for Elliptic Curve Cryptosystems''~\cite{sakai2006simple}
give very specific algorithms \emph{for each} of a number of well known such primes.
An example is shown in \Cref{alg:solina}.

\begin{algorithm}[H]
   \caption{Fast reduction modulo $p_{192} = 2^{192} - 2^{64} - 1$}
   \label{alg:solina}
   \begin{algorithmic}
      \State \textbf{input} $c \gets (c_5, c_4, c_3, c_2, c_1, c_0)$, where each $c_i$ is a 64-bit word, and $0 \le c < p^2_{192}$.
      \State $s_0 \gets (c_2, c_1, c_0)$
      \State $s_1 \gets (0, c_3, c_3)$
      \State $s_2 \gets (c_4, c_4, 0)$
      \State $s_3 \gets (c_5, c_5, c_5)$
      \State \textbf{return} $s_0 + s_1 + s_2 + s_3 \mod p_{192}$.
   \end{algorithmic}
\end{algorithm}

Division by Mersenne primes is a less common task, but a number of well known division algorithms can be specialized, such as 
 classical trial division, Montgomery's method and Barrett reduction.


%Montgomery method:
%\begin{align}
%   (aR\mod N)(bR\mod N) \mod N = (abR)R \mod N
%\end{align}
%We then need to remove the factor of $R$ by multiplying with its inverse $\mod N$.


The state of the art appears to be the modified Crandall Algorithm by Chung and Hasan~\cite{chung2006low}.
This algorithm, given in Algorithm \ref{alg:cch} modifies Crandall's algorithm~\cite{crandall1992method} from 1992 to compute division as well as modulo for generalized $2^b-c$ Mersenne primes.
\begin{algorithm}[H]
   \caption{Crandall, Chung, Hassan algorithm. For $p=2^b-c$, computes $q, r$ such that $x = qp+r$ and $r<p$.}
   \label{alg:cch}
   \begin{algorithmic}
      %\Procedure{Divide}{x, n, c}
         \State $q_0 \gets x \rs n $
         \State $r_0 \gets x \andtt 2^b-1$
         \State $q \gets q_0, r\gets r_0$
         \State $i \gets 0$
         \While{$q_i>0$}
            \State $t \gets q_i*c$
            \State $q_{i+1} \gets t \rs n$
            \State $r_{i+1} \gets t \andtt (2^b - 1)$
            \State $q\gets q+q_{i+1}$
            \State $r\gets r+r_{i+1}$
            \State $i\gets i+1$
         \EndWhile
         \State $t \gets 2^b-c$
         \While{$r\ge t$}
            \State $r\gets r-t$
            \State $q\gets q+1$
         \EndWhile
         \State\textbf{return} $q$
      %\EndProcedure
   \end{algorithmic}
\end{algorithm}
The authors state that for $2n+\ell$ bit input, Algorithm \ref{alg:cch}
requires at most $s$ iterations of the first loop, if $c < 2^{((s-1)n-\ell)/s}$.
This corresponds roughly to the requirement $x < 2^b (2^b/c)^s$, similar to ours.

Unfortunately the algorithm ends up doing double work, by computing the quotient and remainder concurrently.
The algorithm also suffers from the extra while loop for ``cleaning up'' the computations after the main loop.



Chung and Hasan also has an earlier, simpler algorithm from 2003~\cite{chung2003more},
but it appears to give the wrong result for many simple cases.
This appears to be because of a lack of the ``clean up'' while loop at the end of Algorithm \ref{alg:cch}.


% On the number of Mersenne Primes:
%Unfortunately there are only 45 of them known.
%The most useful one perhaps being.
%Heuristically there are $O(\log x)$ Mersenne primes up to $x$.
%Trivia: Euler proved that an even number $n$ is perfect if and only if it is on the form $n=2^{q-1}M_q$, where $M_q=2^q-1$ is prime.
%(Usually we know a number is perfect if its divisors sum to the number itself, e.g. $6=1+2+3$ or $28=1+2+4+7+14$.)


%[[Curve448]] uses the Solinas prime <math>2^{448} - 2^{224} - 1</math>
