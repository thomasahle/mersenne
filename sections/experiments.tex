%! TEX root = ../mersenne.tex
\section{Experiments}\label{sec:experiments}
We perform experiments on fast implementations of Mersenne hashing (\Cref{alg:Mersenne}) and our Mersenne division algorithm (\Cref{alg:division-generalized}).

TODO: What experiment?

\begin{table}[H]
   \centering
   \begin{tabular}{c c}
      \begin{tabular}{r | r r}
         $k$ & Mersenne       & Carryless \\
         \hline
         2 & 23.6 & \textbf{15.1} \\
         4 & \textbf{65.7} & \textbf{65.7} \\
         8 & \textbf{178.4} & 242.4 \\
      \end{tabular}
      \hspace{.5em}
      &
      \hspace{.5em}
      \begin{tabular}{r | r r}
         $k$ & Mersenne       & Carryless \\
         \hline
         $2$ & 19.0           & \textbf{16.7} \\
         $4$ & \textbf{68.7}  & 68.8 \\
         $8$ & \textbf{187.4} & 246.8
      \end{tabular}
   \end{tabular}
   \caption{Milliseconds for $10^7$ $k$-universal hashing operations on 64bit keys with $p=2^{89}-1$.
      The standard deviation is less than $\pm1$ms.
         On the left, Intel Core i7-8850H.
         On the right, Intel Core i7-86650U.
   }
   \label{tab:hashing-experiments}
\end{table}

The results in \Cref{tab:hashing-experiments} show that our methods outperform Carryless Multiplication for larger $k$, while being slower for $k=2$.
We note though that the multiply-shift scheme~\cite{dietzfel96universal} is better yet in that regime.
For $k=4$, which we use for Count Sketch, the results are a tossup.
However, we note that our methods are more portable than Carryless, and we keep the two-for-one advantages described in the article.

\vspace{.5em} % Soft section break

We tested \Cref{alg:division-generalized} against the state of the art modified Crandall's algorithm by Chung and Hasan (\Cref{alg:cch}), as well as the built in truncated division algorithm in the GNU MultiPrecision Library, GMP~\cite{granlund2010gnu}.

\begin{table}[H]
   \centering
   \begin{tabular}{ c c }
      \begin{tabular}{ r | r r r }
         $b$ & Crandall & \Cref{alg:division-generalized} & GMP \\
         \hline
         32 & 396 & \textbf   {138}  & 149\\
         64 & 381 &   \textbf {142}  & 161\\
         128 & 564 &  \textbf {157}  & 239\\
         256 & 433 &  \textbf {187}  & 632\\
         512 & 687 &  \textbf {291}  & 1215\\
         1024 & 885 & \textbf {358}  & 2802
      \end{tabular}
      \hspace{.5em}
      &
      \hspace{.5em}
      \begin{tabular}{ r | r r r }
         $b$ & Crandall & \Cref{alg:division-generalized} & GMP \\
         \hline
         32 & 438 & 172 & \textbf{125}\\
         64 & 422 & 172 & \textbf{141}\\
         128 & 578 &      \textbf{188} & 235\\
         256 & 454 &      \textbf{219} & 469\\
         512 & 703 &      \textbf{297} & 938\\
         1024 & 875 &     \textbf{391} & 2172
      \end{tabular}
   \end{tabular}
   \caption{Milliseconds for $10^7$ divisions of $2b$-bit numbers with $p=2^b-1$.
      The standard deviation is less than $\pm10$ms.
         On the left, Intel Core i7-8850H.
         On the right, Intel Core i7-86650U.
   }
   \label{tab:division-experiments}
\end{table}

The results in \Cref{tab:division-experiments} show that our method always outperforms the modified Crandall's algorithm, which itself outperforms GMP's division at larger bit-lengths.
At shorter bit-lengths it is mostly a toss-up between our method and GMP's.

We note that our code for this experiment is implemented entirely in GMP, which includes some overhead that could perhaps be avoided in an implementation closer to the metal.
This overhead is very visible when comparing \Cref{tab:hashing-experiments} and \Cref{tab:division-experiments}, suggesting that an optimized \Cref{alg:division-generalized} would beat GMP even at short bit-lengths.

