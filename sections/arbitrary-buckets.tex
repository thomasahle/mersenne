
%! TEX root = ../mersenne.tex
\section{Algorithms and analysis with arbitrary number of buckets}\label{sec:arbitrary-buckets}
We now consider the case where we want to hash into
a number of buckets. We will analyze the collision probability
with most uniform maps introduced in Section \ref{sec:most-uniform},
and later we will show how it can be used in connection with the
two-for-one hashing from Section \ref{sec:two-for-one}.

\subsection{Two-for-one hashing from uniform bits to arbitrary number of buckets}
We have a hash function $h:U\to Q$, but we want hash values in $R$, so
we need a map $\mu:Q\to R$, and then use $\mu\circ h$ as
our hash function from $U$ to $R$. We normally assume that the hash values 
with $h$ are pairwise independent, that is, for any distinct $x$ and $y$,
the hash values $h(x)$ and $h(y)$ are independent, but then 
$\mu(h(x))$ and $\mu(h(y))$ are also independent. This means
that the collision probability can be calculated
as 
\[\Pr[\mu(h(x))=\mu(h(y))]=\sum_{i\in R}\Pr[\mu(h(x))=\mu(h(y))=i]=\sum_{i\in R}\Pr[\mu(h(x)=i)]^2.\]
This sum of squared probabilities attains is minimum value $1/|R|$
exactly when $\mu(h(x))$ is uniform in $R$. 

Let $q=|Q|$ and $r=|R|$. Suppose that $h$ is $2$-universal. Then
$h(x)$ is uniform in $Q$, and then we get the lowest collision
probability with $\mu\circ h$ if $\mu$ is most uniform as defined in
Section \ref{sec:most-uniform}, that is, the number of elements from
$Q$ mapping to any $i\in[r]$ is either $\floor{q/r}$ or
$\ceil{q/r}$. To calculate the collision probability,
Let $a\in[r]$ be such that $r$ divides $q+a$. Then the map $\mu$ maps
$\ceil{q/r}=(q+a)/r$ balls to $r-a$ buckets and
$\floor{q/r}=(q+a-r)/r$ balls to $a$ buckets. For a key $x\in [u]$, we
thus have $r-a$ buckets hit with probability $(1+a/q)/r$ and
$a$ buckets hit with probability $(1-(r-a)/q)/r$.
The collision probability is then
\begin{equation}\begin{split}
   \Pr[\mu(h(x))=\mu(h(y))]
                  &= (r-a)((1+a/q)/r)^2+a((1-(r-a)r/q)/r)^2
%                 \\&=\frac{(r-a)+(r-a)2a/q+(r-a)a^2/q^2+ a-a^2(r-a)/p+a(r-a)^2/q^2}{r^2}
%                 \\&=\frac{r +r a (r-a)/q^2}{r^2}
                 \\&=(1+a(r-a)/q^2)/r
                 \\&\le \left(1+(r/(2q))^2\right)/r.\label{eq:coll-a}
\end{split}\end{equation}
Note that the above calculation generalizes the one for \req{eq:coll} which
had $a=1$. We will think of $(r/(2q))^2$ as the general relative rounding
cost when we do not have any information about how $r$ divides $q$.

\subsection{Two-for-one hashing from uniform bits to arbitrary number of buckets}
We will now briefly discuss how would get the two-for-one hash
functions in count sketches with an arbitrary number $r$ of buckets based
on a single $4$-universal hash function $h:[u]\to [2^b]$.  We want to
construct the two hash functions $s:[u]\to\{-1,+1\}$ and
$i:[u]\to[r]$. As usual the results with uniform $b$-bit strings will
set the bar that we later compare with when from $h$ we get hash values that
are only uniform in $[2^b-1]$.

The construction of $s$ and $i$ is presented in 
Algorithm \ref{alg:b-bit-arb-r}.
\begin{algorithm}[H]
   \caption{For key $x\in [u]$, compute $i(x)=i_x\in[r]$ and $s(x)=s_x\in\{-1,+1\}$.
   \newline
    Uses 4-universal $h:[u]\to [2^b]$.}
   \label{alg:b-bit-arb-r}
   \begin{algorithmic}
      \State $h_x\gets h(x)$
      \Comment $h_x$ has $b$ uniform bits
      \State $j_x\gets h_x\andtt(2^{b-1}-1)$
      \Comment $j_x$ gets $b-1$ least significant bits of $h_x$
      \State $i_x\gets (r*j_x)\rs (b-1)$
      \Comment $i_x$ is most uniform in $[r]$
      \State $a_x\gets h_x\rs (b-1)$
      \Comment $a_x$ gets the most significant bit of $h_x$
      \State $s_x\gets (a_x\ls 1)-1$
      \Comment $s_x$ is uniform in $\{-1,+1\}$ and independent of $i_x$.
   \end{algorithmic}
\end{algorithm}
The difference relative to Algorithm \ref{alg:h-and-s} is the computation
of $i_x$ where we now first pick out the $(b-1)$-bit string $j_x$ from
$h_x$, and then apply the most uniform map $(rj_x)\rs (b-1)$
to get $i_x$. This does not affect $s_x$ which remains independent
of $i_x$. 

We now have to study the effect on our estimate error
\[Y=X-F_2\sum_{x,y\in[u],x\neq y} s_x f_x[i_x=i_y]s_y f_y.\]
The expectation is exactly as before. The only
change is in the analysis of the variance where
we before had each $i_x$ uniform in $[r]=[2^\ell]$, hence
$\Pr[i_x=i_y]=1/r$. Now have values uniform in $[2^{b-1}]$ mapped
most uniformly to $[r]$ for an arbitrary $r$. Then by \req{eq:coll-a},
we get $\Pr[i_x=i_y]=\left(1+(r/(2q))^2\right)/r$ where $q=2^{b-1}$. 
This increases
the variance bound accordingly from $2F_2^2/r$ to
\begin{equation}\label{eq:Var-b-bit-arb-r}
   \Var[X]=\Var[Y]\leq 2F_2^2\left(1+(r/2^b)^2\right)/r.
\end{equation}

\subsection{Two-for-one hashing from Mersenne primes to arbitrary number of buckets}
We will now show how wan get the two-for-one hash functions in count
sketches with an arbitrary number $r$ of buckets based on a single
$4$-universal hash function $h:[u]\to [2^b-1]$.  Again we want to
construct the two hash functions $s:[u]\to\{-1,+1\}$ and
$i:[u]\to[r]$.  The construction will be the same as we had in
Algorithm \ref{alg:b-bit-arb-r} when $h$ returned uniform values in
$[2^b]$ with the change that we set $h_x\gets h(x)+1$, so that it
becomes uniform in $[2^b]\setminus\{0\}$. It is also convenient to
swap the sign of the sign-bit $s_x$ setting $s_x\gets 2a_x+1$ instead
of $s_x\gets 1-2a_x$. The basic reason is that we have swapped the
role of \texttt{0} and \texttt{1} in $a_x$.  The resulting algorithm
is presented as Algorithm \ref{alg:Mersenne-arb-r}.
\begin{algorithm}[H]
   \caption{For key $x\in [u]$, compute $i(x)=i_x\in[r]$ and
      $s(x)=s_x\in\{-1,+1\}$.\rule{5ex}{0ex}
   Uses 4-universal $h:[u]\to [p]$ for Mersenne prime $p=2^b-1\geq u$.}
   \label{alg:Mersenne-arb-r}
   \begin{algorithmic}
      \State $h_x\gets h(x)+1$
      \Comment $h_x$ uses $b$ bits uniformly except $h_x\neq 0$
      \State $j_x\gets h_x\andtt(2^{b-1}-1)$
      \Comment $j_x$ gets $b-1$ least significant bits of $h_x$
      \State $i_x\gets (r*j_x)\rs (b-1)$
      \Comment $i_x$ is quite uniform in $[r]$
      \State $a_x\gets h_x\rs (b-1)$
      \Comment $a_x$ gets the most significant bit of $h_x$
      \State $s_x\gets 1-(a_x\ls1)$
      \Comment $s_x$ is quite uniform in $\{-1,+1\}$ and quite independent of $i_x$.
   \end{algorithmic}
\end{algorithm}
The rest of Algorithm \ref{alg:Mersenne-arb-r} is exactly like 
Algorithm \ref{alg:b-bit-arb-r}, and we will now discuss the new
distributions of the resulting variables. We had
$h_x$ uniform in $[2^b]\setminus\{0\}$, and then we set
$j_x \gets h_x\andtt(2^{b-1}-1)$. Then $j_x\in[2^{b-1}]$ with 
$\Pr[j_x=0]=1/(2^{b}-1)$ while  $\Pr[j_x=j]=2/(2^{b}-1)$ for all $j>0$.

Next we set $i_x\gets (rj_x)\rs b-1$. We know from Lemma
\ref{lem:most-uniform} (i) that this is a most uniform map from
$[2^{b-1}]$ to $[r]$.  It maps a maximal number of elements from
$[2^{b-1}]$ to $0$, including $0$ which had half probability for
$j_x$.
We conclude
\begin{align}
   \Pr[i_x=0] &= (\ceil{2^{b-1}/r}2-1)/(2^{b}-1)
   \label{eq:prix0}
   \\
   \Pr[i_x = i] &\in
   \{\floor{2^{b-1}/r}2/(2^{b}-1), \ceil{2^{b-1}/r}2/(2^{b}-1)\}
   \mbox{ for $i \neq  0$}
   \label{eq:prixneq0}
   .
\end{align}
%all $i\in[r]\setminus\{0\}$ have probability
%while
%$0$ has probability $(\ceil{2^{b-1}/r}2-1)/(2^{b}-1)$. 
We note
that the probability for $0$ is in the middle of the two other
bounds and often this yields a more uniform distribution on $[r]$ than
the most uniform distribution we could get from the
uniform distribution on $[2^{b-1}]$.

With
more careful calculations, we can get some nicer bounds
that we shall later.
\begin{lemma}\label{lem:ix-r-dist} For any distinct $x,y\in [u]$, 
   \begin{align}
      \Pr[i_x=0]&\le(1+r/2^b)/r\label{eq:ix=0}\\
      \Pr[i_x=i_y]&\leq \left(1+(r/2^b)^2\right)/r.\label{eq:ix=iy}
   \end{align}
\end{lemma}
\begin{proof}
   The proof of \req{eq:ix=0} is a simple calculation.
   Using \req{eq:prix0} and the fact $\ceil{2^{b-1}/r}\le(2^{b-1}+r-1)/r$ we have
   \begin{align*}
      \Pr[i_x=0]&\le (2(2^{b-1}+r-1)/r)-1)/(2^{b}-1)\\
                %&=((2^b+r-2)/r)/(2^b-1)\\
                &=\left(1+(r-1)/(2^b-1)\right)/r\\
                &\le\left(1+r/2^b\right)/r.
   \end{align*}
   The last inequality follows because $r<u<2^b$.

   % The proof of \ref{eq:ix=iy} follows from {\color{red}Thomas}.
   For \ref{eq:ix=iy},
   let $q=2^{b-1}$ and $p=1/(2q-1)$.
   We define $a\ge 0$ to be the smallest integer, such that $r\setminus q+a$.
   In particular this means 
   $\lceil q/r\rceil = (q+a)/r$ and
   $\lfloor q/r\rfloor = (q-r+a)/r$.

   We bound the sum
   $$
   \Pr[i_x=i_y]
   = \sum_{k=0}^{r-1} \Pr[i_x = k]^2
   $$
   by splitting into three cases:
   1) The case $i_x=0$, where $\Pr[i_x=0]=(2\ceil{q/r}-1)p$,
   2) the $r-a-1$ indices $j$ where $\Pr[i_x=j]=2\ceil{q/r}p$,
   and 3) the $a$ indices $j$ st. $\Pr[i_x=j]=2\floor{q/r}p$.
   \begin{align*}
      \Pr[i_x=i_y]
   &=
   (2p\ceil{ q/r}-p)^2 + (r-a-1) (2p \lceil q/r\rceil)^2 + (r-a) (2p \lfloor q/r \rfloor)^2
 \\&= ((4a+1)r+4(q+a)(q-a-1))p^2/r
 \\&\le (1 + (r^2-r)/(2q-1)^2) / r.
   \end{align*}
   The last inequality comes from maximizing over $a$, which yields $a=(r-1)/2$.

   The result now follows from
   \begin{align}
      (r^2-r)/(2q-1)^2
      \le
      (r-1/2)^2/(2q-1)^2
      \le
      (r/(2q))^2,
   \end{align}
   which holds exactly when $r\le q$.



\end{proof}
Lemma \ref{lem:ix-r-dist} above is all we need to know about the
marginal distribution of $i_x$. However, we also need a replacement
for Lemma \ref{lem:remove-si} for handling the sign-bit $s_x$.
\begin{lemma}\label{lem:remove-si-r-dist} Consider distinct keys
   $x_1,\ldots,x_j$, $j\leq k$ and an expression $B=s_{x_1}A$ where $A$
   depends on $i_{x_1},\ldots,i_{x_j}$ and $s_{x_2},\ldots,s_{x_j}$ but not
   $s_{x_1}$. 
   Then
   \begin{equation}\label{eq:remove-si-r-dist}
      \E[s_xA]=\E[A\suchthat i_x=0]/p.
   \end{equation}
\end{lemma}
\begin{proof}
   The proof follows the same idea as that for Lemma \ref{lem:remove-si}.
   First we have
   \[\E[B]=\E_{h(x_1) \sim \unif([2^b]\setminus\{0\})}[B]=\E_{h(x_1) \sim \unif[2^b]}[B]2^b/p-\E[B \mid h(x_1)=0]/p.\]
   With $h(x_1)\sim \unif[2^b]$, the bit $a_{x_1}$ is uniform and 
   independent of $j_{x_1}$, so $s_{x_1}\in\{-1,+1\}$ is uniform and 
   independent of $i_{x_1}$, and therefore 
   \[\E_{h(x_1) \sim \unif[2^b]}[s_{x_1}A]=0.\]
   Moreover, $h(x_1)=0$ implies $j_x={x_1}$, $i_{x_1}=0$, $a_{x_1}=0$,
   and $s_{x_1}=-1$,
   so 
   \[\E[s_{x_1}A]=-\E[s_{x_1}A \mid h(x_1)=0]/p=\E[A \mid i_{x_1}=0].\]
\end{proof}
We now consider our $F_2$ estimator
\[X=\sum_{i\in[r]}\left( \sum_{x\in[u]}s_x f_x[i_x=i]\right)^2\!
   =F_2+Y\mbox{ where }Y=\sum_{x,y\in[u],x\neq y}
s_x f_x[i_x=i_y]s_y f_y.\]
Using Lemma \ref{lem:remove-si-r-dist} instead of Lemma
\ref{lem:remove-si-r-dist} we get an almost identical calculation 
of the expectation $\E[Y]$ as
in Section \ref{sec:two-for-one} and with the same
conclusion that 
\begin{equation}\label{eq:E-Mersenne-arb-r}
   \E[Y]=(F_1^2-F_2)/p^2\leq (n-1)F_2/p^2.
\end{equation}
Therefore we still have the same bound \req{eq:E-F2-p} as in Theorem \ref{thm:h-and-s-p}.

Concerning the variance, we have some more changes to the 
calculations in Section \ref{sec:two-for-one}.
We still start with
\[\Var[X]=\Var[Y]\leq \E[Y^2]=
   \sum_{x,y,x',y'\in[u],x\neq y,x'\neq y'}\E[(s_x f_x[i_x=i_y]s_y f_y)
(s_{x'}f_{x'}[i_{x'}=i_{y'}]s_{y'} f_{y'})].\] 
For terms with 4 distinct keys, we still get the same expectation
as in Section \ref{sec:two-for-one}, so their
total contribution to the variance is still the $F_2^2 n^2/p^4$ from
\req{eq:distinct}

However, for the terms with two unique keys and one pair of identical
keys, the factor $\E[i_x=r-1]<1/r$ gets replaced with $\E[i_x=0]$
which by \req{eq:ix=0} in Lemma \ref{lem:ix-r-dist} is bound by
$(1+r/2^b)/r$. As a result, for the total contribution of these terms,
we have to multiply the $4 F_2^2 n/(rp^2)$ from \req{eq:one-pair} by
$(1+r/2^b)$, so they now sum up to
\begin{equation}\label{eq:one-pair'}
   4 (1+r/2^b) F_2^2 n/(rp^2)
\end{equation}
Finally, for the terms with two pairs of identical keys, $\Pr[i_x=i_y]$ was bounded
by $(1+1/p)/r$, which is replaced by the bound $\left(1+(r/2^b)^2\right)/r$, so
so 
\begin{align}
   2\sum_{x,y\in[u],x\neq y}&(f_x^2f_y^2)\Pr[i_x=i_y]\nonumber\\
                            &=2\sum_{x,y\in[u],x\neq y}(f_x^2f_y^2)(1+(r/2^b)^2)/k\nonumber\\
                            &=2(F_2^2-F_4)(1+(r/2^b)^2)/k\nonumber\\
                            &\leq 2(1-1/n)F_2^2(1+(r/2^b)^2)/k\label{eq:two-pairs'}
\end{align}
Adding it all up, we have proved that 
\[\Var[Y]\leq F_2^2 n^2/p^4+2(1-1/n)F_2^2(1+(r/2^b)^2)/r+4(1+r/2^b)F_2^2 n/(rp^2).\]
We want the argue that we get the same variance bound as we had in 
\req{eq:Var-b-bit-arb-r} with uniform $b$-bit hash values; namely that
\begin{equation}\label{eq:Var-Mersenne-arb-r}
   \Var[Y]\leq 2\left(1+(r/2^b)^2\right)F_2^2/r.
\end{equation}
which follows if 
\[ru^3/p^4+4(1+r/2^b)n^2/p^2\leq 2.\]
Recall that $2\leq k<u\leq (p+1)/2$. Since $u$ is a power of two,
$u\geq 4$. We conclude $n/p\leq u/p\leq 4/7$, $r/p\leq 3/7$, and
$p+1\geq 8$. Therefore the left-hand side is bounded by
$(3/7)(4/7)^3+4(1+2/8)(4/7)^2<1.8<2$.  This completes the proof of
\req{eq:Var-Mersenne-arb-r}. The basic conclusion is that both with
$r$ is a power of two and when $r$ is arbitrary, we get the same
variance bounds with Mersenne primes as we did with uniform $b$-bit
strings. The only difference is the small error in expectation from
\req{eq:E-Mersenne-arb-r}. Relative to the correct value $F_2$, the
relative error in the expectation is by a factor of at most
$n/p^2$ which is insignificant when $p$ is large.


