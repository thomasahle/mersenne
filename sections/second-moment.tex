


\section{Analysis of second moment estimation using  Mersenne primes}\label{sec:analysis-two-for-one}
In this section, we will prove Theorem \ref{thm:h-and-s-p}.  Thus we
have $r=2^\ell>1$ and $u>r$ both powers of two and $p=2^b-1>u$ a
Mersenne prime.

Now let us set up the relevant variables for the basic count sketch.
For each key $x\in [u]$, we have a value $f_x\in \Z$, and the
goal was to estimate the second moment $F_2=\sum_{x\in u}f^2$.

We had two functions $i:[u]\fct[r]$ and $s:[u]\fct\{-1,+1\}$. 
For notational convinience, we define $i_x=i(x)$ and $s_x=s(x)$.

For each $i\in [r]$, we have a counter 
$C_i=\sum_{x\in[u]} s_x f_x[i_x=i]$, and we define the 
estimator $X=\sum_{i\in[k]} C_i^2$. We want to study how
well it approximates $F_2$.
We have 
\begin{align*}
X&=\sum_{i\in[r]}\left( \sum_{x\in[u]}s_x f_x[i_x=i]\right)^2\\
&=\sum_{i\in[r]}\sum_{x,y\in[u]}s_x f_x[i_x=i=i_y]s_y f_y\\
&=\sum_{x,y\in[u]}s_x f_x[i_x=i_y]s_y f_y.\\
&=\sum_{x\in[u]} f_x^2+\sum_{x,y\in[u],x\neq y}
s_x f_x[i_x=i_y]s_y f_y.
\end{align*}
Thus 
\begin{equation}\label{eq:decomp}
X=F_2+Y\mbox{ where }Y=\sum_{x,y\in[u],x\neq y}
s_x f_x[i_x=i_y]s_y f_y.
\end{equation}
We thus want to provide bounds on the error $Y$.

\subsection{Analysis in the classic case}
First, as a warm-up for later comparison, we analyze the simple classic case
where $i:[u]\fct[2^\ell]$ and
$s:[u]\fct\{-1,1\}$ are independent 4-universal hash
functions. In this case, we first argue that 
\begin{equation}\label{eq:mean}
\E[Y]=0.
\end{equation}
To prove \req{eq:mean}, we argue that each term of
$Y$ from \req{eq:decomp} has 0 expected value. With $x\neq y$, by 2-universality (implied by 4-universality) and independence of $s$ and $h$, we have that $s_x$ is independent of $s_y$, $i_x$, and $i_y$.
Moreover, $\E[s_x]=0$, 
so 
\[\E[s_x f_x[i_x=i_y]s_y f_y]=\E[s_x]\E[f_x[i_x=i_y]s_y f_y]=0.\]
It follows that $\E[Y]=0$, completing the proof of \req{eq:mean}.

Next we want to bound the variance of $X$ which is the same as
that of $Y$ since $F_2$ is a constant. Since
$\E[Y]=0$, we have
\begin{align*}
\Var[X]&=\Var[Y]=\E[Y^2]=\E[(\sum_{x,y\in[u],x\neq y}s_x f_x[i_x=i_y]s_y f_y)^2]\\
&=\sum_{x,y,x',y'\in[u],x\neq y,x'\neq y'}\E[(s_x f_x[i_x=i_y]s_y f_y)
(s_{x'}f_{x'}[i_{x'}=i_{y'}]s_{y'} f_{y'})].
\end{align*}
Consider one of the terms $\E[(s_x f_x[i_x=i_y]s_y f_y)
(s_{x'}f_{x'}[i_{x'}=i_{y'}]s_{y'} f_{y'})]$. We have $x\neq y,x'\neq y'$. 
Suppose that any one of the keys,
say $x$, is unique, that is, $x\not\in\{y,x',y'\}$. Then, by
4-universality, $s_x$ is independent of $s_y$, $s_{x'}$, and $s_{y'}$.
Morever, it is independent of the hash function $h$. Since $\E[s_x]=0$,
we get
\begin{align*}
&\E[(s_x f_x[i_x=i_y]s_y f_y)
(s_{x'}f_{x'}[i_{x'}=i_{y'}]s_{y'} f_{y'})] \\
=&\E[s_x]\E[(f_x[i_x=i_y]s_y f_y) 
(s_{x'}f_{x'}[i_{x'}=i_{y'}]s_{y'} f_{y'})]=0.
\end{align*}
Thus we can restict our attention to terms with no unique keys. 
Since $x\neq y$ and $x'\neq y'$, we conclude that 
$(x,y)=(x',y')$ or $(x,y)=(y',x')$. Therefore
\begin{align*}
\Var[X]&=\sum_{x,y,x',y'\in[u],x\neq y,x'\neq y'}\E[(s_x f_x[i_x=i_y]s_y f_y)
(s_{x'}f_{x'}[i_{x'}=i_{y'}]s_{y'} f_{y'})]\\
       &=2\sum_{x,y\in[u],x\neq y,(x',y')=(x,y)}\E[(s_x f_x[i_x=i_y]s_y f_y)
(s_{x'}f_{x'}[i_{x'}=i_{y'}]s_{y'} f_{y'})]\\
&=2\sum_{x,y\in[u],x\neq y}\E[(s_x f_x[i_x=i_y]s_y f_y)^2]\\
&=2\sum_{x,y\in[u],x\neq y}\E[(f_x^2f_y^2[i_x=i_y])]\\
&=2\sum_{x,y\in[u],x\neq y}(f_x^2f_y^2)/r\\
&=2(F_2^2-F_4)/r.
\end{align*}
This completes the proof of \req{eq:V-F2}. In the above analysis, we
did not need $s$ and $i$ to be completely independent. All we needed
was that for any $j\leq 4$ distinct keys $x_1,\ldots,x_j$, the hash
values $s(x_1),\ldots,s(x_j)$ and $i(x_1),\ldots,i(x_j)$ are all
independent and uniform in the desired domain. This was why we could
use a single $k$-universal hash function $h:[u]\fct[2^b]$ with
$b>\ell$, and use it to construct $s:[u]\fct\{-1,+1\}$ and
$i:[u]\fct[2^\ell]$ as described in Algorithm \ref{alg:h-and-s}
(c.f. Lemma \ref{lem:b-bit-hashing}).




\subsection{The real analysis of two-for-one using Mersenne primes}
We now consider the case where the functions $s:[u]\fct\{-1,+1\}$ and
$i:[u]\fct[2^\ell]$ are constructed as described in Algorithm
\ref{alg:h-and-s} from a single $k$-universal hash function
$h:[u]\fct[2^b-1]$. Here $h$ could be implemented efficiently as
in Algorithm \ref{alg:Mersenne}. As noted earlier, the function
$i$ is the same as $h^r$ with $r=2^\ell$ in Section \ref{sec:power-of-two},
so it satisfies \req{eq:coll-ell<r-1}--\req{eq:coll}. Also, we
have the power of Lemma \ref{lem:remove-si} to handle the correlation
between $s$ and $i$.

As above, for notational convinience, we let $s_x$ denote $s(x)$ and $i_x$ denote $i(x)$. Also, recall that $n\leq u$ is the number of non-zero values
$f_x\neq 0$, $x\in [u]$.



\paragraph{Expectancy}
Recall from \req{eq:decomp} that
we have
\[X=F_2+Y\mbox{ where }Y=\sum_{x,y\in[u],x\neq y} s_x f_x[i_x=i_y]s_y f_y.\]
This equality holds regardless of random choices, hence regardless of
how $h$ and $s$ are implemented.

To bound $\E[Y]$ we study the expectancy of an arbitrary term, that is,
for some $x,y\in[u],x\neq y$, we study $\E[ s_x f_x[i_x=i_y]s_y f_y]$.
Using Lemma \ref{lem:remove-si}, we get 
\begin{align}
\E[s_x f_x[i_x=i_y]s_y f_y]&=\E[f_x[i_x=i_y]s_y f_y\suchthat i_x=r-1]/p.\nonumber\\
&=\E[s_y f_x[r-1=i_y]f_y]/p.\nonumber\\
&=\E[f_x[r-1=i_y]f_y\suchthat i_y=r-1]/p^2.\nonumber\\
&=\E[f_xf_y]/p^2=f_x f_y/p^2.\label{eq:E}
\end{align}
Thus
\[\E[Y]=\sum_{x,y\in[u],x\neq y} f_xf_y/p^2=(F_1^2-F_2)/p^2.\]
Trivially is the equality from \req{eq:E-F2-p}. For the upper bound, we note that 
\begin{equation}\label{eq:F1F2}
F_1^2\leq nF_2 
\end{equation}
This follows because the ``variance'' over the $|f_x|$ is
$F_2-F_1^2/n=\sum_{x\in[u], f_x\neq 0}(f_x-F_1/n)^2\geq 0$.
Therefore
\[\E[Y]=(F_1^2-F_2)/p^2\leq F_2n/p^2.\]
This completes the proof of \req{eq:E-F2-p}.

\paragraph{Variance}
Same method is applied to the analysis of the variance, which is
\[\Var[X]=\Var[Y]\leq \E[Y^2]=
\sum_{x,y,x',y'\in[u],x\neq y,x'\neq y'}\E[(s_x f_x[i_x=i_y]s_y f_y)
  (s_{x'}f_{x'}[i_{x'}=i_{y'}]s_{y'} f_{y'})].\] 
Consider any term in the
sum. Suppose some key, say $x$, is unique in the sense that
$x\not\in\{y,x',y'\}$. Then we can apply Lemma \ref{lem:remove-si}.
Given that $x\neq y$
and $x'\neq y'$, we have either $2$ or $4$ such unique keys. If all
4 keys are distinct, as in \req{eq:E}, we get
\begin{align*}
\E[(s_x f_x[i_x=i_y]s_y f_y)&
(s_{x'}f_{x'}[i_{x'}=i_{y'}]s_{y'} f_{y'})]\\
&=\E[(s_x f_x[i_x=i_y]s_y f_y])]\E[(s_{x'}f_{x'}[i_{x'}=i_{y'}]s_{y'} f_{y'})]\\
&=(f_x f_y/p^2)(f_{x'} f_{y'}/p^2)=f_x f_yf_{x'} f_{y'}/p^4.
\end{align*}
The expected sum over all such terms is thus bounded
as 
\begin{align}
\sum_{{\rm distinct}\, x,y,x',y'\in[u]}&
\E[(s_x f_x[i_x=i_y]s_y f_y)
(s_{x'}f_{x'}[i_{x'}=i_{y'}]s_{y'} f_{y'})]\nonumber\\
&=\sum_{{\rm distinct}\,x,y,x',y'\in[u]}
f_xf_yf_{x'}f_{y'}/p^4<F_1^4/p^4\leq F_2^2 n^2/p^4.\label{eq:distinct}
\end{align}
The last inequalities 
used \req{eq:F1F2}. We also have to consider all the cases with two unique keys, e.g., $x$ 
and $x'$ unique while $y=y'$. Then using Lemma \ref{lem:remove-si} and \req{eq:coll-ell=r-1}, we get
\begin{align*}
\E[(s_x f_x[i_x=i_y]s_y f_y)&(s_{x'}f_{x'}[i_{x'}=i_{y'}]s_{y'} f_{y'})]\\
&=f_xf_{x'}f_y^2\E[s_xs_{x'} [i_x=i_{x'}=i_y]]\\
&=f_xf_{x'}f_y^2\E[s_{x'} [r-1=i_{x'}=i_y]]/p\\
&=f_xf_{x'}f_y^2\E[r-1=i_y]/p^2\\
    &<f_xf_{x'}f_y^2/(rp^2).
\end{align*}    
Summing over all terms with $x$ and $x'$ unique while $y=y'$, and
using \req{eq:F1F2} and $u\leq p$, we get 
\begin{align*}
  \sum_{{\rm distinct}\,x,x',y} f_xf_{x'}f_y^2 /(rp^2)&<
    F_1^2F_2/(rp^2)\leq F_2^2 n/(rp^2).
\end{align*}
There are four ways we can pick the two unique keys $a\in \{x,y\}$
and $b\in \{x',y'\}$, so we conclude that
\begin{equation}\label{eq:one-pair}
\sum_{\begin{array}{c}x,y,x',y'\in[u], x\neq y, x'\neq y',\\
{\rm two\ keys\ are\ unique}\end{array}}
\E[(s_x f_x[i_x=i_y]s_y f_y)
(s_{x'}f_{x'}[i_{x'}=i_{y'}]s_{y'} f_{y'})]\leq 4 F_2^2 n/(rp^2)
\end{equation}
Finally, we need to reconsider the terms with two pairs, that
is where $(x,y)=(x',y')$ or $(x,y)=(y',x')$. In
this case, $(s_x f_x[i_x=i_y]s_y f_y)
(s_{x'}f_{x'}[i_{x'}=i_{y'}]s_{y'} f_{y'})=f_x^2f_y^2[i_x=i_y]$.
By \req{eq:coll-ell<r-1} and \req{eq:coll-ell=r-1},  we 
get 
\begin{align}
\sum_{\begin{array}{c}x,y,x',y'\in[u], x\neq y, x'\neq y',\\
(x,y)=(x',y')\,\vee\,(x,y)=(y',x')\end{array}}&
\E[(s_x f_x[i_x=i_y]s_y f_y)
(s_{x'}f_{x'}[i_{x'}=i_{y'}]s_{y'} f_{y'})]\nonumber\\
&=2\sum_{x,y\in[u],x\neq y}(f_x^2f_y^2)\Pr[i_x=i_y]\nonumber\\
&=2\sum_{x,y\in[u],x\neq y}(f_x^2f_y^2)(1+r/p^2)/r\nonumber\\
&=2(F_2^2-F_4)(1+r/p^2)/r\label{eq:two-pairs}
\end{align}
Adding up add \req{eq:distinct}, \req{eq:one-pair}, and
\req{eq:two-pairs}, we get 
\[\Var[Y]\leq 2(F_2^2-F_4)/r+F_2^2 (n^2/p^4+2/p^2+4 n/(rp^2)).\]
We have defined our parameters to satisfy $2\leq r\leq u/2\leq (p+1)/4$
where $r$, $u$, and $p+1$ are all powers of two. Also $n\leq u$, so it 
follows that
\begin{equation}
n/p\leq u/n\geq\geq 7/4\textnormal{ and }p/r\geq 7/2.\label{eq:ratios}
\end{equation}
In particular, $(n/p)^2\leq(4/7)^2<0.33$, so we conclude
that
\begin{equation}\label{eq:var-good1}
\Var[Y]<2(F_2^2-F_4)/r+F_2^2 (2.33+4 n/r)/p^2.
\end{equation}
Next we note that $F_4\geq F_2^2/n$ follows from \req{eq:F1F2} applied
to vector of squared values $(f_0^2,\ldots,f_{u-1}^2)$. Combined
with \req{eq:var-good1} and \req{eq:ratios}, we get
\begin{align*}
\Var[Y]&<2(F_2^2-F_4)/r+F_2^2 (2.33+4 n/r)/p^2\\
&\leq 2F_2^2/r+F_2^2(-2+2.33nr/p^2+4 n^2/p^2)\\
&<2F_2^2/r.
\end{align*}
This completes the proof of \req{eq:V-F2-p}  and hence
of Theorem \ref{thm:h-and-s-p}.
