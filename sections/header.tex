%! TEX root = ../mersenne.tex
\documentclass[11pt, a4paper]{article}
%\usepackage[a4paper,
%            inner=20mm,
%            outer=50mm,% = marginparsep + marginparwidth
%                       %   + 5mm (between marginpar and page border)
%            top=20mm,
%            bottom=25mm,
%            marginparsep=5mm,
%            marginparwidth=60mm,
%            showframe% for show your page design, normaly not used
%            ]{geometry}
\usepackage{amsmath}
\usepackage{amsthm}
\usepackage{amsfonts}
\usepackage{amssymb}

\usepackage{tikz}
\usetikzlibrary{patterns}

\usepackage{authblk}
\usepackage{fullpage}
\usepackage{todonotes}
\usepackage{hyperref}
\usepackage{algorithm}
\usepackage[noend]{algpseudocode}
\usepackage{nameref}
\usepackage{cleveref}
%\usepackage[ruled,vlined,commentsnumbered,titlenotnumbered]{algorithm2e}

\DeclareMathOperator*{\E}{E}
\DeclareMathOperator*{\Var}{Var}

\newcommand{\floor}[1]{\lfloor {#1} \rfloor}
\newcommand{\bfloor}[1]{\big\lfloor {#1} \big\rfloor}
\newcommand{\bbfloor}[1]{\bigg\lfloor {#1} \bigg\rfloor}
\newcommand{\ceil}[1]{\lceil {#1}\rceil}
\newcommand{\Prp}[1]{\Pr\left[{#1} \right]}
\newcommand{\Ep}[1]{{\E}\left[{#1} \right]}
\newcommand\eps\varepsilon
\newcommand\Z{\mathbb Z}

\newtheorem {lemma} {Lemma}[section]
\newtheorem {fact} [lemma] {Fact}
\newtheorem {property} {Property}
\newtheorem {definition} {Definition}
\newtheorem {corollary} [lemma] {Corollary}
\newtheorem {theorem}[lemma] {Theorem}
\newtheorem {observation}[lemma] {Observation}
\newtheorem {question}{Question}[section]
\newtheorem {exercise}[question]{Exercise}

\newcommand{\unif}{\mathcal{U}}

% Drawing
\newcommand{\andtt}{ \mathbin{\texttt{\&}} }
\newcommand{\xor}{\oplus}
\newcommand{\ls}{ \mathbin{\texttt{<\!<}} }
\newcommand{\rs}{ \mathbin{\texttt{>\!>}} }
